
%%% Local Variables:
%%% mode: latex
%%% TeX-master: "mechano"
%%% End:

\section{Algorithmische Grundlagen}%
\label{ag:sec:algorithmtische_grundlagen}

\subsection{Vorwärtskinematik}%
\label{ag:sub:vorwaertskinematik}
Wo ist meine Hand? \(\rightarrow\) Hier!\\

Bestimmt die Position des Endeffektors abhängig von den Gelenkwinkeln.

\subsection{Inverse Kinematik}%
\label{ag:sub:inverse_kinematik}
Bestimmt die Gelenkwinkel zu einer gegeben Endeffektor Pose

\subsection{Jacobi Matrix}%
\label{ag:sub:jacobi_matrix}
\begin{align*}
  \dot{x} = J(\theta) \cdot \dot{\theta} \qquad \text{Vorwärtskinematik}\\
  \dot{\theta} = J^{-1} (\theta) \cdot \dot{x} \qquad \text{Inverse Kinematik}
\end{align*}

\subsection{Arbeitsraum}%
\label{ag:sub:arbeitsraum}
Der Arbeitsraum \(W\) des Roboters ist in kartesischen Koordinaten beschrieben \(\realnumbers^6\)

\subsection{Konfigurationsraum}%
\label{ag:sub:konfigurationsraum}
Der Konfigurationsraum \(C\) besteht aus kollisionsfreiem \(C_{\mathit{free}}\) und kollidierendem \(C_\mathit{obs}\) Raum.
\[C = C_\mathit{free} \cup C_\mathit{obs}\]

\subsection{Graph Algorithmen}%
\label{ag:sub:graph_algorithmen}
Graph \(G = (V, E)\) als Vertices \(V\), Kanten \(E\) und ggf. Gewichten \(W = {w_1, \ldots, w_k}\)

\subsubsection{A*}%
\label{ag:ssub:a*}
\begin{itemize}
\item best-first Suchalgorithmus
\item Optimal: Findet optimalen Pfad
\item Optimal in Bezug auf Pfad Kosten
\item Heuristic \(h\) darf Kosten nicht überschätzen \(\rightarrow h\) ist admissible
\item optimal effizient: Kein Algorithmus der weniger Knoten besucht und optimalen Pfad findet
\item Für \(h(x) = 0\) entspricht Djikstra
\end{itemize}

\subsubsection{RRT}%
\label{ag:ssub:rrt}
Verschiedene Formen
\begin{itemize}
\item Single Tree
\item Bidirectional (2 Trees)
\end{itemize}
Benötigt eine Nearest Neighbour Suche. Deshalb kd-Bäume relevant

\subsubsection{kd-Bäume}%
\label{ag:ssub:kd-baeume}
Füge \enquote{cutting Dimensions} ein. An Hand dieser lässt sich absteigen und nach Nearest Neighbours suchen:
\begin{itemize}
\item Tue so, als würde der Punkt eingefügt werden
\item Speichere beim Abstieg nächste Distanz
\item Steige auf und vergleiche ob Knoten näher, ggf.\ steige in andere Teilbäume ab
\item Führe solange durch, bis Wurzel erreicht
\end{itemize}

\subsubsection{Octrees}%
\label{ag:ssub:octrees}
Andere Raumunterteilende Datenstruktur

\subsubsection{PRM}%
\label{ag:ssub:prm}
Erstelle Roadmap für mehrere Queries.\\
Besteht aus Buildup und Query phase.\\
Selbst wenn eine Lösung existiert findet PRM sie ggf.\ nicht.