
%%% Local Variables:
%%% mode: latex
%%% TeX-master: "mechano"
%%% End:

\section{Action Representation}%
\label{ar:sec:action_representation}

\subsection{Wished properties}%
\label{ar:sub:wished_properties}
\begin{itemize}
\item Reusability
\item Flexibility
\item Adaptivity
\item Portability
\end{itemize}

Therefore we need:
\begin{itemize}
\item Formalism to describe function, e.g.\ robot movements, as dynamic system
\item Learnable from demonstration
\item Movement generalized to arbitrary start points/goals and duration
\end{itemize}

\subsection{Dynamic Movement Primitives}%
\label{ar:sub:dynamic_movement_primitives}
\textbf{Given}:
\begin{itemize}
\item Demonstration by human (position \& velocity for each time step): \(y_D, v_D\)
\item Start position \(y_0\), velocity \(v_0\)
\item Goal position \(g\)
\item duration (temporal factor) \(\tau\)
\end{itemize}
\textbf{Desired}:\\
Trajectory from \(y_0\) to \(g\) in duration \(\tau\) with characteristic shape of demonstration \(y_D\)\\
\textbf{Solution}:\\
Dynamical Systems; critically damped spring-mass system with perturbation

\subsubsection{DMPs in General}%
\label{ar:ssub:dmps_in_general}
Consist of two parts
\begin{align}
  \dot{s}(t) &= \text{Canoncial}(t, s)\\
  \dot{y}(t) &= \text{Transform}(t, y) + \text{Perturbation}(s)
\end{align}
Canonical system:
\begin{itemize}
\item State of the DMP in time
\item Drives perturbation to control transformation system
\end{itemize}

Transformation system:
\begin{itemize}
\item Generates motion of the robot according to state of canonical
\end{itemize}

\subsubsection{Damped spring-mass system}%
\label{ar:ssub:damped_spring_mass_system}
\[\ddot{x} = K(g-x)-D\dot{x}\]
Transform in 1st order system
\begin{align*}
  \dot{v} &= K(g-x) - D \cdot v\\
  \dot{x} &= v
\end{align*}
Add Temporal scaling with \(\tau \rightarrow \tau > 1\) is slower
\begin{align*}
  \tau\dot{v} &= K(g-x) - D \cdot v\\
  \tau\dot{x} &= v
\end{align*}
Add perturbation force to shape trajectory:
\begin{align*}
  \tau\dot{v} &= K(g-x) - D \cdot v + (g - x_0) \cdot f \cdot u\\
  \tau\dot{x} &= v
\end{align*}
ODE system is called the transformation system of a DMP.\\
Perturbation force learned from demonstration\\

Problem: System oscillates \(\rightarrow\) use critically damped spring-mass system

\subsubsection{Canonical system}%
\label{ar:ssub:canonical_system}
Perturbation term \(f\) depends on time \(\rightarrow\) time-variant system hard to handle\\
Solution: Use of canonical system (with phase variable \(u\))
\[\tau \dot{u} = - \alpha u\]

\subsubsection{Learning the perturbation force}%
\label{ar:ssub:learning_the_perturbation_force}
Discrete: Calculate for points with \[f = \frac{-K(g-x) + Dv + \tau\dot{v}}{(g - x_0)u}\]

Approximate to get a continuous representation

\subsection{Radial Basis Function Networks}%
\label{ar:sub:radial_basis_function_networks}
Modelled as neural network with one hidden layer and one output neuron\\
Approximate Function
\[\hat{f}(x) = w_0 + \sum_{u=1}^k w_u \cdot K_u(x)\]
with Kernel function
\[K_u (x) = e^{-\frac{1}{2\sigma^2_u} || \mu_u - x||^2}\]

\subsubsection{Training a RBF}%
\label{ar:ssub:training_a_rbf}
\begin{enumerate}
\item determine the receptive fields
\item determine the weights \(w_u\)
\end{enumerate}

\subsubsection{Weighted RBFs}%
\label{ar:ssub:weighted_rbfs}
Approximate the perturbation force\\
Locally weighted regression:
\[f_\mathit{approx.}(u) = \frac{\sum_{i=1}^N w_i \Psi_i(u)}{\sum_{i=1}^N \psi_i(u)}\]
RBF:
\[\psi_i(u) = e^{h_i (u-c_i)^2}\]

Benefits:
\begin{itemize}
\item Can remove jitter from original demonstration
\item Can reduce memory consumption
\end{itemize}

\subsection{Execution of a DMP}%
\label{ar:sub:execution_of_a_dmp}
Parameterize the DMP and integrate the ordinary differential equation

\subsection{Online Feedback}%
\label{ar:sub:online_feedback}
One way: Phase stopping: Slow down until external perturbation force is gone\\
Change canonical system
\[\tau\dot{u} = \frac{-\alpha u}{1 + \alpha_{pu}|\tilde{x} - x|}\]
\(a_{pu}\) controls how fast the phase slows down\\
Change transformation sytem
\begin{align*}
  \tau \dot{v} &= K (g-x) - Dv + (g - x_0)u\\
  \tau \dot{x} &= v + \alpha_{px}|\tilde{x} -x |
\end{align*}

Online changing the goal would lead to jump in acceleration \(\rightarrow\) filter with ODE:
\[\dot{g} = \lambda(g-g_d)\]

\subsection{Periodic DMPs}%
\label{ar:sub:periodic_dmps}
\begin{itemize}
\item Start with discrete part called transient
\item Have no fixed start
\item Transients cannot be reconstructed from the period \(\rightarrow\) have to be learned
\end{itemize}

\textbf{Changing the system}
\begin{itemize}
\item Canonical system with frequency \(\Omega = \dot{\Phi}\) 
\item \(\tau\) with \(\frac{1}{\Omega}\)
\item Perturbation force
  \begin{align*}
    f(\Phi) &= \frac{\sum_{i=1}^N w_i \Gamma_i (\phi)}{\sum_{i=1}^N \Gamma_i (\Phi)}r\\
    \Gamma_i (\Phi) &= e^{h_i (\cos(\Phi - c_i) - 1)}
  \end{align*}
\item Transformation System
  \begin{align*}
    \dot{v} &= \Omega K(g - x) - Dv + (g - x_0)f(\Phi)\Phi\\
    \dot{x} &= \Omega v
  \end{align*}
\end{itemize}

\subsection{Extension of periodic DMPs}%
\label{ar:sub:extension_of_periodic_dmps}
DMP should be able to
\begin{itemize}
\item encode the periodic movement itself and
\item all corresponding transients
\end{itemize}
into a single dynamical system\\

\subsubsection{Canonical System}%
\label{ar:ssub:canoncial_system}
Idea: 2 different basis functions
\begin{itemize}
\item outside the limit cycle (transient)
\item on the limit circle (periodic pattern)
\end{itemize}

Two-dimensional i.e. \(s(t) := (\varphi(t), r(t)) \in \realnumbers \times (0, \infty)\) for \(\varphi, r\) solution
\begin{align*}
  \dot{\Phi} &= \Omega\\
  \dot{r} &= \eta (\mu^\alpha - r^\alpha)r^\beta\\
  \Phi (0) &= \Phi_0,\ r(0) = r_0
\end{align*}
Interpret \((\varphi, r)\) as polar coordinates

\subsubsection{Transformation System}%
\label{ar:ssub:transformation_system}
We use well known transformation system from [Ijspeert et al. 2002]
\begin{align*}
  \dot{z} &= \Omega (\alpha_z (\beta_z ( g - y ) - z ) + f(\Phi, r))\\
  \dot{y} &= \Omega z
\end{align*}
\(g\) is anchor point in \(\realnumbers\)

\subsubsection{Shaping Term}%
\label{ar:ssub:shaping_term}
Extend perturbation force of [Gams et al. 2009]
\[
  f(\Phi, r) = \frac{\sum_{j=1}^M \psi_j(\Phi, r) \tilde{w}_j + \sum_{i=1}^N \varphi_i(\Phi, r)w_i}
  {\sum_{j=1}^M \psi_j(\Phi, r) + \sum_{i=1}^N \varphi_i(\Phi, r)}
\]