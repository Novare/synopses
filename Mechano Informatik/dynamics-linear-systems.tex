
%%% Local Variables:
%%% mode: latex
%%% TeX-master: "mechano"
%%% End:

\newcommand{\resolvent}{\ensuremath{\Phi}\xspace}
\newcommand{\statetrans}{\ensuremath{\phi}\xspace}

\tikzstyle{block} = [draw, fill=blue!20, rectangle, 
minimum height=3em, minimum width=6em, align=center]
\tikzstyle{sum} = [draw, fill=blue!20, circle, node distance=1cm]
\tikzstyle{input} = [coordinate]
\tikzstyle{output} = [coordinate]
\tikzstyle{pinstyle} = [pin edge={to-,thin,black}]

\section{Dynamik linearer Systeme}%
\label{dls:sec:dynamik_linearer_systeme}

\subsection{Control Theory}%
\label{dls:sub:control_theory}
(Regelungstechnik) is the theory of automatic, goal-orientated modification of dynamical processes at run-time

\subsubsection{Fundamental question}%
\label{dls:ssub:fundamental_question}
Design of a system for automatic, targeted modification of a process with incomplete knowledge of the underlying
process, and in the presence of disturbances

\subsubsection{Principle Operation}%
\label{dls:ssub:principle_operation}
The plant is to be monitored continuously, and the obtained information is used to modify the system input in such a
way that its output matches the desired output as close as possible, despite the effects of the disturbance.


\subsubsection{Structure of a control system}%
\label{dls:ssub:structure_of_a_control_system}
\begin{figure}[!h]
  \centering
  \begin{tikzpicture}[auto, node distance=2cm, >=latex']
    \node [input, name=input] {};
    \node [sum, right of=input] (sum) {};
    \node [block, right of=sum] (controller) {Controller};
    \node [block, right of=controller, node distance=3cm] (system) {Actuator};
    \node [block, right of=system, node distance=3cm, pin={[pinstyle]above:z}] (way) {Plant};
    
    \draw [->] (controller) -- (system);
    \draw [->] (system) -- node[name=y] {$y$} (way);
    \node [output, right of=way] (output) {};
    \node [block, below of=y] (measurements) {Sensors};

    \draw [draw, ->] (input) -- node {$w$} (sum);
    \draw [->] (sum) -- node {$x_d$} (controller);
    \draw [->] (way) -- node [name=x] {$x$} (output);
    \draw [->] (x) |- (measurements);
    \draw [->] (measurements) -| node[pos=0.99] {$-$}
    node [near end] {$r$} (sum);
  \end{tikzpicture}
  \vspace{2em}

  \begin{tabular}{clcl}
    \(w\) & Reference & \(x_d\) & Control error\\
    \(y\) & Control input & \(x\) & System output\\
    \(r\) & Feedback & \(z\) & Disturbance\\
  \end{tabular}
\end{figure}

Control system is a closed loop:\\
Feedback is substracted from the reference in the deviation analysis.

\subsubsection{Definition: Control System}%
\label{dls:ssub:definition_control_system}
A control system is an arrangement that continuously monitors the plant's output, computes
the deviation from a reference value and uses this error to adjust the system output to match the reference.\\
This is achieved with only limited knowledge about the plant and, especially, about the disturbance.

\subsection{Linear Differential Equations in State-Space Form}%
\label{dls:sub:linear_differential_equations_in_state_space_form}
\[\dot{x} = Ax + Bu\]
has solution 
\[x(t) = e^{A(t - \tau)} x(\tau) + \int_\tau^t e^{A(t - \lambda)} B u(\lambda) d\lambda\]
But does not hold when \(A\) is time varying.

\subsubsection{Output}%
\label{dls:ssub:output}
Output defined by
\[y = Cx \text{ (Moore)} \]
or
\[ y = Cx + Bu \text{ (Mealey)}\]

\subsubsection{Time varying dynamics}%
\label{dls:ssub:time_varying_dynamics}
Write \(\dot{x} = A(t)x\) as
\[x(t) = \statetrans(t,\tau)x(\tau)\]
Where \(\phi(t,\tau)\) is the State-Transition matrix:\\
It relates the state at time \(t\) to the state at time \(\tau\) and defines how state \(x(\tau)\) transitions
into \(x(t)\)
\begin{itemize}
\item time-invariant: \(\statetrans(t,\tau) = e^{A(t-\tau)}\)
\item time-varying: No simple expression
\end{itemize}
General solution
\begin{align*}
  x(t) = \statetrans(t, \tau)x(\tau) &+ \int_\tau^t \statetrans(t,\lambda)B(\lambda)u(\lambda)d\lambda\\
  y(t) = C(t)\statetrans(t, \tau)x(\tau) &+ \int_\tau^t C(t) \statetrans(t,\lambda)B(\lambda)u(\lambda)d\lambda
\end{align*}

\subsubsection{Properties of the State-Transition matrix}%
\label{dls:ssub:properties_of_the_state_transition_matrix}
\begin{itemize}
\item Expression of the solution to the homogeneous equation \(\dot{x} = A(t)x(t)\)
\item The time derivative is given by \[\frac{\delta\statetrans(t,\tau)}{\delta t} = A(t)\statetrans(t,\tau)
    \Leftrightarrow \dot{\statetrans} = A \statetrans\]
\item \(x(t) = \statetrans(t,t) x(t) \rightarrow \) The semi-group property: \(\statetrans(t,t) = I\)
\item When \(A\) constant matrix, then \(\statetrans(t) = e^{At}\)
\item Is never singular even if the \(A\) is singular
\end{itemize}

\subsection{Laplace Transformation}%
\label{dls:sub:laplace_transformation}
Transforms to frequency domain and enables the methodology.

\[\mathcal{L}[f(t)] = f(s) = \int_0^\infty f(t) \cdot e^{-st}dt\]
where \(s\) is called complex frequency \(s = \delta + i\omega\)

\subsubsection{Resolving the State-Transition}%
\label{dls:ssub:resolving_the_state_transition}
\(e^{At}\) called state-transition matrix\\
Laplace transform: \(\resolvent(s) = {(s \cdot I - A)}^{-1}\) called resolvent\\
Characterizes the dynamic behavior of the system\\

\textbf{Calculate State-Transition with Resolvent}
\begin{enumerate}
\item Calculate \((s \cdot I - A)\)
\item Obtain the resolvent by inverting \((s \cdot I - A)\)
\item State-transition matrix by element-wise inverse Laplace transform
\end{enumerate}

\subsection{Transfer Functions}%
\label{dls:sub:transfer_functions}
Relates Laplace transform of the output to the Laplace transform of the input\\
With output \(y(s) = C \cdot x(s)\)  the Transfer-function matrix is
\[H(s) = C \cdot \Phi (s) \cdot B = C \cdot {(s \cdot I - A)}^{-1} \cdot B\]
And impulse-response(inverse Laplace transformation of the transfer-function matrix)
\[H(t) = \mathcal{L}^{-1} [H(s)] = C \cdot e^{A \cdot t} \cdot B\]

\subsection{Transformation of state variables}%
\label{dls:sub:transformation_of_state_variables}
Sometimes other set of variables more convenient
\(\rightarrow\) Transform \(A, B, C, D\) to new set \(\overline{A}, \overline{B}, \overline{C}, \overline{D}\)\\
Change represented in linear transformation \(z=Tx\)\\
\(T\) is non-singular, so \(x = T^{-1}z\)\\
Then \[T^{-1}\dot{z} = AT^{-1}z + Bu\]