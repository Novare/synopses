
%%% Local Variables:
%%% mode: latex
%%% TeX-master: "mechano"
%%% End:

\section{Machine Learning}%
\label{ml:sec:machine_learning}

\subsection{Definition: Machine Learning}%
\label{ml:sub:definition_machine_learning}
A computer program is said to learn from experience \(E\) with respect to some class of tasks \(T\)
and performance measure \(P\), if its performance at tasks in \(T\), as measured by \(P\), improves with
Experience \(E\).

\subsection{ML Foundations}%
\label{ml:sub:ml_foundations}

\subsubsection{Linear Algebra Basics}%
\label{ml:ssub:linear_algebra_basics}
\begin{itemize}
\item Tensor: Generalization of vectors and matrices an array of numbers with a variable number of axes
\item Linear combination: \(\sum_i c_i v^{(i)}\)
\item Span of vectors: Set of all points that can be obtained by linear combination of these vectors
\item Linear independence: No vector in the set can be obtained as a linear combination of the other vectors
\item Norms: function \(f\) that follows
  \begin{itemize}
  \item \(f(x) = 0 \Rightarrow x = 0\)
  \item \(f(x+y) \leq f(x) + f(y)\)
  \item \(\forall \alpha \in \realnumbers, f(\alpha x) = |\alpha|f(x)\)
  \end{itemize}
\item \(L^1 \) Norm: \(||x||_1 = \sum_i |x_i|\)
\item \(L^2\) Norm: \(||x||_2 = \sqrt{x_1^2 + \cdots + x_n^2}\)
\item \(L^\infty\) Norm: \(||x||_\infty = \max_i |x_i|\)
\item Diagonal Matrix: matrix where only entries on the diagonal are non-zero
\item Vectors \(x\) and \(y\) are orthogonal: \(x^T y = 0\)
\item Matrix \(A\) is orthogonal: \(A^T A = A A^T = I \Rightarrow A^{-1} = A^T\)
\item Eigenvectors and eigenvalues of a matrix \(A\): \(Av = \lambda v\) with \(\lambda \in \realnumbers\)
\item Determinant of matrix \(A\ \det(A)\): Product of all Eigenvalues of \(A\)
\item Singular Value Decomposition (SVD): factorize matrix \(A\) in matrices \(U, D\) and \(V\) such that \(A = UDV^T\)
\end{itemize}

\subsubsection{Principal Component Analysis}%
\label{ml:ssub:principal_component_analysis}
\[\frac{w^T A^T A w}{w^T w}\] is called Rayleigh quotient.
\(w\) that maximizes quotient coincides with the eigenvector of \(A^T A\)\\
Maximization:
\[\max w^T X w \text{ s.t. } w^t w = 1 \text{ where } X = A^T A\]
Lagrangian function: \(L = w^T X w - \lambda (w^T w - 1)\)\\
Derivative with respect to \(w\): \[\nabla_w L = Xw - \lambda w = 0\]
\(\rightarrow\) w is the eigenvector of \(X\)

\subsubsection{Definitions}%
\label{ml:ssub:definitions}
\begin{itemize}
\item Continuous probability distribution \(p\) for (continuous) random variable \(x\):\\
  \[\forall x \in X: p(x) \geq 0 \quad \land \quad \int p(x) dx = 1\]
\item Expected value
  \[ \mathbb{E}_{X~p}[f(x)] = \int p(x)f(x)dx\]
\item Variance
  \[\mathit{Var}(f(x)) = \mathbb{E}[{(f(x) - \mathbb{E}[f(x)])}^2]\]
\end{itemize}

\subsubsection{Probability Distributions}%
\label{ml:ssub:probability_distributions}
\begin{itemize}
\item Gaussian (normal) Distribution
  \[f(x) = \frac{1}{\sqrt{2\pi \sigma^2}} \exp (-\frac{(x - \mu)^2}{2 \sigma^2})\]
\item Exponential distribution
\item Laplace distribution
\item mixture distribution: Create more complex by combining simple stochastic distributions:
  \[P(z, x) = P(z = i)p(x|z = i)\]
  \begin{itemize}
  \item \(P(z)\) Multinomial distribution
  \item \(P(x)\) Mixture component, can take different distributions
  \end{itemize}
\end{itemize}

\subsection{Mathematical Optimization}%
\label{ml:sub:mathematical_optimization}
Algorithms to find parameters to minimize/maximize a given function \(f(x)\)

\subsubsection{Gradient Descent}%
\label{ml:ssub:gradient_descent}
\[\Delta_x f(x) = {(\frac{\delta}{\delta x_1} f(x), \ldots, \frac{\delta}{\delta x_n} f(x))}^T\]
Idea: Move in direction of gradient --- \enquote{method of steepest descent}
\[x' = x - \eta \cdot \delta_x f(x)\]
With learning rate \(0 < \eta < 1\)

\subsubsection{Newton's Method}%
\label{ml:ssub:netwons_method}
Include second-order derivate Heassian matrix:
\[H(f)(x)_{i,j} = \frac{\delta^2}{\delta x_i \delta x_j} f(x)\]
And approximate \(f(x)\) around \(x_0\) using Taylor series:
\[f(x) \approx f(x_0) + {(x - x_0)}^T \cdot \nabla_x f(x_0) + \frac{1}{2} {(x - x_0)}^T \cdot H (f)(x_0) \cdot (x - x_0)\]

\subsection{Basic Concepts}%
\label{ml:sub:basic_concepts}
\begin{itemize}
\item Supervised Learning: Dataset (training set) is labeled, i.e.\ each training sample is associated with a
  label (classification) or a target value (regression)
\item Unsupervised Learning: Dataset (training set) is not labeled, learning algorithm must establish structure
  of the dataset by itself
\item Test set: Dataset, disjunct from the training set, to evaluate model performance
\item Generalization: Ability of a learned model to perform well on previously unobserved inputs, such as the test set
\item Overfitting: Model reproduces training data well, but cannot generalize
\item Underfitting: Model cannot even reproduce training data
\item Model capacity: \enquote{Power} of the model, amount of data that model can learn
\item N-Fold Cross Validation: Splitting into training and test set. Split into n equally-sized partitions and fun folds
\end{itemize}

\subsection{Linear Regression}%
\label{ml:sub:linear_regression}
Estimate value of one variable \(y\) given an arbitrary number of input variables (in the n-dimensional vector \(X\))\\

Prediction: \(\hat{Y} = w^T X + b\)\\

How to determine w? Minimize:
\[\frac{1}{m} || (\hat{Y} - Y)||^2_2\]
for Training data \(X / Y\)

\subsection{Support Vector Machines}%
\label{ml:sub:support_vector_machines}
Popular binary classifier for supervised linear classification\\
Divide 2 data classes through hyperplane:
\[wx - b = 0\]
sign of \(wx - b\) defines side of data points.\\

Idea: Only points close are relevant \(\rightarrow\) support vectors\\

Training is optimization problem:
\[\max_w \frac{2}{|| w ||} \text{ or } \min_w ||w||^2\]

How to deal with non-linearly separable data\\
\(\rightarrow\) Soft-margin SVM\\
\(\rightarrow\) Kernel-Trick (Projection in feature space)


\subsection{k-means Clustering}%
\label{ml:sub:k_means_clustering}
Establish a structure in unlabeled data set by assigning points to \(k\) clusters

Goal: Minimize variance inside each cluster:
\[
  \sum_{j=1}^k \sum_i b_{i.j} \cdot || x_i - m_j ||^2_2
  \text{ with } b_{i,j} = \begin{cases} 1 & x_i \text{ is assigned to cluster } j\\ 0 & \text{ otherwise}\end{cases}
\]


\subsubsection{Lloyd's Algorithm}%
\label{ml:ssub:lloyds_algorithm}
Initialize clusters \(m_j\) randomly\\
Iterative approach in 2 steps:
\begin{enumerate}
\item Assignment: Associate each point \(x_i\) to closest cluster:
  \[
    b_{i, j} \leftarrow \begin{cases}1 & \text{if } || x_i - m_j ||_2 = \min_k || x_i - m_k ||_2\\
      0 & \text{otherwise}\end{cases}
  \]
\item Update step: Recompute all \(m_i\) from the assigned samples:
  \[
    m_j \leftarrow \frac{\sum_i b_{i,j} x_i}{\sum_i b_{i,j}}
  \]
\item Terminate when \(m_j\) do not change anymore
\end{enumerate}

Advantages:
\begin{itemize}
\item simple and efficient
\item guaranteed to converge
\end{itemize}
Problems:
\begin{itemize}
\item \(k\) needs to be known in advance
\item initialization of cluster means crucially affects result
\item no way to detect/handle outlier
\end{itemize}