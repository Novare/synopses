\section{Skalarprodukte}%
\label{sp:sec:skalarprodukte}

Sei K ein Körper und V ein K-Vektorraum der Dimension N, $v, v_1, v_2, w, w_1, w_2 \in V$, $\alpha \in K$ sowie
\begin{center}
	$s(\cdot,\cdot) = \langle\cdot,\cdot\rangle: V \times V \rightarrow \realnumbers^N, (v, w) \mapsto \langle v, w\rangle$
\end{center}
Sei weiter $B = \{b_1, ..., b_n\}$ eine Basis von V.

\subsection{Skalarprodukte und Fundamentalmatrizen}%
\label{sp:sub:skalarprodukte_und_fundamentalmatrizen}

Definition einer Fundamentalmatrix:
\begin{center}
	$F(s) = \begin{bmatrix}
	\langle b_1, b_1\rangle & \langle b_1, b_2\rangle & ... & \langle b_1, b_n\rangle\\
	\vdots & \vdots & \ddots & \vdots \\
	\langle b_n, b_1\rangle & \langle b_n, b_2\rangle & ... & \langle b_n, b_n\rangle\\			
	\end{bmatrix}$
\end{center}
Voraussetzungen eines Skalarprodukts (dies gelte für beliebig gewählte Vektoren):
\begin{itemize}
	\item \textbf{Bilinearform}
	\begin{itemize}
		\item $\langle \alpha v_1 + v_2, w\rangle = \alpha\langle  v_1, w\rangle + \langle v_2, w\rangle$ \textbf{und}
		\item $\langle v, \alpha w_1 + w_2\rangle = \alpha\langle  v, w_1\rangle + \langle v, w_2\rangle$
	\end{itemize}
	\item \textbf{Symmetrisch}
	\begin{itemize}
		\item Formal: $\langle v, w\rangle = \langle w, v\rangle$ \textbf{oder}
		\item Anhand der Fundamentalmatrix: Symmetrisch
	\end{itemize}
	\item \textbf{Positiv-definit}
	\begin{itemize}
		\item Formal: $\langle v, v\rangle \geq 0$ für alle $v \in V$ und $\langle v, v\rangle = 0 \Leftrightarrow v = 0$ \textbf{oder}
		\item Anhand der Fundamentalmatrix: Alle Hauptminoren sind $> 0$,\\alternativ: $v^TF(s)v \geq 0$ für alle $v \in V$
	\end{itemize}
\end{itemize}
\textbf{Standardskalarprodukt}: $\langle v, w\rangle = v^Tw$\\\\
\textbf{Cauchy-Schwarz-Ungleichung}: $\langle v, w\rangle^2 \leq \langle v, v\rangle * \langle w, w\rangle$, Gleichheit genau dann, wenn v und w linear abhängig sind

\subsection{Normen}%
\label{sp:sub:normen}

Sei K $\in \{\realnumbers,\complexnumbers\}$ ein Körper und V ein K-Vektorraum, $v, w \in V$ und $\lambda \in K$ beliebig und eine Abbildung:
\begin{center}
	$|\cdot|: V \rightarrow \realnumbers$
\end{center}
Dann heißt $|\cdot|$ \textbf{Norm}, wenn folgende Bedingungen gelten:
\begin{enumerate}
	\item \textbf{Positiv definit}: $\norm{v} \geq 0$ und $\norm{v} = 0 \Leftrightarrow v = 0$
	\item \textbf{Homogen}: $\norm{\lambda v} = \lambda \norm{v}$
	\item \textbf{Dreiecksungleichung}: $\norm{v + w} \leq \norm{v} + \norm{w}$
\end{enumerate}
\textbf{Standardnorm}: $\norm{v} := \sqrt{\langle v, v\rangle}$

\subsection{Abstandsfunktionen / Metriken}%
\label{sp:sub:abstandsfunktionen_metriken}

Sei K $\in \{\realnumbers,\complexnumbers\}$ ein Körper und V ein K-Vektorraum, $v, w, z \in V$ beliebig und eine Abbildung:
\begin{center}
	$d: V \rightarrow \realnumbers$
\end{center}
Dann heißt $d$ \textbf{Metrik}, wenn folgende Bedingungen gelten:
\begin{enumerate}
	\item \textbf{Definitheit}: $d(v, w) > 0$ falls $v \neq w$ und $d(v, w) = 0$ falls $v = w$
	\item \textbf{Symmetrie}: $d(v, w) = d(w, v)$
	\item \textbf{Dreiecksungleichung}: $d(v, z) \leq d(v, w) + d(w, z)$
\end{enumerate}
\textbf{Standardmetrik}: $d(v, w) := \norm{w - v} = \sqrt{\langle w - v, w - v\rangle}$
	
\subsection{Winkel}%
\label{sp:sub:winkel}

Sei V ein euklidischer Vektorraum (Vektorraum über $\realnumbers$ mit Skalarprodukt); das Skalarprodukt sei $\langle \cdot, \cdot\rangle$ und $v, w\in V$ beliebig. So gilt für den Winkel zwischen v und w:
\begin{center}
	$cos(\alpha) := cos(\sphericalangle(v, w)) = \frac{\langle v, w\rangle}{\norm{v}\norm{w}} = \frac{\langle v, w\rangle}{\sqrt{\langle v, v \rangle}\sqrt{\langle w, w\rangle}}$
\end{center}
Ist das \textbf{Skalarprodukt gleich 0}, so sind die Vektoren \textbf{orthogonal}.

\subsection{Änderungen bei unitären Vektorräumen}%
\label{sp:sub:aenderungen_bei_unitaeren_vektorraeumen}

Ist der zugrunde liegende Vektorraum unitär (d.h. K ist $\complexnumbers$) so muss für das Skalarprodukt gelten:
\begin{itemize}
	\item Statt einer Bilinearform ist das Skalarprodukt eine \textbf{Sesquilinearform}:
	\begin{itemize}
		\item $\langle \alpha v_1 + v_2, w\rangle = \alpha\langle  v_1, w\rangle + \langle v_2, w\rangle$ \textbf{und}
		\item $\langle v, \alpha w_1 + w_2\rangle = \overline{\alpha}\langle  v, w_1\rangle + \langle v, w_2\rangle$
	\end{itemize}
	\item Statt symmetrisch ist das Skalarprodukt \textbf{hermitesch}:
		\begin{itemize}
			\item Formal: $\langle v, w\rangle = \overline{\langle w, v\rangle}$ \textbf{oder}
			\item Anhand der Fundamentalmatrix: $F(s) = \overline{F}^T$
		\end{itemize}
	\item Das Skalarprodukt muss weiterhin \textbf{positiv definit} sein wie bisher
\end{itemize}
\textbf{Standardskalarprodukt}: $\langle v, w\rangle := v^T\overline{w} = \sum_{i = 1}^{n} v_i\overline{w}_i$\\\\
\textbf{Unitäre Cauchy-Schwarz-Ungleichung}: $|\langle v, w\rangle|^2 \leq \langle v, v\rangle * \langle w, w\rangle$, Gleichheit genau dann, wenn v und w linear abhängig sind

\subsection{Tricks \& Hinweise}%
\label{sp:sub:tricks_hinweise}

\begin{itemize}
	\item \textbf{Fourierformel}: B Orthonormalbasis, $v \in V \Rightarrow v = \sum_{b \in B} \langle v, b\rangle * b$
	\item Für eine quadratische Matrix A gilt: $|det(A)|^2 = det(A) * \overline{det(A)} = det(A) * det(\overline{A})$ sowie $det(A) = det(A^T)$
	\item Eine Bilinearform heißt nicht ausgeartet, wenn
	\begin{itemize}
		\item für alle $v \in V$ ein $w \in W$ existiert mit $\langle v, w\rangle \neq 0$
		\item für alle $w \in W$ ein $v \in V$ existiert mit $\langle v, w\rangle \neq 0$
	\end{itemize}
\end{itemize}