\section{Zerlegungen}%
\label{zerl:sec:zerlegungen}
Falls nicht anders genannt gilt für folgende Abschnitte: $A, R, L, Q \in \realnumbers^{N \times N}, b \in \realnumbers^N$.

\subsection{LR-Zerlegung}%
\label{zerl:sub:lr}
\textbf{Ziel}:\\Zerlegung von A in eine \textbf{rechte obere Dreiecksmatrix R} und eine \textbf{linke untere\\Dreiecksmatrix L}, sodass gilt: $Ax = LRx = b$
\subsubsection{Berechnung}%
\label{zerl:ssub:berechnung}
\begin{enumerate}
	\item Schreibe Matrix als Produkt $I_N * A$ mit der Einheitsmatrix $I_N$
	\item Forme schrittweise die rechte Matrix zu R um und notiere die Änderungen in der linken Matrix folgendermaßen:
	\begin{enumerate}
		\item Jede Operation wird als $Zeile\ A - Faktor * Zeile\ B$ notiert, auch\\
		Additionen (z.B als \rom{2}$ - (-2)$\rom{1}); Zeilen vertauschen ist nicht gestattet!
		\item Notiere den Vorfaktor an der Stelle, an der in R eine 0 entstanden ist
	\end{enumerate}
\end{enumerate}
\textbf{Bsp.:} $2\times2$ LR-Zerlegung
\begin{center}
	$A = 
	\begin{bmatrix}1 & 2 \\ 3 & 4\end{bmatrix} =
	\begin{bmatrix}1 & 0 \\ 0 & 1\end{bmatrix} * \begin{bmatrix}1 & 2 \\ 3 & 4\end{bmatrix} \stackrel{II - 3I}{=}
	\begin{bmatrix}1 & 0 \\ 3 & 1\end{bmatrix} * \begin{bmatrix}1 & 2 \\ 0 & -2\end{bmatrix} = LR
	$	
\end{center}
\subsubsection{Berechnung mit Pivot-Suche}%
\label{zerl:ssub:berechnung-pivot}
\textbf{Idee:}\\A lässt sich durch Zeilenvertauschungen immer in eine Form bringen, sodass eine LR-Zerlegung sicher existiert (falls A regulär). Hiermit lässt sich auch die Konditionierung der Zerlegung verbessern.\\\\
\textbf{Vorgehen}:
\begin{enumerate}
	\item Wähle eine Permutationsmatrix P und zerlege PA, sodass gilt: $PA = LR$
	\item Löse $LRx = Pbx$
\end{enumerate}
\textbf{Permutationsmatrix für bessere Konditionierung durch Spaltenpivotsuche}:
\begin{enumerate}
	\item Iteriere über die n Zeilen der Matrix A
	\item Bestimmte k, sodass $|A[k, n]| \geq |A[j, n]|$ für $j = n, ..., N$, d.h. suche in der aktuellen Spalte nach einem Eintrag unterhalb des aktuellen Eintrags, der betragsmäßig größer oder gleich ist
	\item Vertausche die Zeilen n und k, falls $k \neq n$
\end{enumerate}
\textbf{Performantes Lösen einer fertigen LR-Zerlegung:}
\begin{enumerate}
	\item Substituiere $Rx$ mit $y\in\realnumbers^N$ und löse $Ly = b$ per Vorwärtssubstitution
	\item Löse $Rx = y$ per Rückwärtssubstitution
\end{enumerate}
\subsection{Cholesky-Zerlegung}%
\label{zerl:sub:cholesky}
\textbf{Ziel}:\\Zerlegung von A in eine \textbf{reguläre linke untere Dreiecksmatrix L}, wobei \textbf{A symmetrisch und positiv definit sein muss}, sodass gilt: $A = LL^T$\\\\
\textbf{Berechnen der Cholesky-Zerlegung:}
\begin{itemize}
	\item Alle Einträge über der Diagonalen sind 0, für alle anderen Einträge befolge dies zeilenweise:
	\begin{enumerate}
		\item Berechne die Summe $s_{ij} = a_{ij} - \sum_{k = 1}^{j - 1} l_{ik}l_{jk}$
		\item Für einen Diagonaleintrag gilt: $l_{ii} = \sqrt{s_{ii}}$ 
		\item Für einen Eintrag unter der Diagonalen gilt: $l_{ij} = \frac{s_{ij}}{l_{jj}}$
	\end{enumerate}
\end{itemize}
\textbf{Bsp.:} 3x3 Cholesky-Zerlegung
\begin{center}
	$A 	= \begin{bmatrix}
			4 & 2 & 2\\
			2 & 17 & 5\\
			2 & 5 & 11
		\end{bmatrix}$\\
		\vspace*{0.5cm}
	$L 	= \begin{bmatrix}
			l_{11} & 0 & 0\\
			l_{21} & l_{22} & 0\\
			l_{31} & l_{32} & l_{33}
		\end{bmatrix}
		= \begin{bmatrix}
	  		\sqrt{4 - 0} & 0 & 0\\
	  		\frac{2 - 0}{2} & \sqrt{17 - 1 * 1} & 0\\
	  		\frac{2 - 0}{2} & \frac{5 - 1 * 1}{4} & \sqrt{11 - (1 * 1 + 1 * 1)}
			\end{bmatrix}
			= \begin{bmatrix}
 			2 & 0 & 0\\
 			1 & 4 & 0\\
 			1 & 1 & 3
			\end{bmatrix}$
\end{center}
\subsection{QR-Zerlegung}%
\label{zerl:sub:qr}
\textbf{Ziel}:\\Zerlegung von A in eine \textbf{rechte obere Dreiecksmatrix R} und eine \textbf{orthogonale Matrix Q}, sodass gilt: $A = QR$. Dies ist nützlich, wenn A nicht quadratisch ist, da die LR-Zerlegung dann nicht funktioniert.\\\\
\textbf{Berechnen der QR-Zerlegung}:
\begin{enumerate}
	\item Berechne $\alpha = sgn(a_{11})\norm{a_1}$ und damit $v_1 = a_1 + \alpha e_1$ und damit $Q_1 = I - \frac{2vv^T}{v^Tv}$ (Householder-Spiegelung)
	\item Berechne $Q_1A$. Wähle von diesem Ergebnis eine Untermatrix (ohne erste Zeile und erste Spalte) und wiederhole das Verfahren mit dieser und erhalte $Q_2^*$.
	\item Bette $Q_2^*$ in eine Einheitsmatrix mit denselben Dimensionen wie A ein und erhalte $Q_2$.
	\item Wiederhole 2. und 3. bis $Q_n * Q_{n - 1} * ... Q_1A = R$ Dreiecksgestalt hat.
	\item Multipliziere von links $Q_1^T * Q_2^T * ... Q_n^T = Q$ an beide Seiten.
\end{enumerate}
\newpage
\noindent\textbf{Performantes Lösen einer fertigen QR-Zerlegung}:
\begin{enumerate}
	\item A sei überbestimmt (keine LR-Zerlegung möglich) und $Ax = QRx = b$
	\item Multipliziere $A^T$ auf beiden Seiten: $A^TAx = (QR)^TQRx = R^TQ^TQRx = R^TRx = b$
	\item Löse $R^TRx = b$ analog zum LR-Verfahren per Vorwärts- und Rückwärtssubstitution
\end{enumerate}
\subsection{Aufwand}
\begin{itemize}
	\item QR-Zerlegung mit N = M:\hfill$\frac{4}{3}N^3$ Operationen
	\item LR-Zerlegung:\hfill$\frac{2}{3}N^3$ Operationen
	\item Cholesky-Zerlegung:\hfill$\frac{1}{3}N^3$ Operationen
\end{itemize}