\section{Integrationsverfahren für gewöhnliche Differentialgleichungen}%
\label{intd:sec:integrationsverfahren}
\subsection{Runge-Kutta-Verfahren}%
\label{intd:sub:runge-kutta}
\textbf{Ziel}: Numerisches Lösen von Anfangswertproblemen. Definition eines Anfangswertproblems:
\begin{center} 
	$y'(t) = f(t, y(t)), y(t_0) = y_0, y: \realnumbers \rightarrow \realnumbers^d$
\end{center}
\textbf{Beispiele}:\\
Die verschiedenen Runge-Kutta-Verfahren unterscheiden sich in erster Linie in der Konvergenzordnung R. Mit höherer Konvergenzordnung wächst die Zahl zu lösender Gleichungen schnell an, die Ergebnisse werden dafür jedoch genauer.
\begin{itemize}
	\item Explizites Euler-Verfahren\hfill R = 1
	\item Verfahren von Heun\hfill R = 2
	\item Klassisches Runge-Kutta-Verfahren\hfill R = 4
\end{itemize}