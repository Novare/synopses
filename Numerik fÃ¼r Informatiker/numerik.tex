\documentclass[10pt,a4paper]{article}
\author{Jannik Koch}
\title{Numerik für Informatiker}

\usepackage[utf8]{inputenc}
\usepackage{amsmath}
\usepackage[a4paper, total={6in, 8in}]{geometry}

\def\realnumbers{{\rm I\!R}}
\newcommand{\rom}[1]{\uppercase\expandafter{\romannumeral #1\relax}}
\newcommand{\norm}[1]{\lVert#1\rVert}

\begin{document}
	\pagenumbering{Roman}
	{\let\newpage\relax\maketitle}
	\tableofcontents
	\newpage
	\pagenumbering{arabic}
	\setcounter{page}{1}

	\section{Grundlagen}
	Falls nicht anders genannt gilt für folgende Abschnitte: $A \in \realnumbers^{N \times N}, b \in \realnumbers^N$.
	
	\subsection{Vorwärts-/Rückwärtssubstitution}
	Zugrunde liegendes Problem ist die Lösung des linearen Gleichungssystems: $Ax = b$.\\
	\textbf{Voraussetzung:} A sei eine rechte obere oder linke untere Dreiecksmatrix.\\\\
	\textbf{Vorgehen:}
	\begin{enumerate}
		\item Lese die Lösung an der Stelle der Dreiecksmatrix ab, wo die Zeile nur einen Wert ungleich Null enthält
		\item Nutze diese Lösung in der nächsten Zeile um die nächste eindeutige Lösung zu ermitteln
		\item Wiederhole dies, bis alle Lösungen gefunden sind
		\item \textbf{Vorwärtssubstitution}:\\Matrix ist eine untere linke Dreiecksmatrix (Lösen von oben nach unten)
		\item \textbf{Rückwärtssubstitution}:\\Matrix ist eine obere rechte Dreiecksmatrix (Lösen von unten nach oben)
		\item Aufwand für Vor- bzw. Rückwärtssubstitution: ca. $\frac{N^2}{2}$
	\end{enumerate}

	\textbf{Bsp.:} Rückwärtssubstitution
	\begin{center}
		$\begin{bmatrix} 1 & 1\\0 & 3\\\end{bmatrix}x = \begin{bmatrix}2 \\ 3\end{bmatrix} \Rightarrow
		x_2 = 1 \Rightarrow x_1 + 1 = 2 \Rightarrow x_1 = 1 \Rightarrow x = \begin{bmatrix}1 \\ 1\end{bmatrix}$
	\end{center}
	
	\subsection{Gleitkommazahlen}
	\textbf{Darstellung}:
	\begin{itemize}
		\item Darstellung einer Gleitkommazahl z: $z = a * d^e$
		\item $d$: Basis, im Zweiersystem eine Zweierpotenz (2, 4, 8 etc.)
		\item $e$: Exponent, eine ganze Zahl zwischen $e_{min}$ und $e_{max}$
		\item $a$: Die Mantisse, entweder 0 oder eine Zahl mit $d^{-1} \leq |a| < 1$ der Form $a = v \sum_{i = 1}^{l} a_id^{-i}$ mit dem Vorzeichenbit v und der Mantissenlänge l
		\item Relative Maschinengenauigkeit: $eps = \frac{d^{(1-l)}}{2}$
		\item Rundungsfunktion $rd(x)$ rundet die Nachkommastellen auf ein maschinell darstellbares Format
	\end{itemize}
	\textbf{Operationen}:
	\begin{itemize}
		\item Standard-Operationen verfügbar, Rundung nach Ausführung $\Rightarrow$ Fehlerquelle!
		\item Aufgrund Rundung sind Operationen nicht assoziativ
		\item \textbf{Auslöschung}: Verlust der Genauigkeit bei der Subtraktion fast gleich großer Gleitkommazahlen
		\begin{itemize}
			\item Bsp.: Ergebnis zweier Operationen ist bis auf Rundungsfehler gleich, Zahlen werden subtrahiert, höherwertige Stellen werden 0 und die übrige Zahl der verfälschten Stellen steigt unverhältnismäßig (enormer relativer Fehler)
		\end{itemize}
		\item Es gilt: $1 + |y| = 1,\ falls\ |y| < eps$
	\end{itemize}
	
	
	\subsection{Matrixnormen}
	\begin{enumerate}
		\item{\makebox[13cm]{\textbf{Spaltensummennorm}: Summiere Spalten, wähle Maximalwert\hfill} $\norm{A}_1$}
		\item{\makebox[13cm]{\textbf{Spektralnorm}: Wurzel des größten Eigenwerts von $A^TA$\hfill} $\norm{A}_2$}
		\item{\makebox[13cm]{\textbf{Zeilensummennorm}: Summiere Zeilen, wähle Maximalwert\hfill} $\norm{A}_\infty$}
	\end{enumerate}
	
	\subsection{Konditionen}
	\textbf{Konditionszahl}:
	\begin{itemize}
		\item Maß für den Einfluss der Störungen von A und b auf x (wie sensibel ist das LGS?)
		\item $1 \leq cond(A) := \norm{A}\norm{A^{-1}}$
		\item $cond(A) = cond(\alpha A), \alpha\in\realnumbers\setminus\{O\}$
		\item $cond(A) = \frac{max_{\norm{y}=1}\norm{Ay}}{max_{\norm{z}=1}\norm{Az}}$
	\end{itemize}
	%TODO
	
	\subsection{Householder-Transformationen}
	
	\section{Zerlegungen}
	Falls nicht anders genannt gilt für folgende Abschnitte: $A, R, L, Q \in \realnumbers^{N \times N}, b \in \realnumbers^N$.

	\subsection{LR-Zerlegung}
	Ziel: Zerlegung von A in eine \textbf{rechte obere Dreiecksmatrix R} und eine \textbf{linke untere Dreiecksmatrix L}, sodass gilt: $Ax = LRx = b$\\\\
	\textbf{Berechnen der LR-Zerlegung:}
	\begin{enumerate}
		\item Schreibe Matrix als Produkt $I_N * A$ mit der Einheitsmatrix $I_N$
		\item Forme schrittweise die rechte Matrix zu R um und notiere die Änderungen in der linken Matrix folgendermaßen:
		\begin{enumerate}
			\item Jede Operation wird als $Zeile\ A - Faktor * Zeile\ B$ notiert, auch\\
			Additionen (z.B als \rom{2}$ - (-2)$\rom{1}); Zeilen vertauschen ist nicht gestattet!
			\item Notiere den Vorfaktor an der Stelle, an der in R eine 0 entstanden ist
		\end{enumerate}
	\end{enumerate}
	\textbf{Bsp.:} $2\times2$ LR-Zerlegung
	\begin{center}
		$A = 
		\begin{bmatrix}1 & 2 \\ 3 & 4\end{bmatrix} =
		\begin{bmatrix}1 & 0 \\ 0 & 1\end{bmatrix} * \begin{bmatrix}1 & 2 \\ 3 & 4\end{bmatrix} \stackrel{II - 3I}{=}
		\begin{bmatrix}1 & 0 \\ 3 & 1\end{bmatrix} * \begin{bmatrix}1 & 2 \\ 0 & -2\end{bmatrix} = LR
		$	
	\end{center}
	\textbf{Performantes Lösen einer fertigen LR-Zerlegung:}
	\begin{enumerate}
		\item Substituiere $Rx$ mit $y\in\realnumbers^N$ und löse $Ly = b$ per Vorwärtssubstitution
		\item Löse $Rx = y$ per Rückwärtssubstitution
	\end{enumerate}
	\subsection{Cholesky-Zerlegung}
	%TODO

	\subsection{QR-Zerlegung}
	%TODO
\end{document}