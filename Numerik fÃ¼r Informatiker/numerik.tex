\documentclass[10pt,a4paper]{article}
\author{Jannik Koch}
\title{Numerik für Informatiker}

\usepackage[utf8]{inputenc}
\usepackage{multicol}
\usepackage{amsmath}
\usepackage[a4paper, total={6in, 8in}]{geometry}

\def\realnumbers{{\rm I\!R}}
\newcommand{\rom}[1]{\uppercase\expandafter{\romannumeral #1\relax}}
\newcommand{\norm}[1]{\lVert#1\rVert}

\begin{document}
	\pagenumbering{Roman}
	{\let\newpage\relax\maketitle}
	\tableofcontents
	\newpage
	\pagenumbering{arabic}
	\setcounter{page}{1}

	\section{Grundlagen}
	Falls nicht anders genannt gilt für folgende Abschnitte: $A \in \realnumbers^{N \times N}, b \in \realnumbers^N$.
	
	\subsection{Vorwärts-/Rückwärtssubstitution}
	Zugrunde liegendes Problem ist die Lösung des linearen Gleichungssystems: $Ax = b$.\\
	\textbf{Voraussetzung:} A sei eine rechte obere oder linke untere Dreiecksmatrix.\\\\
	\textbf{Vorgehen:}
	\begin{enumerate}
		\item Lese die Lösung an der Stelle der Dreiecksmatrix ab, wo die Zeile nur einen Wert ungleich Null enthält
		\item Nutze diese Lösung in der nächsten Zeile um die nächste eindeutige Lösung zu ermitteln
		\item Wiederhole dies, bis alle Lösungen gefunden sind
		\item \textbf{Vorwärtssubstitution}:\\Matrix ist eine untere linke Dreiecksmatrix (Lösen von oben nach unten)
		\item \textbf{Rückwärtssubstitution}:\\Matrix ist eine obere rechte Dreiecksmatrix (Lösen von unten nach oben)
		\item Aufwand für Vor- bzw. Rückwärtssubstitution: ca. $\frac{N^2}{2}$
	\end{enumerate}

	\textbf{Bsp.:} Rückwärtssubstitution
	\begin{center}
		$\begin{bmatrix} 1 & 1\\0 & 3\\\end{bmatrix}x = \begin{bmatrix}2 \\ 3\end{bmatrix} \Rightarrow
		x_2 = 1 \Rightarrow x_1 + 1 = 2 \Rightarrow x_1 = 1 \Rightarrow x = \begin{bmatrix}1 \\ 1\end{bmatrix}$
	\end{center}
	
	\subsection{Gleitkommazahlen}
	\textbf{Darstellung}:
	\begin{itemize}
		\item Darstellung einer Gleitkommazahl z: $z = a * d^e$
		\item $d$: Basis, im Zweiersystem eine Zweierpotenz (2, 4, 8 etc.)
		\item $e$: Exponent, eine ganze Zahl zwischen $e_{min}$ und $e_{max}$
		\item $a$: Die Mantisse, entweder 0 oder eine Zahl mit $d^{-1} \leq |a| < 1$ der Form $a = v \sum_{i = 1}^{l} a_id^{-i}$ mit dem Vorzeichenbit v und der Mantissenlänge l
		\item Relative Maschinengenauigkeit: $eps = \frac{d^{(1-l)}}{2}$
		\item Rundungsfunktion $rd(x)$ rundet die Nachkommastellen auf ein maschinell darstellbares Format
	\end{itemize}
	\textbf{Operationen}:
	\begin{itemize}
		\item Standard-Operationen verfügbar, Rundung nach Ausführung $\Rightarrow$ Fehlerquelle!
		\item Aufgrund Rundung sind Operationen nicht assoziativ
		\item \textbf{Auslöschung}: Verlust der Genauigkeit bei der Subtraktion fast gleich großer Gleitkommazahlen
		\begin{itemize}
			\item Bsp.: Ergebnis zweier Operationen ist bis auf Rundungsfehler gleich, Zahlen werden subtrahiert, höherwertige Stellen werden 0 und die übrige Zahl der verfälschten Stellen steigt unverhältnismäßig (enormer relativer Fehler)
		\end{itemize}
		\item Es gilt: $1 + |y| = 1,\ falls\ |y| < eps$
	\end{itemize}
	
	
	\subsection{Matrixnormen}
	\begin{enumerate}
		\item{\makebox[13cm]{\textbf{Spaltensummennorm}: Summiere Spalten, wähle Maximalwert\hfill} $\norm{A}_1$}
		\item{\makebox[13cm]{\textbf{Spektralnorm}: Wurzel des größten Eigenwerts von $A^TA$\hfill} $\norm{A}_2$}
		\item{\makebox[13cm]{\textbf{Zeilensummennorm}: Summiere Zeilen, wähle Maximalwert\hfill} $\norm{A}_\infty$}
	\end{enumerate}
	
	\subsection{Konditionen}
	\textbf{Konditionszahl}:
	Sei $f: \realnumbers^N \rightarrow \realnumbers^K$ eine differenzierbare Funktion und $x \in \realnumbers^N$
	\begin{itemize}
		\item Maß für den Einfluss der Störungen von A und b auf x (wie sensibel ist das LGS?)
		\item Kondition einer Matrix: $1 \leq cond(A) := \norm{A}\norm{A^{-1}}$, wobei $cond(A) = cond(\alpha A), \alpha\in\realnumbers\setminus\{O\}$
		\item Absolute Konditionszahl: $\kappa^{kn}_{abs}(x) = |\frac{\delta}{\delta x_n}f_k(x)|$
		\item Relative Konditionszahl: $\kappa^{kn}_{rel}(x) = |\frac{\delta}{\delta x_n}f_k(x)| \frac{|x_n|}{|f_k(x)|}$ falls $f_k(x) \neq 0$
	\end{itemize}
	\newpage
	\section{Zerlegungen}
	Falls nicht anders genannt gilt für folgende Abschnitte: $A, R, L, Q \in \realnumbers^{N \times N}, b \in \realnumbers^N$.

	\subsection{LR-Zerlegung}
	Ziel: Zerlegung von A in eine \textbf{rechte obere Dreiecksmatrix R} und eine \textbf{linke untere Dreiecksmatrix L}, sodass gilt: $Ax = LRx = b$\\\\
	\textbf{Berechnen der LR-Zerlegung:}
	\begin{enumerate}
		\item Schreibe Matrix als Produkt $I_N * A$ mit der Einheitsmatrix $I_N$
		\item Forme schrittweise die rechte Matrix zu R um und notiere die Änderungen in der linken Matrix folgendermaßen:
		\begin{enumerate}
			\item Jede Operation wird als $Zeile\ A - Faktor * Zeile\ B$ notiert, auch\\
			Additionen (z.B als \rom{2}$ - (-2)$\rom{1}); Zeilen vertauschen ist nicht gestattet!
			\item Notiere den Vorfaktor an der Stelle, an der in R eine 0 entstanden ist
		\end{enumerate}
	\end{enumerate}
	\textbf{Bsp.:} $2\times2$ LR-Zerlegung
	\begin{center}
		$A = 
		\begin{bmatrix}1 & 2 \\ 3 & 4\end{bmatrix} =
		\begin{bmatrix}1 & 0 \\ 0 & 1\end{bmatrix} * \begin{bmatrix}1 & 2 \\ 3 & 4\end{bmatrix} \stackrel{II - 3I}{=}
		\begin{bmatrix}1 & 0 \\ 3 & 1\end{bmatrix} * \begin{bmatrix}1 & 2 \\ 0 & -2\end{bmatrix} = LR
		$	
	\end{center}
	\textbf{Performantes Lösen einer fertigen LR-Zerlegung:}
	\begin{enumerate}
		\item Substituiere $Rx$ mit $y\in\realnumbers^N$ und löse $Ly = b$ per Vorwärtssubstitution
		\item Löse $Rx = y$ per Rückwärtssubstitution
	\end{enumerate}
	\subsection{Cholesky-Zerlegung}
	Ziel: Zerlegung von A in eine \textbf{reguläre linke untere Dreiecksmatrix L}, wobei \textbf{A symmetrisch und positiv definit sein muss}, sodass gilt: $A = LL^T$\\\\
	\textbf{Berechnen der Cholesky-Zerlegung:}
	\begin{itemize}
		\item Alle Einträge über der Diagonalen sind 0, für alle anderen Einträge befolge dies zeilenweise:
		\begin{enumerate}
			\item Berechne die Summe $s_{ij} = a_{ij} - \sum_{k = 1}^{j - 1} l_{ik}l_{jk}$
			\item Für einen Diagonaleintrag gilt: $l_{ii} = \sqrt{s_{ii}}$ 
			\item Für einen Eintrag unter der Diagonalen gilt: $l_{ij} = \frac{s_{ij}}{l_{jj}}$
		\end{enumerate}
	\end{itemize}
	\textbf{Bsp.:} 3x3 Cholesky-Zerlegung
	\begin{center}
		$A = \begin{bmatrix}4 & 2 & 2\\2 & 17 & 5\\2 & 5 & 11\end{bmatrix}$\\\vspace*{0.5cm}
		$L = \begin{bmatrix}l_{11} & 0 & 0\\l_{21} & l_{22} & 0\\l_{31} & l_{32} & l_{33}\end{bmatrix}
		= \begin{bmatrix}
		  \sqrt{4 - 0} & 0 & 0\\
		  \frac{2 - 0}{2} & \sqrt{17 - 1 * 1} & 0\\
		  \frac{2 - 0}{2} & \frac{5 - 1 * 1}{4} & \sqrt{11 - (1 * 1 + 1 * 1)}
 		\end{bmatrix}
 		= \begin{bmatrix}
 			2 & 0 & 0\\
 			1 & 4 & 0\\
 			1 & 1 & 3
 		\end{bmatrix}$
	\end{center}
		
	\subsection{QR-Zerlegung}
	Ziel: Zerlegung von A in eine \textbf{rechte obere Dreiecksmatrix R} und eine \textbf{orthogonale Matrix Q}, sodass gilt: $A = QR$. Dies ist nützlich, wenn A nicht quadratisch ist, da die LR-Zerlegung dann nicht funktioniert.\\\\
	\textbf{Berechnen der QR-Zerlegung}:
	\begin{enumerate}
		\item Berechne $\alpha = sgn(a_{11})\norm{a_1}$ und damit $v_1 = a_1 + \alpha e_1$ und damit $Q_1 = I - \frac{2vv^T}{v^Tv}$ (Householder-Transformation)
		\item Berechne $Q_1A$. Das Ergebnis sollte die Elemente unter der ersten Spalte zu 0 reduziert haben. Wähle von diesem Ergebnis eine Untermatrix (ohne erste Zeile und erste Spalte) und wiederhole das Verfahren mit dieser und erhalte $Q_2^*$.
		\item Bette $Q_2^*$ in eine Einheitsmatrix mit denselben Dimensionen wie A ein und erhalte $Q_2$.
		\item Wiederhole 2. und 3. bis $Q_n * Q_{n - 1} * ... Q_1A = R$ Dreiecksgestalt hat.
		\item Multipliziere von links $Q_1^T * Q_2^T * ... Q_n^T = Q$ an beide Seiten.
	\end{enumerate}
	\textbf{Bsp.:}
	\begin{center}
		$A = \begin{bmatrix}1 & 1 & 2\\2 & -3 & 0 \\2 & 4 & -4\end{bmatrix}$\\\vspace*{0.5cm}
		$\Rightarrow \alpha_1 = 1 * \sqrt{1^2 + 2^2 + 2^2} = 3 \Rightarrow v_1 = \begin{bmatrix} 1 \\ 2 \\ 2\end{bmatrix} + 3 * \begin{bmatrix}1 \\ 0 \\ 0\end{bmatrix} = \begin{bmatrix} 4 \\ 2 \\ 2\end{bmatrix}$\\
		$\Rightarrow Q_1 = \begin{bmatrix}1 & 0 & 0 \\ 0 & 1 & 0 \\ 0 & 0 & 1\end{bmatrix} - \frac{2}{24} \begin{bmatrix}16 & 8 & 8 \\ 8 & 4 & 4\\ 8 & 4 & 4\end{bmatrix} = \frac{1}{3} \begin{bmatrix}-1 & -2 & -2 \\ -2 & 2 & -1 \\ -2 & -1 & 2\end{bmatrix}
		\Rightarrow Q_1A = \begin{bmatrix}-3 & -1 & 2 \\ 0 & -4 & 0 \\ 0 & 3 & -4\end{bmatrix}$\\\vspace*{0.5cm}
		Wdh. mit $\begin{bmatrix}-4 & 0 \\ 3 & -4\end{bmatrix}$\\
		$\Rightarrow Q_2^* = \frac{1}{5}\begin{bmatrix}-4 & 3 \\ 3 & 4\end{bmatrix} \Rightarrow Q_2 = \frac{1}{5}\begin{bmatrix}5 & 0 & 0\\0 & -4 & 3\\0 & 3 & 4\end{bmatrix} \Rightarrow Q_2Q_1A = \begin{bmatrix}-3 & -1 & 2 \\ 0 & 5 & -\frac{12}{5} \\ 0 & 0 & -\frac{16}{5}\end{bmatrix} = R$\\\vspace*{0.5cm}
		$Q_1^TQ_2^TQ_2Q_1A = Q_1^TQ_2^TR \Leftrightarrow A = Q_1^TQ_2^TR$\\
		$Q := Q_1^TQ_2^T \Rightarrow A = QR$
	\end{center}
	\textbf{Performantes Lösen einer fertigen QR-Zerlegung}:
	\begin{enumerate}
		\item A sei überbestimmt (keine LR-Zerlegung möglich) und $Ax = QRx = b$
		\item Multipliziere $A^T$ auf beiden Seiten: $A^TAx = (QR)^TQRx = R^TQ^TQRx = R^TRx = b$
		\item Löse $R^TRx = b$ analog zum LR-Verfahren per Vorwärts- und Rückwärtssubstitution
	\end{enumerate}
	\subsection{Aufwand}
	\begin{itemize}
		\item QR-Zerlegung mit N = M:\hfill$\frac{4}{3}N^3$ Operationen
		\item LR-Zerlegung:\hfill$\frac{2}{3}N^3$ Operationen
		\item Cholesky-Zerlegung:\hfill$\frac{1}{3}N^3$ Operationen
	\end{itemize}
	\section{Lineare Ausgleichsrechnung}
	\textbf{Problem}: Lineare Gleichungssysteme oft nicht oder nicht eindeutig lösbar.\\
	\textbf{Lösung}: Versuche die bestmögliche Lösung zu finden, sodass gilt: für $x \in \realnumbers^N$ ist $|Ax - b|_2$ minimal
	\begin{center}
		Hierbei gilt: $x \in \realnumbers^N$ minimiert $|Ax - b|_2 \Leftrightarrow$ $x \in \realnumbers^N$ löst die Normalengleichung $A^TAx = A^Tb$
	\end{center}
	\subsection{Per QR-Zerlegung}
	Voraussetzung: $A^TA$ ist invertierbar und gut konditioniert\\
	\textbf{Vorgehen}:
	\begin{enumerate}
		\item Berechne QR-Zerlegung: $A = QR$, dann ist $A^TA = R^TR$ (s. Kapitel zu QR-Zerlegung)
		\item Löse $Rx = Q^Tb$
	\end{enumerate}
	
	\subsection{Per Singulärwertzerlegung}
	Alternative zum QR-Ansatz falls gilt: $A^TA$ ist singulär (besitzt keine Inverse) oder schlecht konditioniert. Sei für diesen Fall $A \in \realnumbers^{K\times N}$. Zerlege A folgendermaßen $A = V\Sigma U^T$. Sei $R = rang(A)$. Hierbei gilt:
	\begin{itemize}
		\item $V \in \realnumbers^{K\times R}$, wobei $V^TV = I_R$
		\item $U \in \realnumbers^{N\times R}$
		\item $\Sigma = diag(\sigma_1, ..., \sigma_R) \in \realnumbers^{R\times R}$ mit den Singulärwerten $\sigma_1, ..., \sigma_R > 0$
	\end{itemize}
	Die Bestimmung der einzelnen Matrizen ist nicht klausurrelevant.
	
	\section{Eigenwertberechnung}
	\section{Iterationsverfahren für lineare Gleichungssysteme}
	\section{Iterationsverfahren für nichtlineare Gleichungssysteme}
	\section{Polynom-Interpolation}
	\section{Splines}
	\section{Trigonometrische Interpolation und FFT}
	\section{Numerische Integration}
	\section{Integrationsverfahren für gewöhnliche Differentialgleichungen}
\end{document}