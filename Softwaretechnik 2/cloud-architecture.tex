\section{Cloud Architecture}
\label{cla:sec:cloud_architecture}

\begin{itemize}
	\item \textbf{5 essenzielle Charakteristika von Cloud Services}:
	\begin{enumerate}
		\item \textbf{Elastic Scalability}
		\item \textbf{On-demand Self-Service}
		\item \textbf{Ubiquitous Network Access}
		\item \textbf{Resource Pooling}
		\item \textbf{Measured Service}
	\end{enumerate}
	\item \textbf{Service Delivery Models}:
	\begin{itemize}
		\item \textbf{Software as a Service (SaaS)}: Web-Applikationen wie z.B Google Apps
		\item \textbf{Platform as a Service (Paas)}: Development-/Execution-Environments wie Windows Azure
		\item \textbf{Infrastructure as a Service (IaaS)}: Infrastruktur, z.B Storage-Solutions wie Google Drive
	\end{itemize}
	\item \textbf{Service Deployment Models}:
	\begin{itemize}
		\item \textbf{Private Cloud}: Organisationsinterne Cloud (z.B File-Server eines Unternehmens)
		\item \textbf{Public Cloud}: Organisationsübereifende Cloud (z.B Google Drive für Privatnutzer)
		\item \textbf{Hybrid Cloud}: Mischung aus private und public Cloud
	\end{itemize}
	\item \textbf{Single Tenant Architecture}: Jeder Nutzer erhält einen gleich großen Anteil an Ressourcen (Hardware, OS, Applikationen)
	\item \textbf{Multi Tenant Architecture}: Unterschiedliche Aufteilung der Gesamtressourcen für jeden Nutzer
\end{itemize}

\subsection{Virtualisierung}
\label{cla:sub:virtualisierung}

\begin{itemize}
	\item \textbf{Allgemein}: Bereitstellen einer logischen Abstraktion von physischen Ressourcen
	\item Beispiel: \textbf{Virtuelle Maschinen}
	\begin{itemize}
		\item Emuliere ein Computersystem innerhalb eines anderen Computersystems
		\item \textbf{Hypervisor}: Kontrolliert installierte virtuelle Maschinen und ihre Nutzung von echter Hardware
		\item \textbf{Partitionierung}: Mehrere Maschinen mit eigenen Betriebssystemen auf demselben Server
		\item \textbf{Isolation}: Virtuelle Maschinen beeinflussen sich nicht gegenseitig
	\end{itemize}
	\item \textbf{Vorteile}: Energie-Effizienz, Verfügbarkeit (leichte Migration), einfacheres Management, kaum Performance-Verlust
\end{itemize}

\subsection{Architecture Principles}
\label{cla:sub:architecture_principles}

\begin{itemize}
	\item \textbf{Decentralization}: Kein \textbf{Single Point of Failure}, vermindere Bottlenecks bei der Skalierung
	\item \textbf{Asynchrony}: Das System macht in jedem Fall Fortschritt
	\item \textbf{Autonomy}: Individuelle Komponenten können Entscheidungen basierend auf lokalen Informationen treffen
	\item \textbf{Local Responsibility}: Jede Komponente ist verantwortlich für ihre eigene Konsistenz
	\item \textbf{Controlled Concurrency}: Keine oder limitierte Concurrency Control ist benötigt
	\item \textbf{Failure Tolerant}: Fehlschlagende Komponenten sorgen nicht für Unterbrechungen
	\item \textbf{Controlled Parallelism}: Granulare Abstraktionen ermöglichen Parallelität für Robustheit und Performance
	\item \textbf{Building Blocks}: Kleine Komponenten, die den Service zusammensetzen
	\item \textbf{Symmetry}: Keine spezifische Konfiguration für jeden Knoten, symmetrischer Umfang an Funktionalität
	\item \textbf{Simplicity}: Das System sollte so simpel sein wie möglich, aber nicht simpler
\end{itemize}