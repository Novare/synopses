\section{Agile}
\label{ag:sec:agile}

\textbf{Agile Methoden stellen}
\begin{itemize}
	\item \dots \textbf{Individuen und Interaktion} über Prozesse und Tools
	\item \dots \textbf{funktionierende Software} über umfassende Dokumentation
	\item \dots \textbf{Kundenkollaboration} über Vertragsverhandlungen
	\item \dots \textbf{Flexibilität} über starre Pläne
\end{itemize}

\subsection{Extreme Programming (XP)}
\label{ag:sub:extreme_programming}

\begin{itemize}
	\item Iteratives Vorgehen mit inkrementellem Design
	\item \textbf{Werte}: Communication, Simplicity, Feedback, Courage
	\item \textbf{Prinzipien}: Quick Delivery, Rapid Feedback, Keep It Simple, Incremental Change, Embrace Change
	\item \textbf{Praxis}: u.a. Test-Driven-Development, Pair Programming, Continuous Integration, Collective Ownership
	\item \textbf{Kritik}:
	\begin{itemize}
		\item Unpassend für große Projekte
		\item Fehlende Dokumentation, kein \quotestyle{cross-project reuse}
		\item Kundenkooperation ist notwendig
		\item Pair-Programming und Test-Driven-Development nicht als sinnvoll bestätigt
	\end{itemize}
\end{itemize}

\subsection{Scrum}
\label{ag:sub:scrum}

\begin{itemize}
	\item \textbf{Management Framework} für agile Methoden
	\item Der Prozess ist eine Kette von \textbf{Sprints} (ca. zweiwöchige Iterationen)
	\item Sprints sollten \textbf{konsistente Zeitrahmen} haben
	\item Regelmäßiges Ausliefern der Software nach jedem \textbf{Sprint}
	\item \textbf{3 Rollen}:
	\begin{itemize}
		\item \textbf{Product Owner}: Verantwortlich für das System, entscheidet über Releases
		\item \textbf{Scrum Master}: Überwacht und verbessert Anwendung von Scrum
		\item \textbf{Team}: Entwickler, entscheiden über die Menge umsetzbarer Ideen eines Sprints
	\end{itemize}
	\item \textbf{4 Meetings}:
	\begin{itemize}
		\item \textbf{Daily Scrum}: Austausch zur aktuellen Lage, Anpassen des Plans, individuelle Fragen, Aktualisieren der \textbf{Burn Down Chart}, welche das Verhältnis von \textbf{Aufgaben des Sprints} zur \textbf{restlichen Zeit} visualisiert
		\item \textbf{Sprint Planning Meeting}: Ausarbeitung eines detaillierten Plans für den Sprint
		\item \textbf{Sprint Review Meeting}: Demonstration von potenziell auslieferbarem Code während dem Sprint
		\item \textbf{Sprint Retrospective}: Bewertung des beendeten Sprints
	\end{itemize}
	\item \textbf{3 Artefakte}:
	\begin{itemize}
		\item \textbf{Product Backlog}: Sortierte Sammlung aller Features, wird durchgehend bearbeitet
		\item \textbf{Sprint Backlog}: Enthält Aufgaben des aktuellen Sprints, wird täglich aktualisiert; das Hinzufügen neuer Aufgaben ist nicht gestattet, diese gehören in den \textbf{nächsten Sprint!}
		\item \textbf{Auslieferbares Produktinkrement}
	\end{itemize}
\end{itemize}