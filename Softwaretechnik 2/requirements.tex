\section{Requirements}
\label{rq:sec:requirements}

\begin{itemize}
	\item \textbf{Requirement}: Etwas, das gewollt oder benötigt wird
	\item \textbf{Requirements Engineering (RE)}: Finden und Dokumentieren von \textbf{Requirements} eines Softwareprojekts
	\item Mangelhaftes RE bedeutet Probleme, welche in späteren Phasen teuer korrigiert werden müssen
	\item \textbf{Prozess}:
	\begin{itemize}
		\item \textbf{Gewinnung (Elicitation)}: Finden und Festhalten
		\item \textbf{Dokumentation (Documentation)}: Erstellung einer \textbf{Software Requirements Specification}
		\item \textbf{Übereinstimmung (Agreement)}: Konfliktlösung, Kompromissfindung (z.B mit Stakeholdern)
		\item \textbf{Validation \& Management}
	\end{itemize}
	\item \textbf{Requirements Engineer} ist Mediator zwischen Nutzern und Kunden, verantwortlich für Gewinnung, Dokumentation und Übereinstimmung
	\item \textbf{Techniken}:
	\begin{itemize}
		\item \textbf{Questioning techniques}: Interviews, Fragebögen, On-Site-Customer
		\item \textbf{Creativity techniques}: Brainstorming, Analogien
		\item \textbf{Retrospective techniques}: System archeology, Konkurrenzsysteme
		\item \textbf{Observation techniques}: Field observation, Praktika
	\end{itemize}
	\item \textbf{Gute Form für Dokumentation}:
	\begin{itemize}
		\item \textbf{Kurze, klare Sätze} mit einem Requirement pro Satz
		\item Aktive Sprache, Klarstellung \textbf{wer wofür verantwortlich} ist
		\item Keine \textbf{unpräzisen Worte} wie \quotestyle{benutzerfreundlich} oder \quotestyle{schnell}
		\item Nutzen eines \textbf{Glossars}
	\end{itemize}
	\item Dokumentation meist in Form von \textbf{User Stories} (Agile) oder \textbf{Use Cases} (Model-based)
\end{itemize}

\subsection{Klassifizierung}
\label{rq:sub:klassifizierung}

\textbf{Typen (Kinds) von Requirements}:
\begin{itemize}
	\item Entscheidung basierend auf \textbf{Concerns}
	\item \textbf{Functional requirements}: Verhalten/Funktionalität, Reaktionen
	\item \textbf{Quality requirements}: Performanz, Verlässlichkeit, Benutzbarkeit, Sicherheit, Wartbarkeit
	\item \textbf{Constraint}: Physikalische Grenzen, Legalität, Umweltbedingungen
\end{itemize}
\textbf{Weitere Klassifikationen}:
\begin{itemize}
	\item \textbf{Representation}:
	\begin{itemize}
		\item Operational (beschreibt einen Ablauf)
		\item Quantitative (beschreibt Zahlenvorgaben, z.B Performance in Sekunden)
		\item Qualitative (beschreibt grobe Qualität, z.B einfache Benutzbarkeit)
		\item Declarative (beschreibt grobe Featurewünsche)
	\end{itemize}
	\item \textbf{Satisfaction}: Hard (objektiv prüfbar, ob erreicht), Soft (debattierbar, ob erreicht)
	\item \textbf{Role}: Prescriptive, Normative, Assumptive
\end{itemize}