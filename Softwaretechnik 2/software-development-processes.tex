\section{Software Development Processes}
\label{sdp:sec:software_development_processes}

\begin{itemize}
	\item \textbf{Code-and-Fix}: Simpelster Ansatz, hat aber viele \textbf{Nachteile}:
	\begin{itemize}
		\item Oftmals schlechte Codestruktur, Verbesserungen sind nicht systematisch
		\item Planung von Tasks und damit Teamarbeit kaum möglich
		\item Kein Design, oft keine Dokumentation, schwer maintainbar, skaliert schlecht
	\end{itemize}
	\item Stattdessen: Strukturierte \textbf{Development Processes} (Vorgehensmodelle)
	\begin{itemize}
		\item Abstrakte Repräsentation des Prozesses
		\item Stellt Richtlinien für \textbf{Aktivitäten, Rollen, Produkte} und ggf. Techniken und Tools auf
		\item Darunter: \textbf{Life Cycle Models}, welche nur Phasen und Phasenübergangskriterien aufstellen (z.B Wasserfall-Modell)
	\end{itemize}
\end{itemize}

\subsection{Wasserfall-Modell}
\label{sdp:sub:wasserfall_modell}

\begin{itemize}
	\item \textbf{Phasen}: Planung, Definition, Entwurf, Implementierung, Testen, Wartung
	\item Standardmäßig keine Korrektur älterer Phasenergebnisse erlaubt (\textbf{Rückkopplung})
	\item \textbf{Problem}: Frühzeitiges Abschätzen der Projektdetails (v.a. Zeitaufwand) selten genau möglich
\end{itemize}

\subsection{Unified Process}
\label{sdp:sub:unified_process}

\begin{itemize}
	\item \textbf{Idee}:
	\begin{itemize}
		\item Statt sequentiellem Prozess, führe \textbf{iterativ und inkrementell} immer wieder kleine Planungs-, Design-, Implementations- und Testphasen durch
		\item Konzentriere auf jeweils aktuelle \textbf{Risiken (risk-driven)} und \textbf{Ansprüche des Kunden (client-driven)}
		\item Prozess besteht aus 4 abstrakten Phasen (\textbf{nicht äquivalent} zum Wasserfall-Modell!) und 9 Disziplinen (\quotestyle{Mini-Wasserfälle})
	\end{itemize}
	\item \textbf{Phasen}:
	\begin{enumerate}
		\item \textbf{Inception}: Klärung von \textbf{Grundfragen} bzgl. Kosten, Vision, Durchführbarkeit, Bauen oder Kaufen etc.
		\item \textbf{Elaboration}: Entwicklung der riskanteren Kernkomponenten, Festigen der Voraussetzungen
		\item \textbf{Construction}: Iterative Implementierung, Vorbereitung von Deployment
		\item \textbf{Transition}: Beta-Tests, Deployment
	\end{enumerate}
	\item \textbf{Disziplinen}:
	\begin{enumerate}
		\item \textbf{Business Modelling}: Firmenspezifische Prozesse
		\item \textbf{Requirements}: Grobe Planung, Analyse, Dokumentation, Validierung, Management
		\item \textbf{Design}: Software-Modellierung (Klasendiagramme etc.)
		\item Rest selbsterklärend: \textbf{Implementation, Test, Deployment, Configuration \& Change Management, Project Management, Environment}
	\end{enumerate}
\end{itemize}

\subsection{Rational Unified Process}
\label{sdp:sub:rational_unified_process}

\begin{itemize}
	\item \textbf{Spezifizierung des UP}, welche zusätzlich Rollen, Aktivitäten und Artefakte festlegt
	\item Im Detail zu kompliziert und entsprechend nicht genauer relevant
\end{itemize}