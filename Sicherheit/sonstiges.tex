\section{Sonstiges}%
\label{snst:sec:sonstiges}

\subsection{Modulo Rechentricks}%
\label{snst:sub:modulo}
\textbf{Handhabbare Repäsentanten}
\[51^2 \bmod 59 = (59-8)^2 \bmod 59 = (-8)^2 \bmod 59 = 64 \bmod 59 = 5 \bmod 59\]
\textbf{Reduzierung des Exponenten modulo der Gruppenordnung}
\[3^{60} \bmod 59 = 3^{60 \bmod 58} \bmod 59 = 3^2 \bmod 59\]
Allgemein: Ordnung ist \(\varphi(N)\)

\subsection{Klausur 2018 Aufgabe 1b}%
\label{snst:sub:klausur}
Geben Sie eine Formel für \(x\) an, sodass gegeben \(P, Q, a, b\) gilt:\\
\(a = x \bmod P\) und \(b = x \bmod Q\)\\
Lösung:\\
\(x = ( a \cdot Q  \cdot (Q^{-1} \bmod P) + b \cdot P \cdot (P^{-1} \bmod Q)) \mod N\)\\
Für die Herleitung sollte der Wikpedia Artikel zum
\href{https://de.wikipedia.org/wiki/Chinesischer_Restsatz}{Chinesischen Restsatz}
zu Rate gezogen werden.