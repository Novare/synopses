%%% Local Variables:
%%% mode: latex
%%% TeX-master: "propa"
%%% End:

\lstset{language=Haskell,mathescape=true}

\section{Funktionale Programmierung in Haskell}

\subsection{Funktion}
Entspricht in Sprachen wie Haskell der mathematischen Sicht:
\begin{itemize}
  \item Bildet Element aus Definitions- in Wertebereich ab
  \item Auswertung keine Effekte auf Daten des Programms
  \item Wert von \(f(x)\) alleine von \(x\) abhängig
\end{itemize}
Es gibt keine Variablen in Haskell, Zustand über Parameter \& Rückgabewert.  

\subsection{Funktionsdefinition}
\code{f x = sin x / x}\\
Defininiere Funktionen mit Leerzeichen um Parameter zu trennen.\\

        
\subsection{Rekursive Definitionen}
Definition mit if-else-then:
\begin{lstlisting}
          binom n k =
            if (k == 0) || (k == n)
            then 1
            else binom (n-1) (k-1) + binom (n-1) k
\end{lstlisting}
Oder mit Type Guard Notation:
\begin{lstlisting}
          binom n k
            | (k == 0) || (k == n) = 1
            | otherwise = binom (n-1) (k-1) + binom (n-1) k
\end{lstlisting}
Vergleichbar mit Mathematischer Notation

\subsection{Ausführung}
Ausführung mit \code{ghci}\\
Programm \code{simple.hs}:
\begin{lstlisting}
          square x = x * x
          cube x = x * square x
\end{lstlisting}
Laden und Ausführen:
\begin{lstlisting}[language=bash]
          ghci
          :l simple.hs
          cube (1+2)
\end{lstlisting}

\subsection{Auswertung}
Schrittweise ausgewertet
\begin{itemize}
  \item \(e_1 \Rightarrow e_2\) für einen Schritt
  \item \(e_1 \Rightarrow^+ e_n \) falls \(e_1 \Rightarrow e_2 \Rightarrow ... \Rightarrow e_n\)
\end{itemize}
        
\subsection{Akkumulatoren}
Auswertung von rekursiven Funktionen \(\Rightarrow\) Zwischenausdrücke wachsen mit Eingabegröße.\\
Deshalb: Verwende Akkumulatoren\\
Idee: Speichere partielle Ergebnisse
\begin{lstlisting}
          fakAcc n acc = if (n==0) then acc else fakAcc (n-1) (n*acc)
          fak n = fakAcc n 1
\end{lstlisting}

\subsection{Endrekursion}
Funktion heiß endrekursiv, falls in jedem Zweig der rekursive Aufruf nicht in einen anderen Aufruf eingebettet ist.

\subsection{Listen}
Liste ist entweder
\begin{itemize}
  \item leere Liste \code{[]}, oder
  \item List \code{(x:xs)}, Restlist \code{xs} und Listenkopf \code{x}
\end{itemize}
\code{(:)} ist Listenkonstruktor, genannt const.
Lässt sich zum Pattern Matching auf Listen verwenden.\\
\textbf{Operationen}
\begin{itemize}
  \item \code{null l} - Teste ob leer
  \item \code{head} und \code{tail} berechnen Kopf und Restliste.
  \item \code{app left right} - Elemente aus \code{left} gefolgt von Elementen aus \code{right}
  \item \code{rev list} - Elemente aus \code{list} in umgedrehter Reihenfolge
  \item \code{take n l} - Erste \code{n} Elemente von \code{l}
  \item \code{drop n l} - \code{l} ohne erste \code{n} Elemente
\end{itemize}
          
\subsection{Pattern Matching}
Verwende mehrere Gleichungen zur Definition einer Funktion.\\
Jede Gleichung gibt Struktur für Argumente vor. Erstes passendes Muster wird angewandt.
Muster sind Konstanten, Variablen oder Konstruktoren.
\begin{lstlisting}
          maximum [] = error 'empty'
          maximum (x:[]) = x
          maximum (x:xs) = max x (maximum xs)
\end{lstlisting}


\subsection{Funktionen höherer Ordnung}
\textbf{Lambda Notation}
Schreibe Funktionsparameter auf rechte Seite
\begin{lstlisting}
          f = \x -> x * x - 3*x
          g = \x y -> x - 2/y
\end{lstlisting}
        Lassen sich auf Parameter anwenden
\begin{lstlisting}
          (\x -> x * x) 3 $\Rightarrow$ 3 * 3
\end{lstlisting}
\textbf{Defintion}\\
Funktionen, die andere Funktionen als Parameter erhalten oder Funktionen als Rückgabewerte liefer, heißen Funktionen höherer Ordnung.
Vgl. Mathematik - Ableitung\\\\
\textbf{Funktionen als Parameter}\\
Funktionsanwendung auf Listen:\\
\code{map :: (s -> t) -> [s] -> t}\\
Filtern von Listen:\\
\code{filter :: (t -> Bool) -> [t] -> [t]}\\\\
\textbf{Funktion als Rückgabewert}\\
Funktionskomposition: \code{f . g}\\
n-fache Anwedung: \code{iter :: (t -> t) -> Integer -> (t -> t)}

\subsection{Currying}
Ersetzung einer mehrstelligen Funktion durch Schachtelung einstelliger Funktionen\\
z.B. \(f = \lambda a, b.F(a,b)\) und \(f_c = \lambda a.\lambda b. F(a, b)\)\\
\(f_c\) kann unterversorgt werden!\\\\
\textbf{Unterversorgung}\\
Wende `mehrstellige` Funktionen auf zu wenige Parameter an.

\subsection{Namensbindung}
Bindungskonstrukte legen Bedeutung und Geltungsbereich von Variablen fest.
\begin{itemize}
  \item Globale Bindung durch Defintion
  \item Parameter im Rumpf gebunden
  \item x in Lambda Ausdruck gebunden
\end{itemize}

\subsection{Lokale Bindung}
\textbf{let} ermöglicht Deklarierung in Block.\\
\textbf{where} vor Allem für Hilfsfunktionen.
\begin{lstlisting}
          let c = 299792458 in m * c * c
          energy m = m * c * c
            where c = 299792458
\end{lstlisting}

\subsection{Verdeckung}
Innere Bindungen verdecken äußere.
Vorsicht insbesondere bei \code{let} Klauseln!
\begin{lstlisting}
          unknown = let x = 3 in
                      let x = 3 * x in 4 +x
\end{lstlisting}
\code{x} in \code{3 * x} gebunden durch inneres \code{let}
\(\Rightarrow\) Auswetung terminiert nicht.

\subsection{Folds}
Verallgemeinere Summe und andere Operationen auf Listen zu \textit{Initialwert, Operator}.
\begin{lstlisting}
          foldr op i [] = i
          foldr op i (x:xs) = op x (foldr op i xs)

          foldl op i [] = i
          foldl op i (x:xs) = foldl op (op i x) xs
\end{lstlisting}
Gibt selbes Ergebnis, falls Operation assoziativ.\\
Besonders bei Kurzschlussauswertung empfiehlt sich \code{foldr} zu verwenden, obwohl es nicht die Endrekursion ausnutzt.

\subsection{List Comprehensions}
Generierung von Listen \code{\([e | q_1, ..., q_m]\)}, wobei die \code{\(q_i\)} Tests von der Form
\code{p <- list}, Muster \code{p} und Listenausdruck \code{list}, sind.

\subsection{Lazy Evaluation}
Auswertung von Audrücken geschieht nur dann, wenn sie auch wirklich benötigt wird. Ermöglicht durch \textbf{Sharing}:
Ausdrücke werden als Graph repräsentiert.\\
Wann ist eine Auswertung nötig?
\begin{itemize}
  \item Vergleichs-Operatoren
  \item Arithmetische Operatoren
  \item in Bedingungen
  \item Beim Pattern-Matching bis Muster gematched
  \item Boolsche Operatoren -> Short-circuit
\end{itemize}

\subsection{Streams}
Durch Lazy Evaluation lässt sich unendlich lange Liste definieren, so genannte Streams.
Selbes Konzept wird auch in Java und anderen neuen Programmiersprachen verwendet.

\subsection{Typen}
Statische Typisierung: Variable hat vom \textit{Compiler} einen festen Typ.\\
Dynamische Typisierung: Typ der Variable zur \textit{Laufzeit} veränderbar.\\
Vorteile Statisch
\begin{itemize}
  \item Code verständlicher
  \item Compiler kann effizienter arbeiten
  \item Keine Abstürze wegen falschem Typ
\end{itemize}
\textbf{Haskell-Typen}
\begin{itemize}
  \item Int
  \item Integer
  \item Float
  \item Double
  \item Bool
  \item Char
  \item struktuierte Typen: Listen, Tupel, ...
\end{itemize}

\subsection{Polymorphe Typen}
Verwende Typvariablen um Polymorphie auszudrücken.
\code{[t]} ist Liste von Typ \code{t} für alle Typen \code{t}.
Dementsprechend ist dann \code{s -> t} Typ einer Funktion von Typ \code{s} nach Typ \code{t}.
        
\subsection{Infixnotation}
Jeder binäre Operator lässt sich mit Hilfe von Backquote-Notation als Infixoperator schreiben.
z.B. \code{l `app` r}.\\
Außerdem lassen sich Operatoren auch als Infix deklarieren mit \code{infix n s} mit Bindungsstärke \code{n} und Symbol \code{s}.

\subsection{Typangabe}
\textbf{Typinferenz}\\
Typen werden durch den Compiler errechnet. Spart explizite Deklaration und ermöglicht dadurch kompaktere Programme.\\
\textbf{Typdeklartion}\\
Manualles Angeben des Typs. Erhöhen Lesbarkeit und helfen beim Finden von Fehlern.

\subsection{Typsynonyme}
Weise einem Typ einen neuen Namen zu. IST KEIN NEUER TYP!
\begin{lstlisting}
        type String = [Char]
\end{lstlisting}

\subsection{Backtracking}
Ansatz zum Lösen von Problemen mit Haskell.
Bestimme mögliche Kombinationen, prüfe ob Kombination zulässig, püfe ob zulässe Lösung.

\subsection{Algebraische Datentypen}
Deklaration mit Hilfe von \code{data}
\begin{lstlisting}
        data Shape = Circle Double --- Radius
                     | Rectangle Double Double --- Seitenlaengen
\end{lstlisting}
\code{x :: Shape} bedeutet dann, dass \code{x} entweder ein Kreis oder ein Rechteck ist.
Einfachste Datentypen sind Aufzählungen.\\
Außerdem lassen sich Funktionen auf Datentypen deklarieren. Dabei wird Pattern Matching verwendet.
\begin{lstlisting}
        weather :: Season -> Temp
        weather Spring = Cold
        weather Summer = Hot
        weather Autumn = Cold
        weather Winter = Cold
\end{lstlisting}

\subsection{Binäre Bäume}
Ein binärer Baum ist entweder ein Blatt, oder ein Knoten mit Teilbäumen rechts und links.
\begin{lstlisting}
        data Tree t = Leaf
                    | Node (Tree t) t (Tree t)
        someTree = Node (Node Leaf 1 Leaf) 3 Leaf 
\end{lstlisting}

\subsection{Typklassen}
Idee: QuickSort oder andere Algorithmen lassen sich allgemein definieren, sobald gewisse Operationen verfügbar.
\(\Rightarrow\) Definieren Typklassen, welche diese Operatoren vorschreiben.
\begin{lstlisting}
        qsort :: Ord t => [t] -> [t]
\end{lstlisting}
Damit kann qsort verwendet werden, falls \code{t} Instanz von \code{Ord}\\
Standard-Typklassen:
\begin{itemize}
  \item Eq --- Gleichheit und Ungleichheit
  \item Ord --- Geordnete Typen
  \item Num --- Numerische Typen \((+,*,-)\)
  \item Show --- Typen lassen sich anzeigen mit show
  \item Enum --- Jeder Wert hat Nachfolger
\end{itemize}
\textbf{Typklassen-Definition}
\begin{lstlisting}
        class Eq t where
          (==) :: t -> t -> Bool
          (/=) :: t -> t -> Bool
\end{lstlisting}
\textbf{Typklassen-Instanziierung}
\begin{lstlisting}
        instance Eq Bool where
          True === True = True
          False === False = True
          False == True = False
          True === False = False
\end{lstlisting}
        

\subsection{Generische Instanziierung}
Tupel und Listen lassen sich auch generisch instanziieren.
\begin{lstlisting}
         instance (Eq s, Eq t) => Eq (s, t) where
           (a, b) == (a', b') = (a==a') && (b==b')
\end{lstlisting} 
