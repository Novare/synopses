\documentclass[10pt,a4paper]{article}
\author{Sebastian Markgraf}
\title{Programmierparadigmen}

\usepackage[utf8]{inputenc}
\usepackage{amsmath}
\usepackage{amssymb}
\usepackage[a4paper, total={6in, 8in}]{geometry}

\def\realnumbers{{\rm I\!R}}
\def\naturalnumbers{{\rm I\!N}}
\def\complexnumbers{{\mathbb{C}}}


\begin{document}
	\pagenumbering{Roman}
	{\let\newpage\relax\maketitle}
	\tableofcontents
	\newpage
	\pagenumbering{arabic}
	\setcounter{page}{1}

        \section{Funktionale Programmierung in Haskell}

        \subsection{Funktion}
        Entspricht in Sprachen wie Haskell der mathematischen Sicht:
        \begin{itemize}
          \item Bildet Element aus Definitions- in Wertebereich ab
          \item Auswertung keine Effekte auf Daten des Programms
          \item Wert von \(f(x)\) alleine von \(x\) abhängig
        \end{itemize}
        Es gibt keine Variablen in Haskell, Zustand über Parameter \& Rückgabewert.  

        \subsection{Funktionsdefinition}
        
\end{document}