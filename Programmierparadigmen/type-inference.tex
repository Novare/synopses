
%%% Local Variables:
%%% mode: latex
%%% TeX-master: "propa"
%%% End:

\section{Typinferenz}%
\label{inf:sec:typinferenz}

\subsection{Problem}%
\label{inf:sub:problem}
\textit{Gegeben}: Term \(t\) mit freien Variablen \(x_1, x_2, \ldots\) und Annahmen \(\Gamma = x_1 : \tau_1, x_2 : \tau_2, \ldots\)\\

\textit{Gesucht}: \(\tau\), so dass \(\Gamma \vdash t: \tau\) (Typisierung des Terms)\\

\textit{Lösungsansatz}: Suche rekursiv die Lösung für Subterme von \(t\). Mache dafür vorläufige Annahmen und Substituiere hinterher.\\

\textit{Typinferenz}: Finde \(\tau\) und Substitution \(\sigma\), so dass \(\sigma \Gamma \vdash t: \tau\)

\subsection{Typ-Substitution}%
\label{inf:sub:substitution}
Die Typ-Substitution eine endliche Abbildung von Typvariablen auf Typen. Die Substitutionen werden alle gleichzeitig angewandt, also müssen alle Substitutionen bis zur endgültigen Form
ausgeschrieben sein.\\

Beispiel: \(\sigma = [\alpha_1 \Rightarrow \text{bool}, \alpha_2 \Rightarrow \alpha_1 \rightarrow \alpha_1]\)\\
Eine Ergänzung auf \([\ldots, \alpha_3 \Rightarrow \alpha_1]\) wäre demnach nicht korrekt.

\subsection{Typinferenz mit Constraints}%
\label{inf:sub:constraints}
\begin{enumerate}
  \item Erstelle Herleitungsbaum. Verwende immer frische Typvariablen \(\alpha_i\)
  \item Stelle Constraints \(C\) für die \(\alpha_i\) auf
  \item Bestimme mgu der die Gleichungen löst
  \item Anwendung von \(\sigma\) gibt endgültigen Lösungsbaum
\end{enumerate}
Die Lösung ist dann durch \(\sigma, \tau\) für \((\Gamma, t)\) gegeben.
% TODO: Add Let-Polymorphismus

\subsection{Implementierung in Prolog}%
\label{inf:sub:prolog}
Constraints \(C\) werden implizit durch Prolog-Unifikator dargestellt.
Implementierung in Foliensatz 40\_Typinferenz auf Seite 330ff