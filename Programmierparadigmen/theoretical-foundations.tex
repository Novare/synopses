
%%% Local Variables:
%%% mode: latex
%%% TeX-master: "propa"
%%% End:

\newcommand\alphaeq{\overset{\alpha}{=}}
\newcommand\etaeq{\overset{\eta}{=}}

\section{Theoretische Grundlagen}

\subsection{Kalküle}
\begin{itemize}
  \item Minimalistische Programmiersprachen zur Beschreibung von Berechnungen
  \item Zum Führen von Beweisen
\end{itemize}
In dieser Vorlesung \(\lambda\)-Kalkül für sequentielle Sprachen

\subsection{untypisiertes \(\lambda\)-Kalkül}
\begin{tabular}{l l l}
  \textbf{Bezeichnung} & \textbf{Notation} & \textbf{Beispiele}\\
  Variablen & \(x\) & \code{x y}\\
  Abstraktion & \(\lambda x.\ t\) & \code{\(\lambda\)y. 0}\\
  Funktionsanwendung & \(t_1\ t_2\) & \code{f 42}
\end{tabular}\\\\
Die Funktionsanwendung ist \textit{linksassoziativ}: \code{\(\lambda\)x. f x y = \(\lambda\)x ((f x)y)}

% TODO: Add shadowing?

\subsection{\(\alpha\)-Äquivalenz}
Gedanke: 2 Terme sind äquivalent, wenn die Variablen in Lambdas des einen Terms in den anderen umbennant werden können.\\
Definition: \(t_1\) und \(t_2\) heißen \(\alpha\)-äquivalent (\(t_1 \alphaeq t_2\)), wenn \(t_1\) in \(t_2\) durch konsitente
Umbennung der \(\lambda\)-gebundenen Variablen überführt werden kann.

\subsection{\(\eta\)-Äquivalenz}
Gedanke: 2 Terme sind äquivalent, wenn Sie immer das gleiche Ergebnis haben.\\
Definition: \(\lambda x.\ f\ x\) und \(f\) heißen \(\eta\)-äquivalent(\(\lambda x.\ f \x \etaeq f\)), falls \(x\)
nicht freie Variable von \(f\).