
%%% Local Variables:
%%% mode: latex
%%% TeX-master: "propa"
%%% End:

\newcommand\alphaeq{\overset{\alpha}{=}}
\newcommand\etaeq{\overset{\eta}{=}}

\section{Theoretische Grundlagen}

\subsection{Kalküle}
\begin{itemize}
  \item Minimalistische Programmiersprachen zur Beschreibung von Berechnungen
  \item Zum Führen von Beweisen
\end{itemize}
In dieser Vorlesung \(\lambda\)-Kalkül für sequentielle Sprachen

\subsection{untypisiertes \(\lambda\)-Kalkül}
\begin{tabular}{l l l}
  \textbf{Bezeichnung} & \textbf{Notation} & \textbf{Beispiele}\\
  Variablen & \(x\) & \code{x y}\\
  Abstraktion & \(\lambda x.\ t\) & \code{\(\lambda\)y. 0}\\
  Funktionsanwendung & \(t_1\ t_2\) & \code{f 42}
\end{tabular}\\\\
Die Funktionsanwendung ist \textit{linksassoziativ}: \code{\(\lambda\)x. f x y = \(\lambda\)x ((f x)y)}

% TODO: Add shadowing?

\subsection{\(\alpha\)-Äquivalenz}
Gedanke: 2 Terme sind äquivalent, wenn die Variablen in Lambdas des einen Terms in den anderen umbennant werden können.\\
Definition: \(t_1\) und \(t_2\) heißen \(\alpha\)-äquivalent (\(t_1 \alphaeq t_2\)), wenn \(t_1\) in \(t_2\) durch konsitente
Umbennung der \(\lambda\)-gebundenen Variablen überführt werden kann.

\subsection{\(\eta\)-Äquivalenz}
Gedanke: 2 Terme sind äquivalent, wenn Sie immer das gleiche Ergebnis haben.\\
Definition: \(\lambda x.\ f\ x\) und \(f\) heißen \(\eta\)-äquivalent (\(\lambda x.\ f\ x \etaeq f\)), falls \(x\)
nicht freie Variable von \(f\).

\subsection{Ausführung von \(\lambda\)-Termen}
Redex: Von Englisch \textit{Reducable Expression} ein vereinfachbarer \(\lambda\)-Term\\
\(\beta\)-Reduktion: Ausführen der Funktionsanwendung auf einem Redex \[(\lambda x.\ t_1)\ t_2 \Rightarrow t_1 [x \rightarrow t_2]\]\\
Substitution: \(t_1 [x \rightarrow t_2]\) ersetze freie Vorkommen von \(x\) in \(t_1\) durch \(t_2\)\\
Normalform: Termn kann nicht weiter reduziert werden.

\subsection{Auswertungsstrategien}
Oft mehrere Terme in Ausdruck \(\Rightarrow\) Welchen wertet man zuerst aus?\\
Volle \(\beta\)-Reduktion: Alle Terme bis zum Ende ausgewertet
\begin{itemize}
  \item Normalreihenfolge: Immer der linkeste äußerste Redex --- Normalform ist eindeutig
  \item Call-By-Name: Reduziere linkesten äußeren Redex, falls nicht von \(\lambda\) umgeben
  \item Call-By-Value: Reduziere linkesten äußeren Redex, falls nicht von \(\lambda\) umgeben und Argument ein Wert
\end{itemize}
Die letzten beiden Varianten werten nicht immer zur Normalform aus!

\subsection{Church-Zahlen}
Natürliche Zahlen lassen sich als Lambda Terme ausdrücken.
Idee: Verstehe natürliche Zahlen, als Anzahl der Funktionsanwendungen \code{s}. Dann:
\begin{lstlisting}
  $c_0$ = $\lambda$s. $\lambda$z. z
  $c_1$ = $\lambda$s. $\lambda$z. s z
  ...
  $c_n$ = $\lambda$s. $\lambda$z. $s^n$ z

\end{lstlisting}
Um die nächste Church Zahl zu erhalten wird die Nachfolgerfunktion \code{succ} definiert.
Außerdem lassen sich auch Mathematische Operationen definieren.
\begin{lstlisting}
  succ = $\lambda$n. $\lambda$s. $\lambda$z. s (n s z)
  plus = $\lambda$m. $\lambda$n. $\lambda$s. $\lambda$z. m s (n s z)
  times = $\lambda$m. $\lambda$n. $\lambda$s. n (m s)
  exp = $\lambda$m. $\lambda$n. n m
\end{lstlisting}
Auch Booleans lassen sich so kodieren. Dabei sind diese eine einfache Fallunterscheidung
\begin{lstlisting}
  $c_{true}$ = $\lambda$t. $\lambda$f. t
  $c_{false}$ = $\lambda$t. $\lambda$f. f
\end{lstlisting}
Test auf 0
\begin{lstlisting}
  isZero = $\lambda$n. n ($\lambda$x. $c_{false}$) $c_{true}$
\end{lstlisting}

\subsection{Divergenz}
Es existieren Terme, welche nicht zu einer Normalform ausführen. Diesen Prozess nennt man Divergenz.
Dies ist die mathematische Modellierung für unendliche Ausführung, z.B.
\(\omega = (\lambda x.\ x\ x)(\lambda x.\ x\ x)\)

\subsection{Rekursion}
\[Y = \lambda f.\ (\lambda x.\ f\ (x\ x))(\lambda x.\ f\ (x\ x))\]
Damit erhält man Rekursion in der Form \(Y\ f = f\ (Y\ f)\) da \(f\) Fixpunkt von \(Y\) ist.

% TODO: Ergänze Konfluenz?

\subsection{Regelsysteme}
\textbf{Frege'scher Schlussstrich}\\
\(\frac{\varphi_1\ \varphi_2\ \varphi_3\ \ldots\ \varphi_n}{\varphi}\) \(\varphi_i\) Voraussetzungen, \(\varphi\) Konklusion.\\
\textbf{Herleitungsbäume}\\
Herleitungen von Konklusionen können durch mehrere Schlusstriche als Bäume dargestellet werden.\\
\textbf{Syntaktische Herleitbarkeit}\\
Mit diesen Mitteln lässt sich Syntaktische Herleitbarkeit \(\psi \vdash \varphi\) zeigen.

\subsection{Typen}
Weise sicheren Programmen Typen zu und lehne unsichere ab
\begin{itemize}
  \item Basistypen: bool, int,...
  \item Funktionstyp: \(\tau_1 \rightarrow \tau_2\) (rechtsassoziativ)
  \item Typvariablen: \(\alpha, \alpha_1, \beta,...\)
\end{itemize}
\textbf{Typsystem}\\
Typsystem \(\Gamma \vdash t: \tau\) --- im Typkontext \(\Gamma\) hat Term \(t\) Typ \(\tau\). Dabei ordnet \(\Gamma\)
freien Variablen \(x\) ihren Typ \(\Gamma(x)\) zu.
Um komplexere Typen zu realisieren werden mehrere Regeln verwendet, mit welchen Herleitungsbäume gebildet werden können:\\
\begin{align*}
  \textsc{Const}:& \frac{c \in \mathit{Const}}{\Gamma \vdash c: \tau_c} &
  \textsc{Var}:&   \frac{\Gamma(x) = \tau}{\Gamma \vdash x: \tau}\\
  \textsc{Abs}:&   \frac{\Gamma, x: \tau_1 \vdash t: \tau_2}{\Gamma \vdash \lambda x.\ t: \tau_1 \rightarrow \tau_2} &
  \textsc{App}:&   \frac{\Gamma \vdash t_1: \tau_2 \rightarrow \tau \ \ \ \ \Gamma \vdash t_2: \tau_2}{\Gamma \vdash t_1\ t_2: \tau}                                                                                                                      
\end{align*}
\textbf{Untypisierbar}\\
\(t\) typisierbar im Kontext \(\Gamma\), falls \(\tau\) mit \(\Gamma \vdash t: \tau\) existiert.
Demnach existieren auch nicht typisierbare Terme, z.B. \((\lambda x.\ x + 42) \mathit{true}\).
Dies wäre allerdings ein sicherer Term, welcher immer zu \(\mathit{true}\) auswertet. Also ist Typsystem nicht vollständig
bzgl. \(\beta\)-Reduktion.

\subsection{Korrektheit des Typsystems}
\textbf{Substitutionslemma}\\
Wenn \(\Gamma, x: \tau_2 \vdash t_1: \tau_1\) und \(\Gamma \vdash t_2: \tau_2\), dann \(\Gamma \vdash t_1[x \mapsto t_2]: \tau_1\).\\
\textbf{Typerhaltungstheorem}\\
Wenn \(\Gamma \vdash t: \tau\) und \(t \Rightarrow t'\), dann \(\Gamma \vdash t': \tau\).
