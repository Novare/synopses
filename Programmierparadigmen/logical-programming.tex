
%%% Local Variables:
%%% mode: latex
%%% TeX-master: "propa"
%%% End:

\lstset{language=Prolog,mathescape=true}

\section{Logische Programmierung}%
\label{lp:sec:logische-programmierung}

\subsection{Prolog Syntax}%
\label{lp:sub:prolog-syntax}
\textbf{Logische Programmierung}\\
Definiere Objekte und deren Beziehung, wird als Terme einer Termalgebra dargestellt.\\

\textbf{Fakten in Prolog}\\
Einzelne Fakten werden per \enquote{.} definiert.
Programmierer ist verantwortlich für deren Korrektheit.

\textbf{Termsyntax}
\begin{itemize}
  \item Atome: \code{hans, inge,...}
  \item Zahlen: 3, 4.5
  \item Variablen: X, Y, Fisch
  \item Term-Listen: 3, 4.5, Fisch,...
  \item Zusammengesetzt: liebt(fritz, fisch)
\end{itemize}
Atome stehen nur für sich selbst. Variablen sind hingegen Platzhalter für unbekannte Terme.\\

\textbf{Abfragen}\\
Alle Fakten in Datenbank zur Laufzeit. Abfragen daran können mit \enquote{?} eingeleitet werden, z.B. \code{?liebt(fritz, fisch).}
Bei erfolgreich gefunder Lösung wird die Variablenbelegung ausgegeben.\\

Mit \enquote{;} kann nach weiteren Lösungen gesucht werden. Bei keinen weiteren Lösungen wird \code{no} angegeben.
Mit \enquote{,} können Anfragen als Logisches Und konjugiert werden.\\

\textbf{Verarbeitung}\\
Teilziele werden von links nach rechts erfüllt. Dabei wird nach passenden Instanziierungen gesucht und diese werden
an das nächste Teilziel weiter vererbt. Falls keine Lösung für Teilziel wird \textit{Backtracking} zum vorherigen Teilziel
durchgeführt, demnach also eine neue Belegung dafür gesucht.\\

\textbf{Regeln}\\
Besteht aus \textit{Regelkopf} und \textit{Regelrumpf}. Dabei sind diese als Konklusion und Voraussetzungen zu sehen:
\code{term :- termlist .}\\

\textbf{Prädikate}\\
Eine Gruppe von Fakten/Regeln mit gleichem Funktor und gleicher Argumentzahl im Regelkopf heißt \enquote{Prozedur}
oder \enquote{Prädikat}.

\newpage
\subsection{Backtracking}%
\label{lp:sub:backtracking}
\textbf{Unifikation} findet Werte für Variablen, sodass 2 Terme gleich werden. Dabei ist zu beachten, dass Terme intern
als Bäume dargestellt werden.\\

\textbf{Informeller Algorithmus}
\begin{enumerate}
  \item Erstelle Box, call-Eingang
  \item Falls keine Regel, verlasse durch fail-Ausgang und lösche
  \item Für passende Regel anlegen von Kind-Boxen + Success zum call-Eingang des Kindes
  \item Falls keine Kinder \(\rightarrow\) Success auf call Eingang des Kindes
  \item Fail zeigt auf redo des vorherigen Teilziels
  \item Box durch redo \(\rightarrow\) Suche Choice Points. Falls kein Choice Point \(\rightarrow\) fail
  \item fail-Ausgang der ersten Box erzeugt \code{no}
  \item success der letzten Box gibt Substitution aus
\end{enumerate}

\subsection{Arithmetik und Listen}%
\label{lp:sub:arithmetik-listen}
Arithmetik in Prolog als syntaktischer Zucker für normale Prädikate.\\
Zusätzliche Erzwingung einer Auswertung mit is: \code{X is B*(C+3)}, aber Variablen
in rechten Term müssen immer instanziiert sein!\\
Für arithmetische Vergleiche müssen auch alle Variablen instanziiert sein.

\subsection{Funktionen}%
\label{lp:sub:funktionen}
Prolog hat außer der Erfüllbarkeit keinen Rückgabewert, also Rückgabewert als zusätzlicher Parameter (uninstanziierte Variable).

\subsection{Generate and Test}%
\label{lp:sub:generate-and-test}
Entwurfsmuster für Prolog. Probiere alle Lösungen durch, teste Lösungskandidaten. Benötigt oft Beschneidung des
Lösungsbaums (Branch and Bound).

\subsection{Cuts}%
\label{lp:sub:cuts}
\textbf{Determinismus}\\
Ein Prädikat heißt deterministisch, wenns es stets auf höchstens eine Weise erfüllt werden kann; hat es
möglicherweise mehrere Lösungen, so heißt es nichtdeterministisch.\\

\textbf{Blauer Cut}: Beeinflusst weder Programmlaufzeit, noch -verhalten\\
\textbf{Grüner Cut}: Beeinflusst Programmlaufzeit, aber nicht -verhalten\\
\textbf{Roter Cut}: Beeinflusst das Programmverhalten\\

\textbf{Faustregel}\\
Der Cut darf erst kommen, wenn man weiß, dass man in der richtigen Regel ist, aber muss vor der
Instanziierung der Ausgabevariablen stehen.

\subsection{Negation}%
\label{lp:sub:negation}
Negation in Prolog ohne Cut nicht möglich.
\begin{lstlisting}
  not(X) :- call(X),!,fail.
  not(X).
\end{lstlisting}

\subsection{Manipulation Terme}%
\label{lp:sub:manipulation-terme}
Untersuche Terme:
\begin{itemize}
  \item atom(X)
  \item integer(X)
  \item var(X)
  \item X = Y, unifizere X mit Y
  \item X == Y, erfolgreich, falls bereits unifiziert
  \item X =:= Y, arithmetischer Vergleich
  \item X=..L, Konstruktion/Zerlegung von X, aus/in Liste L
\end{itemize}

\subsection{Optimierung}%
\label{lp:sub:optimierung}
\textbf{Manuell}\\
Anstatt alle Variablenwerte erst zu generieren und dann zu überprüfen, schiebe Tests so weit nach vorne wie möglich.
Dadurch Begrenzung der Möglichkeiten.\\

\textbf{Freeze}\\
Werte datengetrieben aus, also wird \(T_i\) erst ausgewertet, wenn notwendig.
In Prolog mit Hilfe von \code{freezeAll([$X_1,X_2,..$], T)}

\subsection{Unifikation}%
\label{lp:sub:unifikation}
\textbf{Unifikator}\\
Eine Substitution \(\sigma\) unifiziert Gleichug \(\theta = \theta'\), falls \(\sigma\theta = \sigma\theta'\).
\(\sigma\) unifiziert \(C\), falls \(\forall c \in C\) gilt: \(\sigma\) unifiziert \(c\).\\

\textbf{Allgemeinster Unifikator (mgu)}\\
\(\sigma\) \textbf{mgu}, falls \(\forall\) Unifikator \(\gamma\ \exists\) Substitution \(\delta\). \( \gamma = \delta\ \circ\ \sigma\).\\
\newpage
\textbf{Unifikationsalgorithmus}\\
Informell:
\begin{enumerate}
  \item Ziehe Umbenennungen aus der Constraint Menge
  \item Wende diese auf die Constraint Menge an
  \item Schreibe Sie als Komposition hinten dran
  \item Wende Algorithmus erneut auf verbleibende Constraint Menge an
\end{enumerate}
Formell:
\begin{lstlisting}
  if $C == \varnothing$ then []
  else let $\{\theta_l == \theta_r\} \cup C' = C$ in
    if $\theta_l == \theta_r$ then unify($C'$)
    else if $\theta_l == Y$ and $Y \notin \mathit{FV}(\theta_r)$ then unify ($[Y \mapsto \theta_r]C') \circ [Y \mapsto \theta_r]$
    else if $\theta_r == Y$ and $Y \notin \mathit{FV}(\theta_l)$ then unify ($[Y \mapsto \theta_l]C') \circ [Y \mapsto \theta_l]$
    else if $\theta_l == f(\theta_l^1, \ldots, \theta_l^n)$ and $\theta_r == f(\theta_r^1,\ldots,\theta_r^n)$
         then unify($C' \cup \{\theta_l^1 = \theta_r^1, \ldots, \theta_l^n = \theta_r^n\})$
    else fail
\end{lstlisting}
  
\textbf{Effiziente Algorithmen}\\
Klassisicher Algorithmus ggf exponentiell.\\
Paterson-Wegman-Unifikation: Terme als gerichtete azyklische Graphen.
Verschmelzung von Variablen und sogar inneren Knoten der Terme mittels union/find.

\subsection{Resolution}%
\label{lp:sub:resolution}
Regel:
\[\frac{\tau_1, \tau_2, \ldots, \tau_n \text{Terme;}\quad \alpha \code{ :- } \alpha_, \ldots, \alpha_k \text{ eine Regel;}\quad \sigma \text{ mgu von } \alpha \text{ und } \tau_1}
  {\sigma(\alpha_1), \ldots, \sigma(\alpha_k), \sigma(\tau_2), \ldots, \sigma(\tau_n)}\]

\textbf{Resolutionsprinzip}\\
Sei P ein logisches Programm (i.e. eine Liste von Fakten und Regeln). Sei \(\tau_1, \ldots, \tau_n\) eine Liste von
Zielen (Termen).
\begin{itemize}
  \item Die Schreibweise \( P \vdash \tau_1, \ldots, \tau_n\) bedeutet: Die Ziele \(\tau_1, \ldots, \tau_n\) lassen sich
  mittels Resolution abarbeiten (mittels der Fakten und Regeln von \(P\) in die leere Zielliste verwandeln)
\item Die Schreibweise \( P \vDash \tau, \ldots, \tau_n \) beudetet: Die Zielliste \(\tau_1, \ldots, \tau_n\) ist die
  logische Konsequenz aus den Fakten und Regeln des Programms \(P\).
\end{itemize}
Die Resolutionsregel ist vollständig + korrekt.

\subsection{Vollständigkeit}%
\label{lp:sub:vollstaendigkeit}
Prolog ist nicht vollständig da es deterministisch arbeitet. Beweis kann leicht mit Gegenbeispiel erbracht werden:
\begin{lstlisting}
  aua(X) :- aua(X).
  aua(5).
\end{lstlisting}
Herleitbar mit Resolutionsregel aber nicht mit Prolog.