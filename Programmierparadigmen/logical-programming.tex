
%%% Local Variables:
%%% mode: latex
%%% TeX-master: "propa"
%%% End:

\section{Logische Programmierung}

\subsection{Prolog Syntax}
\textbf{Logische Programmierung}\\
Definiere Objekte und deren Beziehung, wird als Terme einer Termalgebra dargestellt.\\
\textbf{Fakten in Prolog}\\
Einzelne Fakten werden per \enquote{.} definiert.
Programmierer ist verantwortlich für deren Korrektheit.
\textbf{Termsyntax}
\begin{itemize}
  \item Atome: \code{hans, inge,...}
  \item Zahlen: 3, 4.5
  \item Variablen: X, Y, Fisch
  \item Term-Listen: 3, 4.5, Fisch,...
  \item Zusammengesetzt: liebt(fritz, fisch)
\end{itemize}
Atome stehen nur für sich selbst. Variablen sind hingegen Platzhalter für unbekannte Terme.\\
\textbf{Abfragen}\\
Alle Fakten in Datenbank zur Laufzeit. Abfragen daran können mit \enquote{?} eingeleitet werden.
z.B. \code{?liebt(fritz, fisch).}
Bei erfolgreich gefunder Lösung wird die Variablenbelegung ausgegeben. Mit \enquote{;} kann nach weiteren Lösungen
gesucht werden. Keine weiteren Lösungen werden mit \code{no} angegeben.
Mit \enquote{,} können Anfragen als Logisches Und konjugiert werden.\\
\textbf{Verarbeitung}\\
Teilziele werden von links nach rechts erfüllt. Dabei wird nach passenden Instanziierungen gesucht und diese werden
an das nächste Teilziel weiter vererbt. Falls keine Lösung für Teilziel wird \textit{Backtracking} von vorherigem Teilziel
durchgeführt, demnach also eine neue Belegung dafür gesucht.\\
\textbf{Regeln}\\
Besteht aus \textit{Regelkopf} und \textit{Regelrumpf}. Dabei sind diese als Konklusion und Voraussetzungen zu sehen.
\code{term :- termlist .}\\
\textbf{Prädikate}\\
Eine Gruppe von Fakten/Regeln mit gleichem Funktor und gleicher Argumentzahl im Regelkopf heiß \enquote{Prozedur}
oder \enquote{Prädikat}.

\subsection{Backtracking}
