
%%% Local Variables:
%%% mode: latex
%%% TeX-master: "robotik"
%%% End:

\section{Mathematische Grundlagen}

\subsection{Definitionen}
\textbf{Kinematik}\\
Analysiert die Geometrie eines Manipulators oder Roboters. Das essentielle Konzept ist die Position.\\

\textbf{Statik}\\
Behandelt Kräfte und Momente, die sich auf den ruhenden Mechanismus auswirken. Dass essentielle Konzept
ist die Steifigkeit.\\

\textbf{Dynamik}\\
Analysiert die Kräfte und Momente, die durch Bewegung und Beschleunigung eines Mechanismus und einer
zusätzlichen Last entstehen.\\

\textbf{Kinematische Kette}\\
Ist ein Satz an Gliedern die durch Gelenke verbunden sind.\\

\textbf{Freiheitsgrade}\\
Ist eine Anzahl unabhängiger Parameter, die zur kompletten Spezifikation der Position eines
Mechanismus/Objekts benötigt werden. (degrees of freededom DoF)

\subsection{\SO und \SE}
\SO{} ist die Spezielle Orthogonale Gruppe und repräsentiert Rotationen.\\
Elemente: 3\(\times\)3 Matrizen und \(R^{T}R = I\ \text{mit}\ \det(R) = 1\)\\
\SE{} ist die Spezielle Euklidische Gruppe und repräsentiert Bewegungen von Starrkörpern.\\
Elemente: \((p, R)\ \text{mit}\ p \in \realnumbers^3\ \text{und}\ R \in \SO\)

\subsection{Euklidischer Raum}
Ist der Vektorraum \(\realnumbers^3\) mit dem Standardskalarprodukt.

\subsection{Koordinatensysteme}
Es existieren rechtsdrehende und linksdrehende Koordinatensysteme. Lässt sich einfach mit der rechten-Hand-Regel herausfinden:
\begin{itemize}
  \item Daumen in z-Richtung
  \item Zeigefinger in x-Richtung
  \item Mittelfinger in y-Richtung
\end{itemize}
\(\Rightarrow\) Koordinatensystem rechtsdrehend.\\

\textbf{Typen von Koordinatensystemen}
\begin{itemize}
  \item BKS:\@ Basiskoordinatensystem, initialias Bezugssystem
  \item OKS:\@ Objektkoordinatensystem
  \item EKS:\@ Endeffektorkoordinatensystem
  \item SKS:\@ Sensorkoordinatensystem
\end{itemize}

\subsection{Lineare Abbildungen}
\textbf{Endomorphismus}\\
Lineare Abbildung des euklidischen Raums auf sich selbst\\

\textbf{Isomorhpismus}\\
Bijektive Endomorphismen. Spezielle Art von Isomorphismus ist \SO{}

\subsection{Rotation}
\textbf{in 2D}\\
Abbildung mit Matrix \(R_\alpha(x) = \begin{pmatrix} \cos \alpha & -\sin\alpha \\ \sin \alpha & \cos \alpha\end{pmatrix}\)
Ist nur eine lineare Transformation wenn um \(c=(0,0)\) rotiert wird.\\
Sonst muss erst vorher zum Ursprung und hinterher wieder zurück verschoben werden.\\

\textbf{Affine Transformationen}\\
\(T(x) = A(x) + b\) werden affine Transformtionen genannt.\\
Invertieren einer Affinen Transformation aus Rotation \(R\) und Translation \(T\).
\[ A^{-1} = R^{-1} - (R^{-1} T)\]

\textbf{in 3D}\\
Rotationen um Achsen identisch mit 2D Fall + eine weitere 1.

\subsection{Rotations Operationen}
\textbf{Inverse}\\
Inverse wird durch transponieren der Rotationsmatrix erreicht. \(R^{-1} = R^T\)\\

\textbf{Verkettung}\\
\(\phi_{z, \theta_3}(\phi_{y, \theta_2}(\phi_{x, \theta_1}(a))), a \in \realnumbers^3\)\\
Lesen von links nach rechts: Rotation um lokale Achsen\\
Lesen von rechts nach links: Rotation um globale Achsen\\

\textbf{Probleme mit Rotationsmatrizen}\\
Redundanz und keine einfache Erzeugung durch maschinelles Lernen

\subsection{Eulerwinkel}
Lege Rotation um Achsen fest. In \textbf{Euler-Konvention} \(z, x', z''\).
Gebe Rotation in Winkeln \(\alpha, \beta\) und \(\gamma\) an.
\[ R_s = R_z R_{x'} R_{z''}\]

\textbf{Roll, Pitch, Yaw}\\
Andere Konvention, verwendet \(x, y, z\).\\
Drehung um jeweils unveränderte Achsen.
Kann zum Gimbal Lock führen.\\
\[\myMatrix{\cos \alpha \cos \beta & \cos \alpha \sin \beta \sin \gamma - \sin \alpha \cos \gamma & \(\cos \alpha \sin \beta \cos \gamma + \sin \alpha \sin \gamma\)\\
  \sin \alpha \cos \beta & \sin \alpha \sin \beta \sin \gamma + \cos \alpha \cos \gamma & \(\sin \alpha \sin \beta \cos \gamma - \cos \alpha \sin \gamma\)\\
  -\sin \beta &  \cos \beta \sin \gamma& \cos \beta \cos \gamma\\
  }\]
\textbf{Gimbal Lock}\\
Verlust eines Freiheitsgrades, da bei bestimmten Winkeln 2 Achsen voneinaner abhängig werden.


\subsection{Affine Transformationen}
Affiner Raum ist Erweiterung des euklidischen Raums.
\begin{equation}
  b = Ax + t
    \Leftrightarrow
    b = \myVector{b \\ 1} = \myMatrix{A & o \\ o^T & 1} \myVector{x \\ 1} + \myVector{t \\ 0} =
    \myMatrix{A & t \\ o^T & 1}\myVector{x \\ 1}
\end{equation}
Kombiniere Translation und Rotation in einer Matrix. Dann Standardmatrix:
\(T = \myMatrix{A & \mathbf{t} \\ \mathbf{o}^T & 1}\)\\

\textbf{Nachteile}
\begin{itemize}
\item Redundant
\item Schwierige Interpolation
\end{itemize}

\subsection{Quaternionen}
Erweiterung der komplexen Zahlen \complexnumbers{} durch William Rowan Hamilton.\\
Menge der Quaternionen \quaternionen{} als \( \quaternionen = \complexnumbers + \complexnumbers j\)\\
Also ist ein Quaterionen \(q \in \quaternionen\) durch \(q = \myVector{a \\ u} = a + u_1 i + u_2 j + u_3 k\) 
mit \(a \in \realnumbers, u \in \realnumbers^3\) und \( k = i \cdot j\) gegeben.\\

\textbf{Rechenregeln}\\
Quaternionen \(q = {(a, u)}^T,\ r = {(b,v)}^T\).
\begin{itemize}
\item Addition:\@ \(q + r = {(a + b, u + b)}^T\)
\item Skalarprodukt:\@ \(\langle q \lvert r\rangle = a \cdot b + \langle v \lvert u \rangle = a \cdot b + v_1 \cdot u_1 + v_2 \cdot u_2 + v_3 \cdot u_3\)
\item Multiplikation:\@ \(q \cdot r = (a + u_1i + u_2j + u_3k) \cdot (b + v_1 i + v_2 j + v_3 k)\)
\item Konjugation:\@ \(q^* = {(a, -u)}^T\)
\item Norm:\@ \( \lvert q \lvert = \sqrt{q \cdot q^*} = \sqrt{q^* \cdot q} = \sqrt{a^2 + u_1^2 + u_2^2 + u_3^2}\)
\item Inverse:\@ \( q^{-1} = \frac{q^*}{{\lvert q \lvert}^2}\)
\end{itemize}

\textbf{Rotationen}\\
Dafür werden Einheitsquaternionen \(\mathbb{S}^3 = \{q \in \quaternionen\ \lvert\ {\|{q}\|}^2 = 1\}\) definiert.\\
Definiere Rotation um Achse \(a,\ \|a\| = 1\) und Winkel \(\phi\) durch \(q = (\cos \frac{\phi}{2}, a \sin \frac{\phi}{2})\)\\
Dann Rotatoin des Punkes \(v\) mit \(v' = qvq^{-1} = qvq^*\).\\
Vorgehen: (an \(p = {(1, 0, 9)}^T,\ a = {(1, 0, 0)}^T,\ \phi = \ang{90}\))
\begin{enumerate}
\item Darstellung von \(p\) als Quaternion \(v\) --- \(v = 0 + 1i + 0j + 9k\)
\item Rotationsquaternion \(q\) --- \( q = \cos \frac{\phi}{2} + 1i \cdot \sin \frac{\phi}{2} + 0j + 0k\)
\item Konjugiertes Quaternion \(q^*\) --- \(q^* = \cos \frac{\phi}{2} - 1i \cdot \sin \frac{\phi}{2} - 0j -0k\)
\item Rotation von \(v\) um \(q\) --- \(v_r = qvq^*\)
\item Darstellung als Punkt \(p_r\) --- \(p_r = {(1, -9, 0)}^T\)
\end{enumerate}

\textbf{Interpolation SLERP}\\
Interpoliere mit Hilfe von \(t \in [0, 1]\) und \(\langle q_1 \lvert q_2 \rangle = \cos \theta\)
\[ \mathit{Slerp}(q_1, q_2, t) = \frac{\sin(1-t)\theta}{\sin \theta} \cdot q_1 + \frac{\sin t\theta}{\sin \theta} \cdot q_2 \]

\textbf{Duale Quaternionen}\\
Ergänze Quaternionen um Duale Zahlen um Translation des Objektes auszudrücken.