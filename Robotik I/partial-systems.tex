
%%% Local Variables:
%%% mode: latex
%%% TeX-master: "robotik"
%%% End:

\section{Teilsysteme}

\subsection{Mechanische Komponenten}
\subsubsection{Gelenktypen}
\textbf{Rotationsgelenk (R)}\\
Besteht aus Eingang und dazu drehbarem Ausgang.\\
Drehachse in rechten Winkel mit den Achsen der beiden Glieder\\
Ellenbogen\\

\textbf{Torsionsgelenk (T)}\\
Drehachse parallel zu Achsen der beiden Glieder.\\
Unterarm\\

\textbf{Revolvergelenk (V)}\\
Einggangglied parallel zur Drehachse, Ausgangsglied im rechten Winkel dazu.\\
Schultergelenk\\

\textbf{Lineargelenk (L)}\\
Gleitende / Fortschreitende Bewegung\\
Translationsgelenk / Schubgelenk

\subsubsection{Arbeitsraum}
Jene Punkte im 3D-Raum die von der Roboterhand angefahren werden können. Dazu sind 3 Freiheitsgrade in der
Bewegung erforderlich, also mindestens drei Gelenke erforderlich.\\
Grundform: Ignoriere Kollisionen und Winkelbegrenzungen

\subsubsection{Radkonfigurationen}
\textbf{Differentialantrieb}\\
Geradeaus- \& Kurvenfahrten, Drehen auf der Stelle\\
+ einfach\\
- Radregelung in Echtzeit\\

\textbf{Dreirad-Antrieb}\\
Geradeaus- \& Kurvenfahrten, Vorwärts/Rückwärts unterschiedlich.\\
+ einfache Mechanik\\
- eingeschränkt\\

\textbf{Synchro-Antrieb}\\
Geradeaus- \& Kurvenfahrten, Vorwärts/Rückwärts identisch, Plattform dreht nicht mit\\
+ Einfache Regelung\\
+ Geradeaus mechanisch garantiert\\
- Mechanisch Komplex\\

\newpage
\textbf{Mecanum-Antrieb}\\
+ Uneingeschränkte Beweglichkeit\\
- Mechanisch komplex\\ 
- Aufwendige Regelung\\

\subsection{Antriebe}
\subsubsection{Fluidische Antriebe}
\textbf{Linearantrieb}\\
Bewege Kolben durch Flüssigkeit.\\
Geschwindigkeit \(v(t) = \frac{Q(t)}{A}\).\\
Kraft \(F(t) = p(t) \cdot A\) mit \(p(t)\) Druck des Mediums.\\

\textbf{Schaufelrad}\\
Bewegen einer Schaufel in einem Zylinder.\\
Winkelgeschwindigkeit \(\omega (t) = \frac{2 \cdot Q(t)}{h \cdot (R^2 - r^2)}\)\\
Drehmoment \(T(t) = p(t) \cdot h \cdot (R - r) \cdot \frac{ R + r} {2}\)

\subsubsection{Muskelartige Antriebe}
\textbf{Pneumatik}\\
Komprimierte Luft bewegt Kolben\\
Vorteile: billig, einfach, schnell\\
Nachteile: Laut, keine Steuerung der Geschwindigkeit\\
Einsatz: Kleine Roboter, wenig Kraft, schnell\\

\textbf{Hydraulik}\\
Öldruckpumpe und steuerbare Ventile\\
Vorteil: Große Kraft, mittelschnell\\
Nachteil: Laut, viel Platz, Ölverlust\\
Einsatz: Große Roboter

\subsubsection{Elektrische Antriebe}
Schritt- oder Servomotoren\\
Vorteil: wenig Platz, kompakt, leise, gut Regelbar\\
Nachteil: Wenig kraft, langsam\\
Einsatz: Kleine Roboter für Präzision

\newpage
\subsection{Getriebe}
\subsubsection{Stirnradgetriebe}
Standardgetriebe mit Getrieberad + Ritzel.\\
Übersetzung: \(i = \frac{N_2}{N_1}\) mit \(N_i\) Anzahl der Zähne\\
Drehmoment: \(T_2 = T_1 \cdot i\)

\subsubsection{Schrauben- \& Spindelgetriebe}
Schraube/Spindel dreht sich in Richtung.\\
Lineargeschwindigkeit \(v(t) = p \cdot \omega(t)\)

\subsubsection{Harmonic Drive}
Bestehe aus Wavegenerator, Flexible Spine und Circular Spine.\\
Auch \enquote{Spannungswellengetriebe}\\
Hohes Übersetzungsverhältnis in einer Stufe, sehr genau\\
Gutes Getriebe für Leichtbauroboter

\subsection{Sensoren}
Wandelt physikalische Größe \& Änderungen in elektronsiche Signale um.
\begin{itemize}
\item Elementarsensor --- Aufnahme von Messgröße
\item Integrierter Sensor --- Zusätzlich Signalaufbereitung
\item Intelligenter Sensor --- Mit rechnergesteuerter Auswertung + Ausgabe verarbeitete Größe
\end{itemize}

Anforderungen: Genauigkeit, Präzision, Betriebsbereich, Antwortgeschwindigkeit, Kalibrierung, Zuverlässigkeit, Kosten
Installationsaufwand.\\

Klassifizierung:\\
intern / extern je nach Kontakt zur Umwelt\\
aktiv / passiv wobei passiv nur messen und aktive ihre Umwelt simulieren.