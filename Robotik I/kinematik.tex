
%%% Local Variables:
%%% mode: latex
%%% TeX-master: "robotik"
%%% End:

\section{Kinematik}
\subsection{Kinematisches Modell}
Beschreibt die Zusammenhänge zwsichen dem Raum der Gelenkwinkel und dem Raum der Lage des Endeffektors in
Weltkoordinaten.\\

\textbf{Vorwärtskinematik}\\
Bestimme Endeffektor Position aus Winkeln.\\

\textbf{Inverse Kinematik}\\
Bestimme Winkel zu gewünschter Position.\\

\subsubsection{Kinematische Kette}
Wird von mehreren Körpern gebildet, die durch Gelenke kinematisch verbunden sind.\\
Mit allen Endpunkten fest \(\rightarrow\) geschlossen, sonst offen.\\

\textbf{Konventionen}
\begin{itemize}
\item Jedes Armelement entspricht einem starren Körper
\item Jedes Armgelenk durch Gelenk zu nächstem
\item Schub-/Rotationsgelenke nur 1 DoF
\end{itemize}

\textbf{Kinematische Parameter}\\
Jedes Glied 3 Rotations- und 3 Translationsparameter \(\rightarrow\) 6 Parameter pro Glied.

\subsubsection{Denavit-Hartenberg (DH) Konvention}
Koordinatensystem bestimmen durch
\begin{enumerate}
\item \(z_{i-1}\)-Achse liegt entlang der Bewegungsachse des \(i\)-ten Gelenks.
\item \(x_i\)-Achse verläuft entlang der gemeinsamen Normalen von \(z_{i-1}\) und \(z_i\). Dabei zeigt sie weg von \(z_{i-1}\)
\item Die \(y_i\)-Achse vervollständigt das Koordinatensystem entsprechend der Rechte-Hand-Regel.
\item Armelementlänge \(a_i\) beschreibt Abstand von \(z_{i-1}\) zu \(z_i\)
\item Armelementverwinding \(\alpha_i\) beschreibt Winkel von \(z_{i-1}\) zu \(z_i\) um \(x_i\)
\item Gelenkabstand \(d_i\) ist Abstand zwischen \(x_{i-1}\) und \(x_i\)-Achse entlang der \(z_{i-1}\)-Achse
\item Gelenkwinkel \(\theta_i\) ist Winkel von \(x_{i-1}\) zu \(x_i\) um \(z_{i-1}\)
\end{enumerate}

\subsubsection{Direktes Kinematisches Problem}
Bestimme aus DH-Parametern \& Gelenkwinkeln die Stellung des Endeffektors.
\[S_{\mathit{Basis},\ \mathit{Greifer}} = A_{0,1}(\theta_1) \cdot A_{1,2}(\theta_2) \cdot \ldots
  \cdot A_{n-2,n-1}(\theta_{n-1}) \cdot A_{n-1, n}(\theta_n)\]

\textbf{Jacobi-Matrix}\\
Die Jacobi-Matrix setzt kartesische End-Effektor-Geschwindigkeiten in Relation zu Gelenkwinkelgeschwindigkeiten.
\[\dot{x}(t)=J_f(\theta(t)) \cdot \dot{\theta}(t)\]
Die Jacobi-Matrix setzt Kräfte und Momente am End-Effektor in Relation zu Drehmomenten in den Gelenken
\[\tau(t) = J_f^T(\theta(t)) \cdot F(t)\]

\subsubsection{Singularitäten}
Singuläre Konfiguration liegt vor, wenn die Jacobi-Matrix nicht vollen Rang hat.\\
Dann ist sie nicht invertierbar und damit bestimmte Bewegungen unmöglich. Ggf. große Gelenkgeschwindigkeiten
für End-Effektor-Geschwindigkeit notwendig.\\

\textbf{Manipulierbarkeit}\\
Definiere \(A(\theta) = J(\theta){J(\theta)}^T \in R^{nxn}\) und betrachte Eigenwerte \(\lambda\).
Definiere Singulärwerte \(\sigma_i = \sqrt{\lambda_i}\).\\
Daraus Maße:
\begin{itemize}
\item Kleinster Wert \(\mu_1 (\theta) = \sigma_{\min}(A(\theta))\)
\item Inverse Kondition \(\mu_2(\theta) = \frac{\sigma_{\min}(A(\theta))}{\sigma_{\max}(A(\theta))}\)
\item Determinante \(\mu_3(\theta) = \det A(\theta)\)
\end{itemize}

\subsubsection{Erreichbarkeit}
Durch offline Abtastung in Gitter. Ermöglicht schnell Griff- oder andere Erreichbarkeitsanalysen durchzuführen.

\subsection{Geometrisches Modell}
Zur Kollisions- / Abstands- / Kraftbestimmung\\
Zur Darstellung von Körpern.\\

\textbf{Blockwelt}\\
Darstellung durch einhüllende Quader. 2,5D Volumen bzw. Flächen\\
Erster Schritt der Kollisionsvermeidung\\

\textbf{Kantenmodell}\\
Darstellung durch Polygonzüge. 3D, Kanten bzw. Flächen\\
Schnelle Visualisierung\\

\textbf{Volumenmodell}\\
Genaue Darstellung. 3D, Volumen\\
Genaue Kollisionserkennung, Animationen