
%%% Local Variables:
%%% mode: latex
%%% TeX-master: "robotik"
%%% End:

\section{Programmieren durch Vormachen}%
\label{pdv:sec:programmieren-durch-vormachen}

\subsection{Grundfragen}%
\label{pdv:sub:grundfragen}
Hauptziel:
\begin{itemize}
\item Intuitive Roboterprogrammierung ohne Expertenwissen
\item Aufgaben durch Generalisierung auf Basis mehrer Observationen
\end{itemize}

Zerlegt in 3 Probleme, die seperat betrachtet werden könen.


\tikzstyle{block} = [draw, fill=green!20, rectangle, 
minimum height=3em, minimum width=2cm]
\begin{figure}[!h]
  \centering
  \begin{tikzpicture}[auto, node distance=4cm, >=latex']      
    \node [block] (perzeption) {Perzeption};
    \node [block, right of=perzeption] (kognition) {Kognition};
    \node [block, right of=kognition] (aktion) {Aktion};
    
    \draw [->] (perzeption) -- (kognition);
    \draw [->] (kognition) -- (aktion);
  \end{tikzpicture}
\end{figure}

\textbf{Wer soll imitiert werden?}\\
Wähle Lehrer durch den am meisten lernt. Meist sowieso dezidierter Lehrer.\\

\textbf{Wann soll imitiert werden?}\\
Bewegung segmentiert werden und Start und Ende bestimmt werden.\\
Oft explizities Kennzeichnen von Start und Ende\\

\textbf{Was soll imitiert werden?}\\
Manche Eigenschaften irrelevant, finde heruas welche Aspekte interessant sind.\\

\textbf{Wie soll imitiert werden?}\\
Roboter muss Abbildung von Perzeption zu einer Sequenz seiner eigenen Bewegungen lernen.\\
Embodiment!

\subsection{Perzeption}%
\label{pdv:sub:perzeption}
Demonstration in 2 Schritten:
\begin{enumerate}
\item Aktion des Lehrers Aufnehmen (Record Mapping)
\item Aktion auf Roboter abbilden (Embodiment Mapping)
\end{enumerate}

4 Arten des Vormachens:
\begin{enumerate}
\item Teleoperation
\item Shadowing
\item Sensoren am Lehrer
\item Externe Beobachtung
\end{enumerate}

\textbf{Teleoperation (Demonstration)}\\
Direkte Steuerung des Roboters und Sensoren zeichnen Bewegung auf.\\
Kinästhetisches Lernen: Bewegen des Roboters, benötigt mehr DoF als Roboter\\

\textbf{Shadowing (Demonstration)}\\
Nachmachen von anderem Roboter, direktes Embodiment\\

\textbf{Sensoren auf dem Lehrer (Imitation)}\\
Aufnahme von Bewegungen des Lehrers\\
Ermöglicht hohe Genauigkeit und Freiheit für den Mensch.\\
Abbildung auf Roboter notwendig\\
Unterschieden in marker-basierte und marker-lose Verfahren.\\
Verfahren:
\begin{itemize}
\item optisch-passive Bewegungserfassung
\item optisch-aktive Bewegungserfassung
\item IMU-basierte Bewegungserfassung
\item Mechanische Bewegungserfassung
\item Magnetische/Akustische Bewegungserfassung
\end{itemize}

Master Motor Map Framework:\\
Vereinheitlichende Darstellung menschlicher Ganzkörperbewegungen\\
Referenzmodell Mensch mit 104 DoF\\

\textbf{Externe Beobachtung (Imitation)}\\
Beobachtung durch Kameras oder andere Sensoren. Benötigt komplexe Algorithmen zur Vearbeitung und erreicht nur
niedrige Präzision.\\

Verfahren:
\begin{itemize}
\item Marker-lose optische Bewegungserfassung
\item Skeleton Tracing
\item Segmentation \& Tracking
\end{itemize}


\subsection{Kognition}%
\label{pdv:sub:kognition}
Fähigkeit als Trajektorie oder symbolisch als Fähigkeiten repräsentiert

\subsubsection{Trajektorie}%
\label{pdv:ssub:trajektorie}
Signal in latenten Bewegungsraum projiziert\\
Spezifich zu zyklischer, diskreter oder beides beinhaltender Bewegung

\subsubsection{Symbolisch}%
\label{pdv:ssub:symbolisch}
Segmentation von Aufgaben in vordefinierte Aktionen\\
Reproduzieren und mit klassichen Machine-Learning-Verfahren\\

2 Arten von Segmetierung
\begin{itemize}
\item Bewegungssegmentierung
\item Semantische Segmentierung (Aufgabensegmentierung)
\end{itemize}


\subsection{Demonstrationsverarbeitung}%
\label{pdv:sub:demonstrationsverarbeitung}
In 2 Arten
\begin{itemize}
\item Batch Lernen --- Lernen nach gesamter Aufzeichnung
\item Inkrementelles Lernen --- Aktionsrepräsentation nach jeder Aktion/Wiederholung aktualisiert
\end{itemize}


\subsection{Aktion}%
\label{pdv:sub:aktion}
Ausführen und Adaption der Aktion

\subsubsection{Online Anpassung}%
\label{pdv:ssub:online-anpassung}
Anpassung durch Kopplungsterm in DMP\\
DMP Online: \[\tau\ddot{y} = K (g-y)-D\dot{y} + f_\theta(x)(g-y_0) + \color{red}C\]

\subsubsection{Ausführung von komplexen Aufgaben}%
\label{pdv:ssub:ausfuehrung-von-komplexen-aufgaben}
Roboter braucht Weltwissen, aber weiß nicht wo Objekte sind\\
\(\rightarrow\) Hypothesen anhand von Wissen generieren\\

Fehlerhaftes Verhalten muss erkannt und behandelt werden\\
\(\rightarrow\) Suchen von Alternativen / Neuplanen

