
%%% Local Variables:
%%% mode: latex
%%% TeX-master: "robotik"
%%% End:

\section{Bewegungsplanung}
Erzeuge kollisionsfreie Trajektorie unter Einschränkungen.

\subsection{Grundlagen}
\textbf{Konfiguration}\\
Eine Konfiguration \(q \in C\) beschreibt den Zustand eines Roboters.\\

\textbf{Konfigurationsraum}\\
Der Konfigurationsraum \(C\) eines Roboters \(R\) ist der Raum aller möglicher Konfigurationen von \(R\).\\

\textbf{Arbeitsraumhindernis}\\
Ein Arbeitsraumhindernis \(H\) ist der Raum, welcher von einem Objekt im Arbeitsraum eingenommen wird.\\

\textbf{Konfigurationsraum}\\
Ein Konfigurationsraumhindern \(C_H\) ist die Menge aller Punkte des Konfigurationsraumes \(C\),
welche zu einer Kollision mit dem Hindernis \(H\) führen.\\

\textbf{Hindernisraum}\\
Der Hindernisraum \(C_{\mathit{obs}}\) ist die Menge aller Konfigurationsraumhindernisse.
\[C_{\mathit{obs}} = \bigcup_i C_{H_i} \]

\textbf{Freiraum}\\
Der Freiraum ist die Menge aller Punkte aus \(C\), welche nicht im Hindernisraum \(C_{\mathit{obs}}\) liegen
\[C_{\mathit{free}} = \{q \in C\ |\ q \not \in C_{\mathit{obs}}\} = C \setminus C_{\mathit{obs}}\]

\textbf{Umweltmodellierung}\\
Exakt: Beispielsweise über CSG (constructed solid geometry), in Form einer algebraischen Beschreibung\\
Aproximiert: Die Umwelt wird durch Näherungen beschrieben (Boxen, Zylinder, \ldots)\\

\textbf{Pfadplanung}\\
Starres Objekt, 2D / 3D Problem.\\

\textbf{Bewegungsplanung}\\
Mehrkörpersystem (z.B. Roboterarme, Systeme mit mehreren Robotern).\\
Hochdimensionale Problemstellungen\\

\textbf{Komplexität}\\
Allgemeine Planungsaufgaben sind PSPACE-vollständig

\textbf{Alrogithmen}\\
\begin{itemize}
\item vollständig --- Mindestens eine Lösung oder in endlicher Zeit, dass keine Lösung
\item randomisiert --- Verwenden Zufallsgrößen und oft heuristische Annahmen zur Beschleunigung
\item auflösungsvollständig --- Vollständiger approximativer Algorithmus für diskretisierte Problemstllung
\item probabilistisch-vollständig --- findet mindestens eine Lösung falls Sie existiert. Falls kein Lösung existiert
  konvergiert er gegen 1
\end{itemize}

\subsection{Problemklassen}
\begin{tabular}{lllll}
  Buchstabe & Umwelt & Gesucht & Problem\\
  a & vollständig &  Kollisionsfreie Trajektorie Start zu Ziel & \\
  b & unvollständig & Kollisionsfreie Trajektorie Start zu Ziel & Kollision mit unbekannten Objekten\\
  c & zeitvariant &  Kollisionsfreie Trajektorie Start zu Ziel & Hindernisse in Ort und Zeit variant\\
  d & keins &  Kollisionsfreie Trajektorie Start zu Ziel & Kartographieren\\
  e & zeitvariant &  Trajektorie zu einem beweglichen Ziel & Zielzustand in Ort und Zeit beweglich\\
\end{tabular}

\subsection{Voronoi-Diagramme}
Zerlegung eines raumes in regionen basierend auf vorgegeben Punkten bzw. Bereichen\\
Region eines Bereiches sind alle die Punkte deren Abstand zum Hindernis geringe ist, als zu allen anderen Hindernisse.\\

\textbf{Konstruktion}\\
Rekursive Unterteilung von 2 oder 3 Punkten und dann verbinden an den nächsten.


\subsection{Sichtgraphen}
Verbinde jedes Paar von Eckpunkten im freien Raum durch gerades Liniensegment, falls kein Hindernis geschnitten wird.
\enquote{Die Punkte müssen sich sehen}\\
Start und Ziel werden analog verbunden\\

Damit Kanten nicht als Wege genommen werden, müssen die Hindernisse erweitert werden.

\subsection{Zellzerlegung}
Zerlegung des freien Raums in einzelne Zelle und suche dann in Zellen nach kürzesten Wegen. Ermöglicht dann
suchen eines Pfades im entstehenden Graphen.


\subsection{Suchen im Wegenetz}
Konstruierte Graph liegt von den vorhergehenden Verfahren vor. Nun muss ein Weg im Graph gefunden werden.

\subsubsection{Baumsuche}
Darstellung als Quadtree

\subsubsection{A*}
Beliebter Algorithmus\\
Findet optimalen Pfad von Start zu Zielknoten\\
\textbf{Algorithmus}\\
Solange \(O \neq \varnothing\)
\begin{itemize}
\item Besteimme den zu erweiternden Knoten --- Finde \(v_i \in O\) mit minimalem \(f(v_i) = g(v_i) + h(v_i)\)
\item Wenn \(v_i = v_{\mathit{Ziel}\)} Lösung gefunden: Traversiere Vorgänger von \(v_i\) bis \(v_{\mathit{start}}\) erreicht ist.
\item \(O\).remove(\(v_i\))
\item \(C\).add(\(v_i\))
\item Update für alle Nachfolger \(v_j\) von \(v_i\) durchführen
  \begin{itemize}
  \item Wenn \(v_j \in C\), dann überspringe \(v_j\)
  \item Wenn \(v_j \not \in O\), dann \(O\).add(\(v_j\))
  \item Wenn \(g(v_i) + \mathit{cost}(v_i, v_j) < g(v_j)\)
    \begin{itemize}
    \item \(g(v_j) = g(v_i) + \mathit{cost}(v_i, v_j)\)
    \item \(h(v_j) = \mathit{heuristic}(v_j, v_{\mathit{ziel}})\)
    \item \(\mathit{pred}(v_j) = v_i\)
    \end{itemize}
  \end{itemize}
\end{itemize}

\subsection{Potentialfelder}
Roboter bewegt sich unter dem Einfluss von Kräften, welche ein Potentialfeld auf ihn ausübt.\\
Ein Potentialfeld \(U\) ist eine Skalarfunktion über dem Freiraum \[U: C_{\mathit{free}} \rightarrow \realnumbers\]
Die Kraft in einem Punkt \(q\) des Potentialfeldes ist der negative Gradient in diesem Punkt
\[F(q) = - \nabla U(q)\]
Es muss abstoßende Potentiale um Hindernis und anziehende Potentiale mit Minimum in \(q_{\mathit{Ziel}}\) geben.\\
z.B. Lineare Funktion der Distanz zum Ziel \( U_{\mathit{an}}() = k \cdot || q - q_{\mathit{Ziel}} ||\)
oder quadratischer Funktion \(U_{\mathit{an}}(q) = k \cdot \frac{1}{2} {|| q - _{\mathit{Ziel}}||}^2\)\\
Durch Bewegung des Roboters in Richtung des negativen Gradienten bewegt er sich zu den lokalen Minima.
Durch Flucht aus den lokalen Minima bewegt er sich zum nächsten Minium.


\subsection{Probabilistic Roadmaps (PRM)}
Mehrere Anfragen in gleicher Umgebung\\
Approximation des Freiraums durch Graphen\\
\begin{enumerate}
\item Erzeuge Graph durch zufällige verbundene Stichproben
\item Suche Weg im Graph
\end{itemize}

\subsection{Dynamic Roadmaps}
Mehrere Anfragen bei gleichbleibender kinematischen Kette\\
Approximation des Konfigurationsraums durch Graph und Arbeitsraum durch Voxel\\

Berechnet Roadmap durch Sampling und erzeugt eine Abbildung zwichen den Knoten/Kanten und allen Voxeln\\
Bei einer Anfrage müssen alle Voxel mit Hindernis bestimmt werden und die Kanten / Knoten aus der Roadmap entfernt werden.
Schließlich werden Start und Ziel im Graphen verbunden.\\


\subsection{Rapidly-exploring Random Trees (RRTs)}
Algorithmus zur Einmalanfrage ohne Vorverarbeitung\\
probabilistisch, vollständig, randomisiert\\

\begin{enumerate}
\item Beginne mit leerem Baum nur mit Start
\item Erzeuge zufälligen Punkt
\item Verbinde mit nächstem Nachbarn, falls kollisionsfrei
\item Überprüfe ob Ziel erreicht, sonst erzeuge nächsten Punkt
\end{enumerate}

Durch Nachbearbeitung können Lösungen verbessert werden. Wähle zufällige Punkte im Lösung und teste ob diese
verbunden werden können. Dadurch kann die Trajektorie geglättet werden.

\subsection{Erweiterungen von RRT}
\subsubsection{Constrained RRT}
Erfüllen von Nebenbedingungen.\\
Idee: Projeziere auf Mannigfaltigkeit, die Nebendingung erfüllt und suche dort

\subsubsection{RRT*}
RRT finden nicht optimale Trajektorien.\\
Idee: Optimiere Suchbaum während der Suche. Optimierung durch Ermittlung der Wegkosten jedes neuen Knotens und
passender Neuverbindunge des Suchbaums.

\subsubsection{Enge Passagen}
Durch zufällige Stichproben schwierig enge Passagen zu überwinden.\\
Idee: Sample nur in Kegel um Knoten (sichtbaren Bereich)
\(\rightarrow\) Dynamic Domain RRT\\

Idee2: Wähle Zielpunkte in Passage
\(\rightarrow\) Bridge Sampling\\

