
%%% Local Variables:
%%% mode: latex
%%% TeX-master: "robotik"
%%% End:

\tikzstyle{block} = [draw, fill=blue!20, rectangle, 
    minimum height=3em, minimum width=6em, align=center]
\tikzstyle{sum} = [draw, fill=blue!20, circle, node distance=1cm]
\tikzstyle{input} = [coordinate]
\tikzstyle{output} = [coordinate]
\tikzstyle{pinstyle} = [pin edge={to-,thin,black}]

\section{Regelung}
Lehre von der selbsttätigen, gezielten Beeinflussung dynamischer Prozesse während des Prozessablaufs.

\subsection{Aufbau und Wirkungsweise}
Aufgabe: Die Ausgangsgröße eines dynamischen Systems soll mittels der Stellgröße ein Sollverhalten, d.h. ein
gewünschtes Verhalten aufgeprägt werden, und zwar gegen den Einfluss einer Störgröße, die nur unvollständig bekannt
ist.\\

Lösung: Die Strecke ist laufend zu beobachten und mit der so gewonnen Information ist die Stellgröße derart zu
verändern, dass trotz der Störgrößeneinwirkung die Ausgangsgröße an den gewünschten Verlauf (Sollverlauf)
angeglichen wird.\\

\begin{tikzpicture}[auto, node distance=2cm, >=latex']
  \node [input, name=input] {};
  \node [sum, right of=input] (sum) {};
  \node [block, right of=sum] (controller) {Korrektur-\\einrichtung \\ (Regelglied)};
  \node [block, right of=controller, node distance=3cm] (system) {Stelleinrichtung};
  \node [block, right of=system, node distance=3cm, pin={[pinstyle]above:z}] (way) {Strecke};

  \draw [->] (controller) -- (system);
  \draw [->] (system) -- node[name=y] {$y$} (way);
  \node [output, right of=way] (output) {};
  \node [block, below of=y] (measurements) {Messeinrichtung};

  \draw [draw, ->] (input) -- node {$w$} (sum);
  \draw [->] (sum) -- node {$x_d$} (controller);
  \draw [->] (way) -- node [name=x] {$x$} (output);
  \draw [->] (x) |- (measurements);
  \draw [->] (measurements) -| node[pos=0.99] {$-$}
  node [near end] {$r$} (sum);
\end{tikzpicture}\\

\begin{tabular}{clcl}
  \(w\) & Führungsgröße & \(x_d\) & Regeldifferenz\\
  \(y\) & Stellgröße & \(x\) & Regelgröße\\
  \(r\) & Rückführgröße & \(z\) & Störgröße\\
\end{tabular}

\textbf{Wirkungsweise}\\
\begin{tabular}{ll}
  Sollwert von \(x\): & \(x_s\)\\
  Messeinrichtung: & \(r = K_j x\)\\
  Führungsgröße: & \(w = K_j x_s\)\\
  Dann: & \(x_d = w - r = K_j x_s - K_j x = K_j (x_s - x)\)\\
\end{tabular}\\
Die Regelgröße folgt der Führungsgröße. Die Regelung ist ein Wirkungskreislauf.\\

\subsection{Regelung}
Unter einer Regelung versteht man eine Anordnung, durch welche unvollständig bekannter Strecke,
insbesondere unvollständiger Kenntnis der Störgröße, die Regelgröße, d.h. die Ausgangsgröße der Strecke,
laufend erfasst und mit der Führungsgröße verglichen wird, um mittels der so gebildeten Differenz die
Regelgröße an den Sollverlauf anzugleichen.

\subsection{Laplace-Transformation}
\[L\{f(t)\} = F(s) = \int_0^{\infty} f(t)e^{-st}dt \qquad s := \sigma + j\omega; \quad f(t) = 0,\ t < 0\]

\textbf{Regeln}\\
\begin{tabular}{lc}
  Linearitätssatz & \(L\{\alpha f_1(t) + \beta f_2(t)\} = \alpha F_1(s) + \beta F_2(s)\)\\
  Faltungssatz & \(L\{f_1(t) * f_2(t)\} = F_1 (s) \cdot F_2(s)\)\\
  Grenzwertsatz & \(f_1(t=0) = \lim_{s \rightarrow \infty} s * F(s)\)\\
  Differentiationssatz & \(L \{\frac{d}{dt} f(t)\}= s F(s) \) \\
  Integrationssatz & \(L \{\int f(t)dt\} = \frac{1}{s} F(s)\)\\
  Verschiebung & \(L \{ f(t-\tau)\} = e^{-\tau s} F(s) \)\\
  \(L\{e^{\alpha t}\} = \frac {1}{s-\alpha}\) & \(L\{t^n\} = \frac{n!}{s^{n+1}} \quad (n = 1,2, \ldots)\)\\
  \(L\{\sin(\alpha t)\} = \frac{\alpha}{s^2 + \alpha^2}\) & \(L\{\cos (\alpha t)\} = \frac{s}{s^2 + \alpha^2}\)\\ 
\end{tabular}

\subsection{Elementare Glieder}
\begin{tabular}{lll}
  P-Glied & Proportionsgleid & \(y(t) = K \cdot u(t)\)\\
  I-Glied & Integrierglied & \(y(t) = K \cdot \int_0^t u(\tau)d\tau\)\\
  D-Glied & Differenzglied & \(y(t) = K \cdot \dot(u)(t)\) \\
  \(T_t\)-Glied & Totzeitglied & \(y(t) = K \cdot u(t - T_t)\)\\
  S-Glied & Summenglied & \(y(t) = \pm u_1(t) \pm u_2(t)\)\\
  KL-Glied & Kennlinienglied & \(y(t) = K \cdot f(u(t))\)\\
  M-Glied & Multiplizierglied & \(y(t) = K \cdot u_1(t) \cdot u_2(t)\)\\
\end{tabular}

\subsection{PID-Regelung}
Proportional-Integral-Derivative Controller
\[\tau = K_p \theta_d + K_i \int \theta_d(t)dt + K_d \dot{\theta}_d\]

\begin{tikzpicture}[auto, node distance=2cm, >=latex']
  \node [sum, right of=input] (sum) {};

  \node [block, node distance=6cm, right of=sum] (kp) {\(K_p\)};
  
  \node [block, below of=kp] (ki) {\(K_i\)};
  \node [block, node distance=3cm, left of=ki] (int) {\(\int dt\)};
  
  \node [block, below of=ki] (kd) {\(K_d\)};
  \node [block, below of=int] (dt) {\(\frac{d}{dt}\)};
  
  \node [sum, right of=kp, node distance=3cm] (sum2) {};

  \node[block, node distance=3cm, right of=sum2] (dynamic) {Dynamik des Arms};
  
  \draw [draw, ->] (input) -- node {$\theta_v(t)$} -- node[pos=0.90] {\(+\)} (sum);

  \draw [->] (sum) -- node [pos=0.2, name=td] {\(\theta_d(t)\)} (kp);
  \draw [->] (td) |- (int);
  \draw [->] (td) |- (dt);

  \draw [->] (int) -- (ki);
  \draw [->] (dt) -- (kd);

  \draw [->] (kp) -- node[pos=0.90] {\(+\)} (sum2);
  \draw [->] (ki) -| node[pos=0.90] {\(+\)} (sum2);
  \draw [->] (kd) -| node[pos=0.90] {\(+\)} (sum2);

  \draw [->] (sum2) -- node {\(\tau\)} (dynamic);
 % \draw [->] (measurements) -| node[pos=0.99] {$-$}
 % node [near end] {$r$} (sum);

  \draw [->] (dynamic) -- node[pos=0.10] {\(\theta\)} +(0,-5) -| (sum);
\end{tikzpicture}


\subsection{Stabilität}
Ziel: Regelabweichung geht mit der Zeit gegen 0


\subsection{Regelung von Manipulatoren}
Manipulatoren = Mehrgrößensystem
\(\Rightarrow\) Unabhängige Einzelregelkreise\\

\textbf{Dynamikmodell}\\
Wirken von Gravitation-, Zentrifugal-, Coriolis- und Reibungskräften.
\[ Q = M(q)\ddot{q} + n(\dot{q}, q) + g(q) + R\dot{q}\]

\textbf{Konzepte}\\
Exakte Systemmodellierung: A priori exakte Kenntnis des Dynamikmodells + Umgebung.\\
Kraft-/Positionsregelung: Ausführen von Aufgaben mit Interaktionskräften.\\
Hybride Kraft-/Positionsregelung: Regele jede kartesische Richtung einzeln.\\

\textbf{Impedanz-reglung}\\
Regelt die dynamische Beziehung zwischen Kraft und Position im Kontaktfall.\\
\[f(t) = d \cdot x(t) + b \cdot \dot{x}(t) + m \cdot \ddot{x}(t)\]