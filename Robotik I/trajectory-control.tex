
%%% Local Variables:
%%% mode: latex
%%% TeX-master: "robotik"
%%% End:

\section{Bahnsteuerung}

\subsection{Grundproblem}
Gegeben:
\begin{itemize}
\item \(S_{\mathit{Start}}\): Zustand zum Startzeitpunkt
\item \(S_{\mathit{Ziel}}\): Zustand zum Zielzeitpunkt
\end{itemize}
Gesucht:
\begin{itemize}
\item \(S_i\): Zwischenzustände, damit die Trajektorie \enquote{glatt} wird.
\end{itemize}

Zustände können in 2 Räumen dargestellt werden:
\begin{itemize}
\item Konfigurationsraum: \(\realnumbers^n\) nahe an der Ansteuerung der Teilsysteme
\item Arbeitsraum: \(\realnumbers^3\), \SE näher an der zu lösenden Aufgabe
\end{itemize}


\subsection{Bahnsteuerung}
\textbf{Konfigurationsraum}\\
Funktion der Gelenkwinkelzustände. Werden entweder synchron oder asynchron abgefahren.\\

\textbf{Arbeitsraum}\\
Funktion der Zustände des Roboters. Continuous Path: Endeffektor folgt Lage und Orientierung einer Bahn.\\



\subsection{Programmierung der Schlüsselpunkte}
Anfahren markanter Punkte der Bahn mit manueller Steuerung


\subsection{Interpolationsarten}
\textbf{Punkt-zu-Punkt (PTP)}\\
Roboter führt Punkt zu Punkt Bewegung aus. Einfache Berechnung\\

\textbf{PTP mit Rampenprofil}\\
Einfache Art der Berechnung. Aufschaltung der Beschleunigung kann zu Eigenschwingungen führen.\\

\textbf{PTP mit Sinoidenprofil}\\
Weichere Bewegung und geringere Beanspruchung. Dauert aber länger.\\

\textbf{Linearinterpolation}\\
Interpolation zwischen je 2 Teiltrajektorien.\\

\textbf{Zirkularinterpolation}\\
Benötigt Stützpunkt und interpoliert zu Kreisbahn.\\

\textbf{Segementweise Bahninterpolation}\\
Gleiche Endbedingungen der Teiltrajektorien aneinander an.\\
z.B. mit Splines


\subsection{Approximation}
\textbf{Bézierkurven}\\
Verlaufen nicht durch Punkte, sondern werden nur von diesen beeinflusst.\\

\textbf{De-Casteljau-Algorithmus}\\
Annäherung an die Bézierkurve\\
Teile Bézierkurve und stelle sie durch 2 aufeinanderfolgende dar.