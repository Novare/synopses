
%%% Local Variables:
%%% mode: latex
%%% TeX-master: "robotik"
%%% End:

\section{Roboterprogrammierung}
Komplexe Umgebung \& viele Bewegungsfreiheitsgrade \(\rightarrow\) Programmierung?\\
Klassische Roboterprogrammierung erfordert Experten + hohen Aufwand\\
Besser: interaktiv oder durch beobachten des Menschen

\subsection{Klassische Verfahren}
\textbf{Kategorien}\\
\begin{tabular}{lcc}
  Ort & on-line & off-line\\
      & Direkt am Roboter & Mit graphischer oder interaktiver Modelle\\
  \\
  Abstraktionsgrad & explizit & implizit\\
      & Bewegungen direkt in Programmiersprache & Durch Aufgaben\\
\end{tabular}\\

\textbf{Arten}
\begin{itemize}
\item direkt
\item textuell
\item graphisch
\item gemischte
\end{itemize}

\subsubsection{Direkte}
\textbf{Direkte Programmierung}\\
Durch Stopper an den Gelenkwinkeln \enquote{Bang-Bang}\\

\textbf{Teach-in}\\
Anfahren wichtiger Punkte mit Teachbox\\

\textbf{Play-back}\\
Einstellen des Roboters auf Zero-Force-Control und Speichern der Gelenkwinnkel\\

\textbf{Master-Slave}\\
Programmierung schwerer Roboter durch bewegen eines kleineren Roboters \(\rightarrow\) synchrone Ausführung\\

\textbf{Sensorunterstützt}\\
Erfassen der Werksstücke oder anzufahrender Punkte


\subsubsection{Textuelle}
Programmierung mittels erweiterter höherer Programmiersprachen wie PASRO, VAl, V+


\subsubsection{Hybride Verfahren}
Graphische Programmierung basierend auf sensorieller Erfassung der Benutzervorführung \(\rightarrow\) Simulation


\subsubsection{Graphische Programmierung}
Virtuelles Teach-in oder graphische Modellierung


\subsection{Statecharts}
Graphischer Formalismus zum Entwurf komplexer Systeme\\
States mit hierarchischer, nacheinander, parallel Ausführung.\\
Original: Kein Datenfluss\\

\textbf{Erweiterung am \(H^2T\)}\\
Datenfluss Transitionsbasiert\\
Keine Inter-level Transitionen\\
Erfolgs- und MIsserfolgszustände in jedem Zustand


\subsection{Symbolische Planung}
Repräsentation des Weltzustandes durch Booleschen Prädikten um Aktionssequenzen zu planen.\\

\textbf{STRIPS}\\ 
einfache, eingeschränkte Sprache\\
Funktionsfreie Sprache erster Ordnunge (Prädikatenlogik erster Stufe)
\begin{itemize}
\item Konstanstensymbole \(A, B, C, \ldots\) (Namen der Blöcke)
\item Variablensymbole: \(u, v, x_1, x_2, \ldots\)
\item Prädikate
  \begin{itemize}
  \item \textit{handempty}
  \item \textit{ontable(bottle)}
  \item \textit{on(bottle, table)}
  \item \textit{in(water,cup)}
  \item \textit{at(robot, fridge)}
  \item \ldots
  \end{itemize}
\item Operatorensymbole: \textit{pickup, putdown, stack, unstack, \ldots}
\end{itemize}

Weltrepräsentation: Konjunktion positiver aussagenlogischer Literale\\
Keine Variablen, negativen Literale, Funktionen!\\

Ziele: Teilweise spezifizierter Zustand\\

Aktion: Deklaration, Vorbedingungen, Effekte\\

Ausführbarkeit: Aktion ausführbar, wenn Vorbedingungen erfüllt.





  