\newglossaryentry{framebuffer}
{
  name=Framebuffer,
  description={Digitale Rasterbild-Kopie des Monitorbildes}
}

\newglossaryentry{farbtiefe}
{
  name=Farbtiefe,
  description={Anzahl möglicher Farb- bzw. Grauwerte, charakterisiert durch die Bit-Anzahl}
}

\newglossaryentry{dithering}
{
  name=Dithering,
  description={Absichtlich eingesetztes Rauschen zur Erzeugung eines im Farbraum sonst nicht verfügbaren Farbeindrucks (z.B Schwarz-Weiß-Dithering für Graustufen)}
}

\newglossaryentry{color banding}
{
  name=Color Banding,
  description={Bandeffekte aufgrund von Problemen bei der Farbdarstellung ohne Dithering, oftmals aufgrund zu geringer Farbtiefe}
}

\newglossaryentry{gamma-korrektur}
{
  name=Gamma-Korrektur,
  description={Anpassen eines digitalen Farbwertes, sodass der Farbeindruck auf dem Display für das menschliche Auge bei steigendem Pixelwert linear wirkt}
}

\newglossaryentry{transparenz}
{
  name=Transparenz,
  description={Physikalische Eigenschaft der teilweisen oder vollständigen Lichtdurchlässigkeit}
}

\newglossaryentry{opazitaet}
{
  name=Opazität,
  description={Gegenteil von Transparenz}
}

\newglossaryentry{metamerismus}
{
  name=Metamerismus,
  description={Phänomen, bei dem unterschiedliche Farbspektren dieselbe Farbe ergeben}
}

\newglossaryentry{farbmodell}
{
  name=Farbmodell,
  description={Modell zur Beschreibung von Farben durch Wertetupel, z.B RGB-Farbmodell}
}

\newglossaryentry{farbraum}
{
  name=Farbraum,
  description={Raum aller möglichen Farben eines Farbmodells}
}

\newglossaryentry{tristimuluswerte}
{
  name=Tristimuluswerte,
  description={Wertetupel eines Farbmodells}
}

% \newglossaryentry{}
% {
%   name=,
%   description={}
% }