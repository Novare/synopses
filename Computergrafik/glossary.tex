\newglossaryentry{framebuffer}
{
  name=Framebuffer,
  description={Digitale Rasterbild-Kopie des Monitorbildes}
}

\newglossaryentry{farbtiefe}
{
  name=Farbtiefe,
  description={Anzahl möglicher Farb- bzw. Grauwerte, charakterisiert durch die Bit-Anzahl}
}

\newglossaryentry{dithering}
{
  name=Dithering,
  description={Absichtlich eingesetztes Rauschen zur Erzeugung eines im Farbraum sonst nicht verfügbaren Farbeindrucks (z.B Schwarz-Weiß-Dithering für Graustufen)}
}

\newglossaryentry{color banding}
{
  name=Color Banding,
  description={Bandeffekte aufgrund von Problemen bei der Farbdarstellung ohne Dithering, oftmals aufgrund zu geringer Farbtiefe}
}

\newglossaryentry{gamma-korrektur}
{
  name=Gamma-Korrektur,
  description={Anpassen eines digitalen Farbwertes, sodass der Farbeindruck auf dem Display für das menschliche Auge bei linear steigendem Pixelwert ebenfalls linear steigend intensiver wirkt}
}

\newglossaryentry{transparenz}
{
  name=Transparenz,
  description={Physikalische Eigenschaft der teilweisen oder vollständigen Lichtdurchlässigkeit}
}

\newglossaryentry{opazitaet}
{
  name=Opazität,
  description={Gegenteil von Transparenz}
}

\newglossaryentry{metamerismus}
{
  name=Metamerismus,
  description={Phänomen, bei dem unterschiedliche Farbspektren dieselbe Farbe ergeben}
}

\newglossaryentry{farbmodell}
{
  name=Farbmodell,
  description={Modell zur Beschreibung von Farben durch Wertetupel, z.B RGB-Farbmodell}
}

\newglossaryentry{farbraum}
{
  name=Farbraum,
  description={Raum aller möglichen Farben eines Farbmodells}
}

\newglossaryentry{tristimuluswerte}
{
  name=Tristimuluswerte,
  description={Wertetupel eines Farbmodells}
}

\newglossaryentry{bildsynthese}
{
  name=Bildsynthese,
  description={Erzeugung eines Bildes aus Rohdaten (auch: Rendering)}
}

\newglossaryentry{raytracing}
{
  name=Raytracing,
  description={Bildsynthese durch Simulation des Lichttransports}
}

\newglossaryentry{shading}
{
  name=Shading,
  description={Simulation der Oberflächenwahrnehmung, ermöglicht realistische Tiefenwahrnehmung}
}

\newglossaryentry{brdf}
{
  name=BRDF,
  description={Bidirektionale Reflexions-Distributionsfunktion}
}

\newglossaryentry{btdf}
{
  name=BTDF,
  description={Bidirectional Transmission-Distribution-Function}
}

\newglossaryentry{bsdf}
{
  name=BSDF,
  description={Bidirectional Scattering-Distribution-Function, Summe aus BRDF und BTDF}
}

\newglossaryentry{ambienteslicht}
{
  name=Ambientes Licht,
  description={Grundbeleuchtung durch indirekte Beleuchtung}
}

\newglossaryentry{diffuseslicht}
{
  name=Diffuses Licht,
  description={Grobe Beleuchtung}
}

\newglossaryentry{spekulareslicht}
{
  name=Spekulares Licht,
  description={Imperfekte Spiegelung, Highlights}
}

\newglossaryentry{reflexion}
{
  name=Reflexion,
  description={Zurückwerfen von Licht an einer Oberfläche}
}

\newglossaryentry{transmission}
{
  name=Transmission,
  description={Durchlassen von Licht durch einen Körper}
}

\newglossaryentry{aliasing}
{
  name=Aliasing,
  description={Fehler aufgrund von unzureichender Abtastung}
}

\newglossaryentry{}
{
  name=,
  description={}
}