\section{Prozedurales Modellieren}%
\label{pm:sec:prozedurales_modellieren}

\subsection{Textursynthese}%
\label{pm:sub:textursynthese}

\textbf{Ziel}: Synthesiere eine Texture die genau wie Eingabetexture aussieht\\
$\rightarrow$ Stochastische Muster, nicht wiederholen der Textur\\

\textbf{Pixelbasiert}\\
Erzeuge Textur Pixel für Pixel durch Betrachten der Nachbarschaft\\
Einfach, aber langsam\\

\textbf{Patchbasiert}\\
Verwende größere Bereiche und setze diese wieder zusammen

\subsection{Rauschen und Rauschtexturen}%
\label{pm:sub:rauschen_und_rauschtexturen}

\textbf{Gewünschte Eigenschaften an Rauschfunktion}
\begin{itemize}
\item reproduzierbar
\item keine Periodizität
\item begrenzter Wertebereich
\item definierte Frequenzverteilung, bandlimitiert
\item Stetigkeit, räumliche Korrelation
\end{itemize}

\textbf{Noise-Funktion}
\begin{itemize}
	\item Grundlage für stoch. Modellierung / prozedurale Texturen
	\item Erzeugen von Bildern aus Pseuo-Zufallszahlen
	\item 2 Klassen
        \begin{itemize}
          \item Lattice Value Noise (Werte auf 1D/2D/3D Gitter)
          \item Lattice Gradient Noise (nicht klausurrelevant)
        \end{itemize}
    \item Auswertung in Abhängigkeit von Frequenz $f$ und Offset $o$: $n(f \cdot x + o)$
\end{itemize}

\textbf{Turbulenz}\\
Summiere $k$ Frequenzen $f$ (Oktaven) einer Noise-Funktion $n$ an Stelle $x$:\\
$$\text{turbulence}(x) = \sum_k (1 / f)^k n(f^k \cdot x)$$
Höhere Frequenzen $\rightarrow$ Feinere Strukturen + kleinere Amplitude\\

\textbf{Texturen}\\
Durch geschicktes Kombinieren (Addieren, Multiplizieren) und Mappings lassen sich Texturen oder auch
ganze Landschaften erzeugen

\subsection{Hypertextures}%
\label{pm:sub:hypertextures}

Beschreiben von komplexen Objekten durch Dichtefunktionen. Verwende Turbulunce innerhalb des Objekts
um Rauschen zu erzeugen und Strukturen wie Wolken oder Feuer zu erzeugen. Außerhalb verwende 0.

\subsection{Distanzfelder, Distanzfunktionen}%
\label{pm:sub:distanzfelder_distanzfunktionen}

Stelle Geometrie implizit als Abstandsfunktion dar z.B. Kugel $f(x) = | x - m | - r$\\
Vorzeichenbehaftet, negativ also im Inneren. Bei $f(x) = 0$ spricht man von der Isofläche\\

\textbf{Sphere Tracing}\\
Finden des Schnittpunkts durch Sphere Tracing $\rightarrow$ Lege immer Strecke $f(x)$ zurück
$\rightarrow$ Da kürzester Abstand zu nächster Fläche überspringen wir garantiert keinen Schnittpunkt\\

\textbf{Normalen}\\
Gradient zeigt in Richtung der schnellsten Zunahme $\rightarrow$ Normale. Approximiere Gradient
$$\nabla f(x) = \left[
    \begin{tabular}{c}
      d / dx\\
      d / dy \\
      d / dz\\
      \end{tabular}
    \right] f(x)
    \approx
    \left[
      \begin{tabular}{c}
        f (x + 1/2 h, y, z) - f (x - 1/2 h, y, z)\\
        f (x, y + 1/2 h, z) - f (x, y - 1/2 h, z)\\
        f (x, y, z + 1/2 h) - f (x, y, z - 1/2 h)\\
      \end{tabular}
    \right]
  $$

Durch Verknüpfen von einzelnen Funktionen lassen sich komplexe Geometrien modellieren
\begin{itemize}
        \item Vereinigung $f_{A \cup B} (x) = \min (f_A(x), f_B(x))$
        \item Schnitt $f_{A \cap B} (x) = \max (f_A(x), f_B(x))$
        \item Differenz $f_{A \setminus B} (x) = \max (f_A(x), - f_B(x))$
\end{itemize}

\subsection{L-Systeme}%
\label{pm:sub:l_systeme}
\begin{itemize}
    \item Theoretisches Modell für biologische Entwicklung
    \item Definiere ein komplexes Objekt durch sukzessives Ersetzen von Teilen eines einfacheren Objekts
    \item Arbeitet ähnlich wie formale Grammatiken
\end{itemize}