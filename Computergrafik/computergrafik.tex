\documentclass[10pt,a4paper]{article}
\author{Jannik Koch}
\title{Computergrafik}

\usepackage[utf8]{inputenc}
\usepackage[ngerman]{babel}
\usepackage{multicol}
\usepackage{amsmath}
\usepackage[a4paper, total={6in, 8in}]{geometry}
\usepackage{mathrsfs}
\usepackage{hyperref}
\usepackage[toc, xindy]{glossaries}

\def\realnumbers{{\rm I\!R}}
\def\polynomials{{\rm I\!P}}

\newcommand{\rom}[1]{\uppercase\expandafter{\romannumeral #1\relax}}
\newcommand{\norm}[1]{\lVert#1\rVert}
\renewcommand{\arraystretch}{1.5}

\makenoidxglossaries

\newglossaryentry{framebuffer}
{
  name=Framebuffer,
  description={Digitale Rasterbild-Kopie des Monitorbildes}
}

\newglossaryentry{farbtiefe}
{
  name=Farbtiefe,
  description={Anzahl möglicher Farb- bzw. Grauwerte, charakterisiert durch die Bit-Anzahl}
}

\newglossaryentry{dithering}
{
  name=Dithering,
  description={Absichtlich eingesetztes Rauschen zur Erzeugung eines im Farbraum sonst nicht verfügbaren Farbeindrucks (z.B Schwarz-Weiß-Dithering für Graustufen)}
}

\newglossaryentry{color banding}
{
  name=Color Banding,
  description={Bandeffekte aufgrund von Problemen bei der Farbdarstellung ohne Dithering, oftmals aufgrund zu geringer Farbtiefe}
}

\newglossaryentry{gamma-korrektur}
{
  name=Gamma-Korrektur,
  description={Anpassen eines digitalen Farbwertes, sodass der Farbeindruck auf dem Display für das menschliche Auge bei steigendem Pixelwert linear wirkt}
}

\newglossaryentry{transparenz}
{
  name=Transparenz,
  description={Physikalische Eigenschaft der teilweisen oder vollständigen Lichtdurchlässigkeit}
}

\newglossaryentry{opazitaet}
{
  name=Opazität,
  description={Gegenteil von Transparenz}
}

\newglossaryentry{metamerismus}
{
  name=Metamerismus,
  description={Phänomen, bei dem unterschiedliche Farbspektren dieselbe Farbe ergeben}
}

\newglossaryentry{farbmodell}
{
  name=Farbmodell,
  description={Modell zur Beschreibung von Farben durch Wertetupel, z.B RGB-Farbmodell}
}

\newglossaryentry{farbraum}
{
  name=Farbraum,
  description={Raum aller möglichen Farben eines Farbmodells}
}

\newglossaryentry{tristimuluswerte}
{
  name=Tristimuluswerte,
  description={Wertetupel eines Farbmodells}
}

% \newglossaryentry{}
% {
%   name=,
%   description={}
% }

\begin{document}
	\pagenumbering{Roman}
	{\let\newpage\relax\maketitle}
	\tableofcontents
	\newpage
	\pagenumbering{arabic}
	\setcounter{page}{1}

	\section{Bilder, Farbe, Perzeption}
	\subsection{Rasterbilder}
	Rasterbilder als hauptsächlicher Fokus (statt z.B Vektorgrafik)
	\begin{itemize}
		\item Bild ist rechteckiges Pixelgitter endlicher Pixelzahl
		\item Digitaler \Gls{framebuffer} als Kopie des Monitorbildes
		\item Pixel haben bestimmte \Gls{farbtiefe}
		\begin{itemize}
			\item Schwarz/Weiß (1 Bit/Pixel), Graustufen (8 Bit/Pixel), True Color (24 Bit/Pixel)
			\item Farbe mit Farbtabelle (\glqq lookup table, LUT\grqq) und High Dynamic Range (3 x 32 Bit Floating Point/Pixel)
			\item Farbe wird i.d.R. durch RGB-Wert (Rot-Grün-Blau) charakterisiert
		\end{itemize}
	\end{itemize}

	\subsection{Bildtransfer zum Display}
	Die Umsetzung eines digitalen Bildes zu einem sichtbaren Bild auf einem Display kann als Transferfunktion f betrachtet werden. Displaycharakteristika beeinflussen diese Darstellung, darunter:
	\begin{itemize}
		\item Maximale Displayhelligkeit $I_{max}$
		\item Minimale Displayhelligkeit $I_{min}$ (Helligkeit eines schwarzen Pixels)
		\item Reflektiertes Umgebungslicht k
	\end{itemize}
	womit sich der erreichbare Kontrast $R_d = \frac{I_{max} + k}{I_{min} + k}$ ergibt. Man benötigt eine Transferfunktion, bei welcher der Unterschied zwischen aufeinanderfolgenden Pixelwerten nicht bemerkbar (unter 2\%) ist, um \Gls{color banding} zu vermeiden.

	\subsection{Gamma-Korrektur}
	\textbf{Problem}: Digitale Pixelwerte befinden sich in einem linearen Raum, d.h. doppelter Wert impliziert doppelte Helligkeit. Displays verhalten sich nicht linear, weshalb die Pixelwerte auf das Display angepasst werden müssen.
	\\\\
	Dieses Displayverhalten wird durch einen Gamma-Wert ($\gamma$) charakterisiert, die entsprechende Korrektur heißt \Gls{gamma-korrektur}. Die Intensität wächst nicht proportional zum Farbwert n bei N Schritten, sondern proportional zu $(\frac{n}{N})^{\gamma}$. Zur \Gls{gamma-korrektur} wird entsprechend jeder Farbkanal-Wert mit $\frac{1}{\gamma}$ potenziert.

	\subsection{Alpha-Kanal}
	Zusätzlich zu RGB-Farbwerten wird oftmals auch ein Alpha-Kanal gespeichert, dessen Inhalt die \Gls{opazitaet} des Pixels ist.

	\subsection{Licht}
	\begin{itemize}
		\item Licht ist elektromagnetische Strahlung mit verschiedenen Charakteristika von Strahlen, Wellen und Teilchen
		\item Licht besitzt eine Wellenlänge $\lambda$, die u.a. eine Spektralfarbe repräsentiert (Lichtfrequenz $v = \frac{c}{\lambda}$)
		\item Sichtbares Licht: 380nm < $\lambda$ < 700nm
		\item Licht verschiedener Wellenlängen und verschiedener Intensitäten setzen weitere Farben zusammen
		\item \Gls{metamerismus}: Unterschiedliche Spektren ergeben dieselbe Farbe
	\end{itemize}

	\subsection{Farbräume}
	\textbf{Allgemein}:\\
	Ein \textbf{\Gls{farbmodell}} ermöglicht die Beschreibung von Farben durch Wertetupel, jedes \Gls{farbmodell} erzeugt einen \textbf{\Gls{farbraum}} aller möglichen Farben und diese können durch entsprechende \textbf{\Gls{tristimuluswerte}} beschrieben werden.

	\begin{itemize}
		\item Jeder Farbeindruck kann mit 3 Grundgrößen beschrieben werden (Graßmannsche Gesetze)
		\item Additive Farbmischung (vgl. RGB-\Gls{farbmodell})
		\begin{itemize}
			\item Summe der \Gls{tristimuluswerte} Rot, Grün und Blau ergibt finalen Farbwert
		\end{itemize}
		\item Subtraktive Farbmischung (vgl. CMY(K)-\Gls{farbmodell})
		\begin{itemize}
			\item Statt RGB: Cyan, Magenta, Yellow (und in der Praxis Schwarz als 4. Key-Color, da CMY typischerweise kein Schwarz ergibt)
			\item Differenz der Farbwerte ergibt finalen Farbwert
		\end{itemize}
		\item Weder additiv noch subtraktiv (vgl. HSV-\Gls{farbmodell})
		\begin{itemize}
			\item Charakterisierung der finalen Farbe durch Farbton (Hue), Sättigung (Saturation) und Helligkeit (Value)
		\end{itemize}
	\end{itemize}

	\subsection{Farbraumkonversion}
	\textbf{Ziel}: \Gls{farbraum} zur standardisierten Konversion zwischen Farbräumen.\\\\
	\textbf{Color Matching Funktionen}
		\begin{itemize}
			\item Reproduktion von Spektralfarben durch RGB-Primärfarben
			\item RGB ist kein perfekter \Gls{farbraum}, manche Spektralfarben sind nicht realisierbar!
		\end{itemize}		
	\textbf{XYZ-\Gls{farbraum}}
		\begin{itemize}
			\item Beschreibt alle wahrnehmbaren Farben (\glqq Gamut der menschlichen Wahrnehmung\grqq) mit rein positiven Color Matching Funktionen
			\item Primärfarben sind imaginär, übersaturiert und nicht physikalisch realisierbar
			\item Lineare Abbildung $XYZ \Leftrightarrow RGB$, Transformationsmatrix $M$
			\item Problem: $M^{-1}$ enthält negative Werte, XYZ kann auf negative, nicht darstellbare RGB Werte abbilden
		\end{itemize}
	\textbf{xyY-\Gls{farbraum}}
		\begin{itemize}
			\item Beobachtung: $kX, kY, kZ (k > 0)$ repräsentiert dieselbe Farbe mit unterschiedlicher Intensität
			\item Idee: Normalisierung auf der $X + Y + Z = 1$ Ebene, daraufhin Projektion auf die XY-Ebene (z weglassen)
			\item Ergebnis: Weiterhin alle Farbtöne und -sättigungen in XY erhalten, neuer xyY \Gls{farbraum} mit Helligkeit Y und Farbe/Chromatizität xy
		\end{itemize}

	\section{Raytracing}
	\section{Transformationen}
	\section{Texturen}
	\section{Räumliche Datenstrukturen}
	\section{Rasterisierung und Projektion}
	\section{OpenGL und Grafik-Hardware}

	\printnoidxglossary
\end{document}
