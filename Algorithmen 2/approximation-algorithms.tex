\section{Approximation Algorithms}%
\label{aa:sec:approximation_algorithms}

\begin{itemize}
	\item \textbf{Ziel}: Näherungsweises Lösen von NP-schweren Problemen (fast alle interessanten Optimierungsprobleme sind NP-schwer)
	\item \textbf{Lösungsansätze}: Spezialisierung des Problems, Heuristiken, Approximationsalgorithmen
	\item \textbf{Approximationsfaktor}: Obere Schranke $\rho$ für das Verhältnis von gefundener Lösung zu optimaler Lösung bei beliebiger Eingabe (\quotestyle{Güte})
	\item \textbf{APX (\quotestyle{approximable})}: Approximationsalgorithmus, der in \textbf{polynomieller Zeit} in der Eingabegröße eine Approximation mit \textbf{konstantem Approximationsfaktor} berechnet
	\item \textbf{PTAS (\quotestyle{polynomial time approximation scheme})}: Approximationsalgorithmus, der eine $(1 \pm \epsilon)$-Approximation in \textbf{polynomieller Zeit} in der Eingabegröße berechnet
	\item \textbf{FPTAS (\quotestyle{fully PTAS})}: PTAS Algorithmus, welcher zusätzlich \textbf{polynomiell} in $\frac{1}{\epsilon}$ ist
	\item Jeder \textbf{FPTAS} Algorithmus ist ein \textbf{PTAS} Algorithmus ist ein \textbf{APX} Algorithmus!
\end{itemize}

\subsection{List Scheduling}%
\label{aa:sub:list_scheduling}

\begin{itemize}
	\item Teile Problem in \textbf{unabhängige, gewichtete Jobs} und führe diese auf parallelen Maschinen aus
	\item \textbf{List Scheduling}: Liste an Jobs, Maschine wählt einen aus und streicht diesen von der Liste, berechnet das Ergebnis und wiederholt dies, bis die Liste leer ist
\end{itemize}

\subsection{Nichtapproximierbarkeit}%
\label{aa:sub:nichtapproximierbarkeit}

\begin{itemize}
	\item \textbf{Es ist NP-schwer das Traveling-Salesman-Problem (TSP) innerhalb irgendeines Faktors a zu approximieren}
	\item Beweis: Transformation von Hamilton-Kreis-Problem zu a-Approximation von TSP
\end{itemize}

\subsection{Pseudopolynomielle Algorithmen}%
\label{aa:sub:pseudopolynomielle_algorithmen}

\begin{itemize}
	\item Ein pseudopolynomieller Algorithmus ist \textbf{polynomiell} bei \textbf{unär kodierten Eingaben}
\end{itemize}