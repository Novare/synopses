\section{Zero-Knowledge}%
\label{zn:sec:zero-knowledge}

\subsection{Definition}%
\label{zn:sub:definition}
\textbf{Ununterscheidbarkeit}\\
Die Verteilungen \(X\) und \(Y\) sind ununterscheidbar, falls für alle PPT-Angreifer
\(\mathcal{A}\) die Differenz
\(\Pr_{x \leftarrow X}[\mathcal{A}(1^k, x) = 1] - \Pr_{y \leftarrow Y}[\mathcal{A}(1^k, y) = 1]\)
vernachlässigbar (in \(k\)) ist.\\
\textbf{Zero-Knowledge}\\
Das PK-Identifikationsprotokoll \((\mathbf{Gen}, \mathbf{P}, \mathbf{V})\) ist Zero-Knowledge, falls für alle
PPT-Angreife \(\mathcal{A}\) ein PPT-Simulator \(\mathcal{S}\) exisitiert, sodass die Verteilungen\\
\begin{center}
  \((\mathit{pk}, \langle\mathcal{P}(\mathit{sk}),\mathcal{A}(1^k, \mathit{pk})\rangle)\)
  und
  \((\mathit{pk}, \mathcal{S}(1^k, \mathit{pk}))\)
\end{center}
ununterscheidbar sind (wobei \(\mathit{pk}, \mathit{sk}\) von \(\mathbf{Gen}(1^k)\) erzeugt wurde).

\subsection{Commitments}%
\label{zn:sub:commitments}
\textbf{Eigenschaften}
\begin{itemize}
\item \textit{hiding} - \(\textsc{Com}(M; R)\) verrät zunächst keinerlei Informationen über M
\item \textit{binding} - \(\textsc{Com}(M; R)\) legt den Ersteller des Commitment auf \(M\) fest, d.h.
  der Ersteller der Nachricht kann später nicht glaubhaft behaupten, dass \(M' \neq M\) zur Erstellung
  Commitment
\end{itemize}