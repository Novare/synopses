\section{Grundlagen}%
\label{gl:sec:grundlagen}

\subsection{Kerckhoffs' Prinzip}%
\label{gl:sub:kerckhoffs-prinzip}
Die Sicherheit eines Verfahrens beruht auf der \textbf{Geheimhaltung des Schlüssels}
anstatt auf der \textbf{Geheimhaltung des Verschlüsselsungsverfahrens}.

\subsection{EUF-CMA}%
\label{gl:sub:euf-cma}
Ein Verfahren ist EUF-CMA sicher, wenn für alle PPT-Angreifer \(\mathcal{A}\) die Wahrscheinlichkeit,
dass \(\mathcal{A}\) im EUF-CMA Spiel gewinnt, vernachlässigbar (im Sicherheitsparameter) ist.

\subsection{IND-CPA}%
\label{gl:sub:ind-cpa}
Ein symmetrisches Verfahren \((\mathbf{Enc}, \mathbf{Dec})\) heißt \textbf{IND-CPA-sicher}, wenn der Vorteil des
PPT-Algorithmus \(\mathcal{A}\) gegenüber dem Raten einer Lösung, also
\(\Pr\left[\mathcal{A}\ \text{gewinnt}\right] - \frac{1}{2}\), für alle
PPT-Algorithmen \(\mathcal{A}\) vernachlässigbar im Sicherheitsparameter \textit{k} ist.

\subsection{Vernachlässigbar}%
\label{gl:sub:vernachlaessigbar}
Eine vernachlässigbare Funktion „verschwindet“ (d.h. geht gegen Null) also schneller als der
Kehrwert jedes Polynoms.