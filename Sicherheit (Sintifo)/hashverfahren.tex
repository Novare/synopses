\section{Hashverfahren}%
\label{hv:sec:hashverfahren}

\subsection{Kollisionsresistenz}%
\label{hv:sub:kollisionsresistenz}
\(H_k\) ist kollisionsresistent, wenn jeder PPT-Algorithmus mit höchstens vernachlässigbarer Wahrscheinlichkeit
eine Kollision, also zwei verschiedene Nachrichten \(M \neq M'\) für die \(H(M) = H(M')\) gilt, findet.\\
oder\\
\(H_k\) ist kollisionsresistent, wenn für alle PPT-Algorithmen \(\mathcal{A}\) gilt:
\[\Pr\Big[(M, M') \leftarrow \mathcal{A}(1^k) \Big| M \neq M' \and H_k(M) = H_k(M')\Big]\]
ist vernachlässigbar (in k).

\subsection{Key-Strenghtening}%
\label{hv:sub:key-strengthening}
\begin{itemize}
\item Mehrfachanwendung (z.B. 1000 Mal)
\item Suche kleinste Zahl i, sodass die ersten z.B. 20 Bits von \(H(m,i)\) gleich 0 sind.
\end{itemize}

\subsection{Birthday-Angriff}%
\label{hv:sub:birthday-angriff}
Wahrscheinlichkeit für Kollision steigt mit Quadrat.
Vorgehen:
\begin{itemize}
\item Wähle \(2^{\frac{k}{2}})\) zufällige Urbilder und berechne Hashwert
\item Sortiere nach Hashwert
\item Suche nach identischen Hashwerten
\end{itemize}
Laufzeit: \(O(k \cdot 2^{\frac{k}{2}})\) (Grund für k: Sortieren)\\
Speicherbedarf: \(O(k \cdot 2^{\frac{k}{2}})\)

\subsection{HMAC}%
\label{hv:sub:hmac}
kurz für \textbf{Keyed-Hash Message Authentication Code} \\
\[\mathit{HMAC}_{K}(M) = H((K \oplus \mathit{opad}) || H((K \oplus \mathit{ipad}) || M))\]