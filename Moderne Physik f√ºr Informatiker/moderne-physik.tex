\documentclass[10pt,a4paper]{article}
\author{Jannik Koch}
\title{Moderne Physik für Informatiker}

\usepackage[utf8]{inputenc}
\usepackage[ngerman]{babel}
\usepackage{multicol}
\usepackage{amsmath}
\usepackage[a4paper, total={6in, 8in}]{geometry}
\usepackage{mathrsfs}
\usepackage{csquotes}
\usepackage{hyperref}
\usepackage{graphicx}
\usepackage{trfsigns}

\def\realnumbers{{\rm I\!R}}
\def\polynomials{{\rm I\!P}}

\newcommand{\rom}[1]{\uppercase\expandafter{\romannumeral #1\relax}}
\newcommand{\norm}[1]{\lVert#1\rVert}
\renewcommand{\arraystretch}{1.5}
\newcommand{\quotestyle}[1]{\enquote{#1}}
\newcommand{\itquote}[1]{\textit{\quotestyle{#1}}}

\begin{document}
	\pagenumbering{Roman}
	{\let\newpage\relax\maketitle}
	\tableofcontents
	\newpage
	\pagenumbering{arabic}
	\setcounter{page}{1}

	\section{Grundlagen}%
\label{grnd:sec:grundlagen}

\subsection{Notation}%
\label{grnd:sub:Notation}

\begin{itemize}
	\item \textbf{Vektoren}, z.B Position: $\vec{r}$
	\item \textbf{1. Ableitung}, z.B Geschwindigkeit: $\vec{v} = \dot{\vec{r}}$
	\item \textbf{2. Ableitung}, z.B Geschwindigkeit: $\vec{a} = \dot{\vec{v}} = \ddot{\vec{r}}$
	\item \textbf{3. Ableitung}: analog
\end{itemize}

\subsection{Grundbegriffe}%
\label{grnd:sub:grundbegriffe}

\begin{itemize}
	\item \textbf{Gleichförmige Bewegung}: Unbeschleunigt, mit konstanter Geschwindigkeit
	\item \textbf{Kraft}: Einwirkung, welche Körper verformt oder beschleunigt
	\item \textbf{Arbeit}: Energie, welche durch Kräfte auf Körper übertragen wird
\end{itemize}
	\section{Klassische Newtonsche Mechanik}%
\label{newt:sec:klassische_newtonsche_mechanik}

\subsection{Newtonsche Gesetze}%
\label{newt:sub:newtonsche_gesetze}

\begin{enumerate}
	\item \textbf{Galileisches Trägheitsgesetz}
	\begin{itemize}
		\item \itquote{Ein Körper verharrt im Zustand der Ruhe oder der gleichförmig geradlinigen Translation, sofern er nicht durch einwirkende Kräfte zur Änderung seines Zustandes gezwungen wird.}
		\item Körper sind \textbf{träge}, sie setzen einwirkenden Kräften einen Trägheitswiderstand entgegen
		\item \textbf{Impuls}: $\vec{p} = \vec{m}\vec{v}$
	\end{itemize}
	\item \textbf{Bewegungsgesetz}
	\begin{itemize}
		\item \itquote{Die Änderung der Bewegung eines Körpers ist proportional zu der auf ihn wirkenden Kraft und geschieht in die Richtung, in welche die Kraft weist.}
		\item \textbf{Also}: $\dot{\vec{p}} = \vec{F}$, wobei bei konstanter Masse: $\vec{F} = m\vec{a}$
	\end{itemize}
	\item \textbf{actio = reactio}
	\begin{itemize}
		\item \itquote{Übt ein Körper A auf einen anderen Körper B eine Kraft aus (actio), so wirkt von Körper B auf Körper A eine gleich große aber entgegen gerichtete Kraft (reactio).}
		\item Kräfte treten immer \textbf{paarweise} auf
	\end{itemize}
\end{enumerate}

\subsection{Kräfte}%
\label{newt:sub:kraefte}

\begin{itemize}
	\item \textbf{Gravitationskraft}:
	\begin{itemize}
		\item Wirkt zwischen zwei Massen $M_1, M_2$ und bewirkt deren Anziehung
		\item Unendliche Reichweite, aber nimmt mit zunehmendem Abstand ab
		\item Gravitationskraft von $M_2$ zu $M_1$ in Richtung $r$ mit der Gravitationskonstante $\gamma$: $$\vec{F_2} = -\gamma\frac{m_1m_2}{r^2}\hat{r} = -F_1\ \text{mit}\ \hat{r} = \frac{\vec{r}}{|\vec{r}|}$$
	\end{itemize}
	\item \textbf{Coulombkraft}:
	\begin{itemize}
		\item Wirkt zwischen zwei Punktladungen oder kugelsymmetrisch verteilten elektrischen Ladungen $Q_1, Q_2$
		\item Anziehend oder abstoßend je nach Vorzeichen der Ladungen in Richtung der Verbindungsgeraden
		\item Coulombkraft in Richtung $r$ mit der elektrischen Feldkonstante $\epsilon_0$: $$\vec{F} = \frac{1}{4\pi\epsilon_0}\frac{Q_1Q_2}{r^2}\hat{r}$$
	\end{itemize}
	\newpage
	\item \textbf{Lorentzkraft}:
	\begin{itemize}
		\item Wirkt auf bewegte Ladung in einem elektrischen oder magnetischen Feld
		\item Magnetische Komponente wirkt senkrecht zur Bewegung der Ladung und zur Richtung des magnetischen Feldes
		\item Lorentzkraft einer Ladung $e$ in einem Feld mit Feldstärke $\vec{E}$ und magnetischer Flussdichte $\vec{B}$: $$\vec{F} = e(\vec{E} + \vec{v} \times \vec{B})$$
	\end{itemize}
	\item \textbf{Federkraft}:
	\begin{itemize}
		\item Wirkt auf eine Feder
		\item Federkraft einer Feder mit Federkonstante $\alpha$ bei einer Auslenkung von $x$: $$F = \alpha|x| < 0$$
	\end{itemize}
\end{itemize}

\subsubsection{Konservative Kräfte}%
\label{newt:ssub:konservative_kraefte}

\begin{itemize}
	\item Verrichten längs des geschlossenen Weges \textbf{keine Arbeit}
	\item \textbf{Beispiele}: Gravitationskraft, Coulombkraft, Federkraft
\end{itemize}

\subsection{Inertialsysteme}%
\label{newt:sub:inertialsysteme}

\begin{itemize}
	\item \textbf{Bezugssystem (Perspektive eines Beobachters)}, in dem sich kräftefreie Körper geradlinig und gleichförmig bewegen
	\item Unendlich viele Inertialsysteme möglich, welche sich geradlinig und gleichförmig gegeneinander bewegen; hängen über \textbf{Galilei-Transformation} zusammen
	\item Unter einer Galilei-Transformation sind die Gesetze der Newtonschen Mechanik invariant (\textbf{Galilei-Invarianz})
	\item \textbf{Relativitätsprinzip der Newtonschen Mechanik}: Kein absolutes Bezugssystem vorhanden, Geschwindigkeit der Bewegung zweier Inertialsysteme somit nicht absolut messbar
	\item \textbf{Beschleunigte Bezugssysteme} (z.B rotierende Systeme) sind \textbf{keine} Inertialsysteme
\end{itemize}
	\section{Lagrangeformalismus}%
\label{lag:sec:lagrangeformalismus}

\begin{itemize}
	\item Umformulierung der Newtonschen Mechanik
	\item \textbf{Lagrangefunktion}: Skalare Funktion, umfasst gesamte klassische Mechanik
	\item \textbf{Vorteil}: Einfache Behandlung von Problemen mit Zwangsbedingungen (z.B Bewegung von verbundenen Kugeln)
\end{itemize}

\subsection{Lagrangegleichungen 1. Art}%
\label{lag:sub:lagrangegleichungen_1_art}

\begin{itemize}
	\item \textbf{Gegeben}: System aus $N$ Massenpunkten mit Massen $m_i, i = 1..N$
	\item \textbf{Freiheitsgrade}: Massenpunkte können sich in z.B $3$ Dimensionen bewegen $\Rightarrow$ $3N$ Freiheitsgrade
	\item \textbf{Holonome Zwangsbedingungen}: $N_Z$ Zwangsbedingungen von $N$ Teilchen mit Koordinaten $\vec{r_1}..\vec{r_N}$ sind \textbf{holonom}, wenn sie sich folgendermaßen schreiben lassen: $$A_\mu(\vec{r_1}..\vec{r_N}, t) = 0,\ \mu = 1..N_Z$$ ansonsten sind sie \textbf{nichtholonom} (z.B Ungleichungen)
	\item \textbf{Skleronome \& Rheonome Zwangsbedingungen}: Zwangsbedingungen mit expliziter Zeitabhängigkeit sind \textbf{rheonom}, alle anderen \textbf{skleronom}
\end{itemize}

\subsection{Lagrangegleichungen 2. Art}%
\label{lag:sub:lagrangegleichungen_2_art}

\subsection{Erhaltungsgrößen}%
\label{lag:sub:erhaltungsgroessen}

	\section{Variationsprinzipien}%
\label{var:sec:variationsprinzipien}

\begin{itemize}
	\item \textbf{Variationsrechnung}: Lösung von Problemen, bei denen der \textbf{Extremwert einer Größe} gefunden werden soll, die als \textbf{Integral über einen Funktionalausdruck} darzustellen ist
	\item \textbf{Variationsprinzip}: Allgemeine Methode der Variationsrechnung
	\item \textbf{Funktional}: Funktion von Funktionen, bildet \textbf{Funktionen auf Zahlen} ab
	\item \textbf{Euler-Lagrange-Gleichung}:
	\begin{itemize}
		\item \textbf{Problem}: Finde Funktion $y(x)$ mit Randwerten $y(x_1) = y_1$ und $y(x_2) = y_2$, welche das Funktional $J = J[y] = \int^{x_2}_{x_1} dx F(y, y', x)$ minimiert
		\item Differentialgleichung für \textbf{eine Funktion}:
		$$
			\frac{\partial F}{\partial y} - \frac{d}{dx}(\frac{\partial F}{\partial y'}) = 0
		$$
		\item Differentialgleichung für \textbf{mehrere Funktionen} $y_i(x)$,\\d.h. $F(y_1(x), \dots, y_n(x), y'_1(x), \dots, y'_n(x), x)$:
		$$
			\frac{\partial F}{\partial y_i} - \frac{d}{dx}(\frac{\partial F}{\partial y_i'}) = 0,\ i = 1, \dots, n
		$$
	\end{itemize}
	\item \textbf{Hamiltonsches Prinzip}:
	\begin{itemize}
		\item \textbf{Variationsprinzip}, dessen Euler-Lagrange-Gleichungen die \textbf{Lagrangegleichungen} der Mechanik sind
		\item \textbf{Stationaritätsprinzip}: $S = \int^{t_2}_{t_1} dt L(q, \dot{q}, t) = stationaer$
		\item \textbf{S} bezeichnet man auch als \textbf{Wirkungsfunktional} oder kurz \textbf{Wirkung}
		\item \textbf{Hamiltonsches Prinzip}: $\delta S[q] = 0$
	\end{itemize}
\end{itemize}
	\section{Hamiltonformalismus}%
\label{hform:sec:hamiltonformalismus}

\begin{itemize}
	\item \textbf{Alternative} zu Lagrangeformalismus
	\item Keine praktischen Vorteile zu Lagrange, aber wichtig für Relation \textbf{Mechanik-Quantenmechanik}
	\item Betont \textbf{verallgemeinerte Impulse} gleichwertig zu \textbf{verallgemeinerten Koordinaten}
\end{itemize}

\section{Kanonische Gleichungen}
\label{hform:sec:kanonische_gleichungen}

\begin{itemize}
	\item \textbf{Verallgemeinerte Impulse}: $p_i = \frac{\partial L}{\partial \dot{q}_i}, i = 1, \dots, f$
	\item \textbf{Hamiltonfunktion}:
	$$
		H(q, p, t) = \sum^f_{i=1}\dot{q}_ip_i - L = \sum_i\dot{q}_i(q, p, t)p_i - L(q, \dot{q}(q, p, t), t)
	$$
	\item \textbf{Kanonische/Hamiltonsche Bewegungsgleichungen}:
	$$
		\dot{q}_i = \frac{\partial H}{\partial p_i}, \dot{p}_i = -\frac{\partial H}{\partial q_i}\ \text{mit}\ i = 1, 2, \dots, f
	$$
	\item \textbf{Vorgehen zur Problemlösung mit Hamiltonformalismus}:
	\begin{enumerate}
		\item Finde generalisierte Koordinaten: $\vec{q} = (q_1, \dots, q_f)$
		\item Finde Transformationen: $\vec{r}_k = \vec{r}_k(q_1, \dots, q_f, t), \dot{\vec{r}}_k = \dot{\vec{r}}_k(\vec{q}, \dot{\vec{q}}, t)$
		\item Drücke kinetische und potentielle Energie in den Teilchenkoordinaten durch verallgemeinerte Koordinaten aus
		\item Bestimme generalisierte Impulse: $p_j = \frac{\partial L}{\partial \dot{q}_j} \Rightarrow p_j = p_j(\vec{q}, \dot{\vec{q}}, t)$
		\item Löse für $\dot{q_j}$: $\dot{q_j} = \dot{q_j}(\vec{q}, \dot{\vec{q}}, t)$
		\item Ersetze dies in der Lagrangefunktion: $L(\vec{q}, \dot{\vec{q_j}}(\vec{q}, \vec{p}, t), t) = \widetilde{L}(\vec{q}, \vec{p}, t)$
		\item Bestimme H per Legendre-Transformation: $H(\vec{q}, \vec{p}, t) = \sum^f_{j=1}p_j\dot{q}_j(\vec{q}, \vec{p}, t) - \widetilde{L}(\vec{q}, \vec{p}, t)$
		\item Stelle die kanonische Bewegungsgleichungen auf und löse sie
	\end{enumerate}
\end{itemize}

\section{Poissonklammer}
\label{hform:sec:poissonklammer}

\begin{itemize}
	\item \textbf{Alternative Form} der Bewegungsgleichungen
	\item \textbf{Definition}:
	$$
		\{F, K\} = \sum^f_{i=1}(\frac{\partial F}{\partial q_i}\frac{\partial K}{\partial p_i} - \frac{\partial F}{\partial p_i}\frac{\partial K}{\partial q_i})
	$$
	\newpage
	\item \textbf{Eigenschaften}:
	\begin{enumerate}
		\item $\{F, K\} = -\{K, F\}, \{F, F\} = 0$
		\item $\frac{\partial F}{\partial p_j} = -\{F, q_j\}, \frac{\partial F}{\partial q_j} = -\{F, p_j\}$
		\item $\{q_i, q_j\} = 0, \{p_i, p_j\} = 0, \{p_i, q_j\} = -\delta_{ij}$
		\item $\{F, c\} = 0,\ c = const.$
		\item $\{F_1 + F_2, K\} = \{F_1, K\} + \{F_2, K\}$
		\item $\frac{\partial}{\partial t}\{F, K\} = \{\frac{\partial F}{\partial t}, K\} + \{F, \frac{\partial K}{\partial t}\}$
		\item $\{F, \{K, J\}\} = \{K, \{J, F\}\} + \{J, \{F, K\}\} = 0\ (\mathbf{Jacobi-Identitaet})$
		\item $\frac{dF}{dt} = \{F, H\} + \frac{\partial F}{\partial t}$
	\end{enumerate}
\end{itemize}

\section{Hamiltonsches Prinzip}
\label{hform:sec:hamiltonsches_prinzip}

\begin{itemize}
	\item \textbf{Aussage}: Die Wirkung S für die tatsächliche Bewegung ist \textbf{stationär}, d.h.
	$$
		\delta S[q] = \delta \int^{t_2}_{t_1}dt\ L = \delta \int^{t_2}_{t_1}dt\ (\sum^f_{i=1}p_i\dot{q}_i - H(q, p, t)) = 0
	$$
	\item \textbf{Hamiltonsches Prinzip}:
	$$
		\delta S = \delta S[q, p] = 0
	$$
\end{itemize}

\section{Zustand eines Systems}
\label{hform:sec:zustand_eines_systems}

\begin{itemize}
	\item Unterschiedliche \textbf{Räume} zur Beschreibung eines Systems
	\item \textbf{Konfigurationsraum}:
	\begin{itemize}
		\item f-dimensionaler Raum der möglichen \textbf{generalisierten Koordinaten}, keine zeitlichen Informationen
	\end{itemize}
	\item \textbf{Ereignisraum}:
	\begin{itemize}
		\item (f+1)-dimensionaler Raum, Konfigurationsraum \textbf{mit Zeit}, keine Impulsinformationen
	\end{itemize}
	\item \textbf{Phasenraum}:
	\begin{itemize}
		\item 2f-dimensional, verallgemeinerte Koordinaten \textbf{und Impulse}, beschreibt alle möglichen Zustände, keine zeitlichen Informationen
	\end{itemize}
	\item \textbf{Zustandsraum}:
	\begin{itemize}
		\item (2f+1)-dimensional, Phasenraum \textbf{mit Zeit}, Darstellungsraum mit größter Information
		\item Konfigurations-, Ereignis- und Phasenraum sind \textbf{Projektionen} des Zustandsraumes
	\end{itemize}
	\item \textbf{Zustand} $\mathbf{\psi}$:
	\begin{itemize}
		\item \textbf{Gesamtheit} aller Informationen, die zur vollständigen Beschreibung der momentanten Eigenschaften des Systems erforderlich sind
		\item Punkt im \textbf{Phasenraum}, Bewegung wird als \textbf{Differentialgleichung 1. Ordnung} beschrieben: $\dot{\psi}(t) = \widetilde{f}(\psi(t))$
	\end{itemize}
\end{itemize}
	\section{Spezielle Relativitätstheorie}%
\label{srel:sec:spezielle_relativitaetstheorie}
	\section{Quantenmechanik}%
\label{quant:sec:quantenmechanik}

\subsection{Historische Erkenntnisse}
\label{quant:sub:historische_erkenntnisse}

\begin{itemize}
	\item \textbf{Hohlraumstrahlung}:
	\begin{itemize}
		\item Elektromagnetische Strahlung im abgeschlossenen Hohlraum im thermischen Gleichgewicht (keine Temperaturänderung durch die Wände)
		\item \textbf{Rayleigh-Jeans-Gesetz} verbindet Lichtwellenlänge und spezifische Ausstrahlung eines Schwarzen Körpers
		\item \textbf{Ultraviolett-Katastrophe}: Rayleigh-Jeans-Gesetz liefert bei kleinen Wellenlängen viel zu große Werte, Versagen der klassischen Physik
	\end{itemize}
	\item \textbf{Welle-Teilchen-Dualismus}:
	\begin{itemize}
		\item Ob Licht sich wie eine Welle oder ein Teilchen verhält hängt vom Experiment ab
		\item \textbf{Licht als Welle}: \textbf{Interferenz} (Wellenüberlagerung)
		\item \textbf{Licht als Teilchen}: \textbf{Photoelektrischer Effekt} (Herausschlagen von Elektronen)
	\end{itemize}
	\item \textbf{Atomphysik}:
	\begin{itemize}
		\item Im Rutherfordschen Atommodell müssten Elektronen ständig Energie abstrahlen und in den Kern stürzen
		\item Emissionssppektrum müsste \textbf{kontinuierlich} sein aufgrund \textbf{kontinuierlich variierender Umlaufbahn}
		\item \textbf{Stattdessen}: Diskrete Emissionslinien, es folgte die \textbf{Quantenhypothese für Elektronenbahnen}
		\item \textbf{Quantenhypothese}: Energiemenge die Strahlung und Materie austauschen können ist \textbf{nicht beliebig}
	\end{itemize}
	\item \textbf{Teilchenwellen}: Welle-Teilchen-Dualismus gilt auch für \textbf{konventionelle Teilchen} (Elektronen)
\end{itemize}

\subsection{Schrödinger Gleichung}
\label{quant:sub:schroedinger_gleichung}

\begin{itemize}
	\item Beschreibt in Form einer partiellen Differentialgleichung die \textbf{zeitliche Veränderung des quantenmechanischen Zustands} eines \textbf{nichtrelativistischen} Systems
	\item \textbf{Zeitabhängige Schrödingergleichung}:
	$$
		ih\frac{\partial}{\partial t}\psi(\vec{x},t) = H\psi(\vec{x}, t) = \frac{-h^2}{2m}\Delta\psi(\vec{x}, t)
	$$
	\item \textbf{Zeitunabhängige Schrödingergleichung}:
	$$
		H\phi(\vec{r}) = E\phi(\vec{r})
	$$
\end{itemize}
\end{document}
