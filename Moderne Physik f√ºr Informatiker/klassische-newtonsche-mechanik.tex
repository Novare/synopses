\section{Klassische Newtonsche Mechanik}%
\label{newt:sec:klassische_newtonsche_mechanik}

\subsection{Newtonsche Gesetze}%
\label{newt:sub:newtonsche_gesetze}

\begin{enumerate}
	\item \textbf{Galileisches Trägheitsgesetz}
	\begin{itemize}
		\item \itquote{Ein Körper verharrt im Zustand der Ruhe oder der gleichförmig geradlinigen Translation, sofern er nicht durch einwirkende Kräfte zur Änderung seines Zustandes gezwungen wird.}
		\item Körper sind \textbf{träge}, sie setzen einwirkenden Kräften einen Trägheitswiderstand entgegen
		\item \textbf{Impuls}: $\vec{p} = m\vec{v}$
	\end{itemize}
	\item \textbf{Bewegungsgesetz}
	\begin{itemize}
		\item \itquote{Die Änderung der Bewegung eines Körpers ist proportional zu der auf ihn wirkenden Kraft und geschieht in die Richtung, in welche die Kraft weist.}
		\item \textbf{Also}: $\dot{\vec{p}} = \vec{F}$, wobei bei konstanter Masse: $\vec{F} = m\vec{a}$
	\end{itemize}
	\item \textbf{actio = reactio}
	\begin{itemize}
		\item \itquote{Übt ein Körper A auf einen anderen Körper B eine Kraft aus (actio), so wirkt von Körper B auf Körper A eine gleich große aber entgegen gerichtete Kraft (reactio).}
		\item Kräfte treten immer \textbf{paarweise} auf
	\end{itemize}
\end{enumerate}

\subsection{Kräfte}%
\label{newt:sub:kraefte}

\begin{itemize}
	\item \textbf{Konservative Kräfte}:
	\begin{itemize}
		\item Verrichten längs des geschlossenen Weges \textbf{keine Arbeit}
		\item \textbf{Beispiele}: Gravitationskraft, Coulombkraft, Federkraft
	\end{itemize}
	\item \textbf{Gravitationskraft}:
	\begin{itemize}
		\item Wirkt zwischen zwei Massen $M_1, M_2$ und bewirkt deren Anziehung
		\item Unendliche Reichweite, aber nimmt mit zunehmendem Abstand ab
		\item Gravitationskraft von $M_2$ zu $M_1$ in Richtung $r$ mit der Gravitationskonstante $\gamma$: $$\vec{F_2} = -\gamma\frac{m_1m_2}{r^2}\hat{r} = -F_1\ \text{mit}\ \hat{r} = \frac{\vec{r}}{|\vec{r}|}$$
	\end{itemize}
	\item \textbf{Coulombkraft}:
	\begin{itemize}
		\item Wirkt zwischen zwei Punktladungen oder kugelsymmetrisch verteilten elektrischen Ladungen $Q_1, Q_2$
		\item Anziehend oder abstoßend je nach Vorzeichen der Ladungen in Richtung der Verbindungsgeraden
		\item Coulombkraft in Richtung $r$ mit der elektrischen Feldkonstante $\epsilon_0$: $$\vec{F} = \frac{1}{4\pi\epsilon_0}\frac{Q_1Q_2}{r^2}\hat{r}$$
	\end{itemize}
	\newpage
	\item \textbf{Lorentzkraft}:
	\begin{itemize}
		\item Wirkt auf bewegte Ladung in einem elektrischen oder magnetischen Feld
		\item Magnetische Komponente wirkt senkrecht zur Bewegung der Ladung und zur Richtung des magnetischen Feldes
		\item Lorentzkraft einer Ladung $e$ in einem Feld mit Feldstärke $\vec{E}$ und magnetischer Flussdichte $\vec{B}$: $$\vec{F} = e(\vec{E} + \vec{v} \times \vec{B})$$
	\end{itemize}
	\item \textbf{Federkraft}:
	\begin{itemize}
		\item Wirkt auf eine Feder
		\item Federkraft einer Feder mit Federkonstante $\alpha$ bei einer Auslenkung von $x$: $$F = \alpha|x| < 0$$
	\end{itemize}
\end{itemize}

\subsection{Inertialsysteme}%
\label{newt:sub:inertialsysteme}

\begin{itemize}
	\item \textbf{Bezugssystem (Perspektive eines Beobachters)}, in dem sich kräftefreie Körper geradlinig und gleichförmig bewegen
	\item Unendlich viele Inertialsysteme möglich, welche sich geradlinig und gleichförmig gegeneinander bewegen; hängen über \textbf{Galilei-Transformation} zusammen
	\item Unter einer Galilei-Transformation sind die Gesetze der Newtonschen Mechanik invariant (\textbf{Galilei-Invarianz})
	\item \textbf{Relativitätsprinzip der Newtonschen Mechanik}: Kein absolutes Bezugssystem vorhanden, Geschwindigkeit der Bewegung zweier Inertialsysteme somit nicht absolut messbar
	\item \textbf{Beschleunigte Bezugssysteme} (z.B rotierende Systeme) sind \textbf{keine} Inertialsysteme
\end{itemize}