\section{Quantenmechanik}%
\label{quant:sec:quantenmechanik}

\subsection{Historische Erkenntnisse}
\label{quant:sub:historische_erkenntnisse}

\begin{itemize}
	\item \textbf{Hohlraumstrahlung}:
	\begin{itemize}
		\item Elektromagnetische Strahlung im abgeschlossenen Hohlraum im thermischen Gleichgewicht (keine Temperaturänderung durch die Wände)
		\item \textbf{Rayleigh-Jeans-Gesetz} verbindet Lichtwellenlänge und spezifische Ausstrahlung eines Schwarzen Körpers
		\item \textbf{Ultraviolett-Katastrophe}: Rayleigh-Jeans-Gesetz liefert bei kleinen Wellenlängen viel zu große Werte, Versagen der klassischen Physik
	\end{itemize}
	\item \textbf{Welle-Teilchen-Dualismus}:
	\begin{itemize}
		\item Ob Licht sich wie eine Welle oder ein Teilchen verhält hängt vom Experiment ab
		\item \textbf{Licht als Welle}: \textbf{Interferenz} (Wellenüberlagerung)
		\item \textbf{Licht als Teilchen}: \textbf{Photoelektrischer Effekt} (Herausschlagen von Elektronen)
	\end{itemize}
	\item \textbf{Atomphysik}:
	\begin{itemize}
		\item Im Rutherfordschen Atommodell müssten Elektronen ständig Energie abstrahlen und in den Kern stürzen
		\item Emissionssppektrum müsste \textbf{kontinuierlich} sein aufgrund \textbf{kontinuierlich variierender Umlaufbahn}
		\item \textbf{Stattdessen}: Diskrete Emissionslinien, es folgte die \textbf{Quantenhypothese für Elektronenbahnen}
		\item \textbf{Quantenhypothese}: Energiemenge die Strahlung und Materie austauschen können ist \textbf{nicht beliebig}
	\end{itemize}
	\item \textbf{Teilchenwellen}: Welle-Teilchen-Dualismus gilt auch für \textbf{konventionelle Teilchen} (Elektronen)
	\item \textbf{Operator}:
	\begin{itemize}
		\item Vorschrift $A$, sodass für eine quadratintegrable Funktion $\phi(\vec{x}) \in L^2$ gilt: $A\phi(\vec{x}) = \theta(\vec{x}) \in L^2$
		\item \textbf{Linearität}: $A(c_1\phi_1 + c_2\phi_2) = c_1\theta_1 + c_2\theta_2$
		\item \textbf{Kommutator}: $[A, B] = AB - BA$
		\item \textbf{Skalarprodukt}: $(\phi, \theta) = \int d^3x\phi^*(x)\theta(x)$
		\item \textbf{Adjungierter Operator}: $A^\dagger$ adj. zu $A$, wenn $(A^\dagger\phi, \theta) = (\phi, A\theta) \forall \phi, \theta$
		\item \textbf{Selbstadjungiert/Hermitesch}: $A = A^\dagger$; dann gilt auch: $(AB)^\dagger = B^\dagger A^\dagger$
		\item \textbf{Erwartungswert im Zustand n}: $\langle A\rangle = \langle n | A | n \rangle$
	\end{itemize}
\end{itemize}

\subsection{Schrödinger Gleichung}
\label{quant:sub:schroedinger_gleichung}

\begin{itemize}
	\item Beschreibt in Form einer partiellen Differentialgleichung die \textbf{zeitliche Veränderung des quantenmechanischen Zustands} eines \textbf{nichtrelativistischen} Systems
	\item \textbf{Zeitabhängige Schrödingergleichung}:
	$$
		ih\frac{\partial}{\partial t}\psi(\vec{x},t) = H\psi(\vec{x}, t) = \frac{-h^2}{2m}\Delta\psi(\vec{x}, t)
	$$
	\item \textbf{Zeitunabhängige Schrödingergleichung}:
	$$
		H\phi(\vec{r}) = E\phi(\vec{r})
	$$
	\item $H$ ist der \textbf{Hamilton-Operator} und beschreibt die \textbf{Gesamtenergie des Systems}
\end{itemize}

\subsection{Eindimensionale Rechteckpotentiale}%
\label{quant:sub:eindimensionale_rechteckpotentiale}

\begin{itemize}
	\item \textbf{Ziel}: Untersuchung des Einflusses von Potentialen $V(x)$ (welche äußere Kräfte beschreiben) auf die Wellenfunktion $\phi(x)$
	\item \textbf{Allgemein}: $\frac{d^2}{dx^2}\phi(x) + \frac{2m}{\hbar^2}(E - V)\phi(x) = 0$
	\item \textbf{Fallunterscheidung}:
	\begin{itemize}
		\item $E > V$: $\phi(x) = Ae^{ikx} + Be^{-ikx}, A, B = const.$
		\item $E < V$: $\phi(x) = Ce^{\rho x} + De^{-\rho x}, C, D = const.$
		\item $E = V$: $\phi(x)$ lineare Funktion
	\end{itemize}
	\item \textbf{Anschlussbedingungen}: Funktionswerte der Wellenfunktion und ihrer Ableitung an Sprungstellen müssen gleich sein (nützlich zur Herleitung der Parameter)
\end{itemize}

\subsection{Dirac-Notation}%
\label{quant:sub:dirac_notation}

\begin{itemize}
	\item \textbf{Ziel}: Bequeme Notation eines Zustands ohne Bezug auf eine Ortsvariable; sei hierfür $H$ der \textbf{Hilbertraum}
	\item \textbf{Definitionen: }\textbf{Ket}: $|\theta\rangle \in H$, \textbf{Bra}: $\langle\phi| \in H$, \textbf{Skalarprodukt}: $\langle \phi | \theta \rangle$
	\item \textbf{Rechenregeln}:
	\begin{align*}
		\langle \phi | \theta \rangle^* &= \langle \theta | \phi \rangle\\
		\langle \alpha | \phi + \theta \rangle &= \langle \alpha | \phi \rangle + \langle \alpha | \theta \rangle\\
		\langle \phi | \lambda_1\theta_1 + \lambda_2\theta_2 \rangle &= \lambda_1\langle \phi | \theta_1\rangle + \lambda_2\langle \phi |\theta_2 \rangle\\
		\langle \lambda_1\phi_1 + \lambda_2\phi_2 | \theta \rangle &= \lambda_1^*\langle \phi_1 | \theta\rangle + \lambda_2^*\langle \phi_2 |\theta \rangle\\
	\end{align*}
\end{itemize}

\subsection{Harmonischer Oszillator}%
\label{quant:sub:harmonischer_oszillator}

\begin{itemize}
	\item \textbf{Parabelförmiges Potential}: $V(x) = \frac{1}{2}m\omega^2x^2$
	\item \textbf{Hamilton-Operator}: $H = \frac{p^2}{2m} + \frac{1}{2}m\omega^2x^2$
	\item \textbf{Algebraischer Lösungsansatz}: $\hat{H}|\phi_v\rangle = \epsilon_v|\psi_v\rangle$, wobei $\hat{H} = \frac{1}{\hbar\omega}H = \frac{1}{2}(\hat{x}^2 + \hat{p}^2)$ und $\epsilon = \frac{E}{\hbar\omega}$
	\item \textbf{Größen der algebraischen Lösung}:
	\begin{align*}
		\hat{x} &= \sqrt{\frac{m\omega}{\hbar}}x = \frac{1}{\sqrt{2}}(a + a^\dagger)\\
		\hat{p} &= \frac{1}{\sqrt{m\hbar\omega}}p = \frac{i}{\sqrt{2}}(a^\dagger - a)
	\end{align*}
	\item \textbf{Leitoperatoren}:
	\begin{align*}
		a &= \frac{1}{\sqrt{2}}(\hat{x} + i\hat{p})\\
		a^\dagger &= \frac{1}{\sqrt{2}}(\hat{x} - i\hat{p})
	\end{align*}
	\item $a$ ist der \textbf{Vernichtungsoperator}, der den \textbf{Energiewert} um $\hbar\omega$ erniedrigt: $a|n\rangle = \sqrt{n}|n - 1\rangle$
	\item $a^\dagger$ ist \textbf{Erzeugungsoperator}, der den \textbf{Energiewert} um $\hbar\omega$ erhöht: $a^\dagger|n\rangle = \sqrt{n + 1}|n + 1\rangle$
\end{itemize}