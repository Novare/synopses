\documentclass[10pt,a4paper]{article}
\author{Jannik Koch}
\title{Lineare Algebra 2 für Informatiker}

\usepackage[utf8]{inputenc}
\usepackage{amsmath}
\usepackage[a4paper, total={6in, 8in}]{geometry}

\def\realnumbers{{\rm I\!R}}
\def\naturalnumbers{{\rm I\!N}}

\newcommand{\rom}[1]{\uppercase\expandafter{\romannumeral #1\relax}}
\newcommand{\norm}[1]{\lVert#1\rVert}

\begin{document}
	\pagenumbering{Roman}
	{\let\newpage\relax\maketitle}
	\tableofcontents
	\newpage
	\pagenumbering{arabic}
	\setcounter{page}{1}

	\section{Jordan Normalform}
	Seien $A \in \realnumbers^{N\times N}$, $\lambda \in \realnumbers$.
	
	\subsection{Berechnung der Normalform}
	\begin{itemize}
		\item Berechnung des \textbf{charakteristischen Polynoms} $p_A(\lambda) = |A - \lambda I_N| = 0$
		\item \textbf{Algebraische Vielfachheit} eines Eigenwerts $r_\lambda$: Potenz des Eigenwerts in $p_A(\lambda)$
		\item \textbf{Geometrische Vielfachheit} eines Eigenwerts $d_\lambda$: Dimension des zugehörigen Eigenraums
		\item Hauptdiagonale der Jordan-Normalform $J_A$ entspricht Eigenwerten in beliebiger Reihenfolge, jeder davon jeweils so oft wie $r_\lambda$ vorgibt
		\item $d_\lambda$ gibt die Anzahl der Jordankästchen an
		\item Falls die Anzahl der Jordankästchen keine eindeutige Form impliziert
			\begin{enumerate}
				\item Hauptraum bilden (Eigenraum verallgemeinern)
				\item $q$ entspricht der Länge des längsten Jordankästchens
				\item Ansonsten: Anzahl der Kästchen der Länge k = $dim K_k(\lambda) - dim K_{k - 1}(\lambda¸)$
			\end{enumerate}
	\end{itemize}
	
	\subsection{Eigenräume verallgemeinern}
	Annahme: Dimension des Eigenraums ist kleiner als die algebraische Vielfachheit. Für die Jordan-Normalform wird ein Eigenraum zum Hauptraum erweitert.
	\begin{itemize}
		\item $K_k(\lambda) = ker(A - \lambda I_N)^k$, d.h. wir bilden $A - \lambda I_N$ jeweils $k$-mal auf sich selber ab und bilden den Kern für $K_k$; der Eigenraum selber entspricht somit $K_1(\lambda)$
		\item Sobald $K_q(\lambda) = K_{k + 1}(\lambda)$ für ein $q \in \naturalnumbers$ heißt $K_q(\lambda)$ Hauptraum, danach ergeben sich keine neuen Änderungen mehr
	\end{itemize}
	
	\subsection{Minimalpolynom bestimmen}
	\begin{enumerate}
		\item Charakteristisches Polynom bestimmen, Eigenwerte ablesen
		\item Haupträume der Eigenwerte bestimmen
		\item Exponenten des char. Polynoms anpassen für Minimalpolynom $m_A = (X - \lambda_1)^{q_1} ... (X - \lambda_n)^{q_n}$
	\end{enumerate}
	\textbf{Cayley-Hamilton}: $m_A(A) = 0$ und $m_A|p_A$

	\subsection{Jordan-Basis}
	Ziel: Finden einer Matrix S, sodass $J_A = S^{-1} * A * S$
	\begin{enumerate}
		\item Für jeden Eigenwert Vektoren $v \in ker(A - \lambda I_N)^q \setminus ker(A - \lambda I_N)^{q - 1}$ wählen;\\diese bilden die rechteste Spalte in S für das entsprechende Jordankästchen
		\item Das Kästchen wird nach links aufgefüllt, indem man v immer wieder abbildet:\\$(A - \lambda I_N)^{q - 1}v, (A - \lambda I_N)^{q - 2}v ... (A - \lambda I_N)v$
		\item Die Reihenfolge muss zur aufgestellten Normalform passen und ist nicht eindeutig!
	\end{enumerate}

	\subsection{Tricks}
		\begin{itemize}
			\item $dim(ker \phi)$ bzw. $dim(ker A)$ impliziert die Dimension des Eigenraumes zum Eigenwert 0
			\item $Spur(A)$ = Summe der Eigenwerte mit Vielfachheit (z.B Eigenwert mit algebraischer Vielfachheit 2 zählt doppelt)
		\end{itemize}
	
	\section{Skalarprodukte}
	Sei K ein Körper und V ein K-Vektorraum der Dimension N, $v, v_1, v_2, w, w_1, w_2 \in V$, $\alpha \in K$ sowie
	\begin{center}
		$s(\cdot,\cdot) = \langle\cdot,\cdot\rangle: V \times V \rightarrow \realnumbers^N, (v, w) \mapsto \langle v, w\rangle$
	\end{center}
	
	Sei weiter $B = \{b_1, ..., b_n\}$ eine Basis von V.
	
	\subsection{Skalarprodukte und Fundamentalmatrizen}
		Definition einer Fundamentalmatrix:
		\begin{center}
			$F(s) = \begin{bmatrix}
			\langle b_1, b_1\rangle & \langle b_1, b_2\rangle & ... & \langle b_1, b_n\rangle\\
			\vdots & \vdots & \ddots & \vdots \\
			\langle b_n, b_1\rangle & \langle b_n, b_2\rangle & ... & \langle b_n, b_n\rangle\\			
			\end{bmatrix}$
		\end{center}
		Voraussetzungen eines Skalarprodukts (dies gelte für beliebig gewählte Vektoren):
		\begin{itemize}
			\item Bilinearform
			\begin{enumerate}
				\item $\langle \alpha v_1 + v_2, w\rangle = \alpha\langle  v_1, w\rangle + \langle v_2, w\rangle$
				\item $\langle v, \alpha w_1 + w_2\rangle = \alpha\langle  v, w_1\rangle + \langle v, w_2\rangle$
			\end{enumerate}
			\item Symmetrisch
			\begin{itemize}
				\item Formal: $\langle v, w\rangle = \langle w, v\rangle$
				\item Anhand der Fundamentalmatrix: Symmetrisch
			\end{itemize}
			\item Positiv-definit
			\begin{itemize}
				\item Formal: $\langle v, v\rangle \geq 0$ für alle $v \in V$ und $\langle v, v\rangle = 0 \Leftrightarrow v = 0$
				\item Anhand der Fundamentalmatrix: Alle Hauptminoren sind $> 0$,\\alternativ: $v^TF(s)v \geq 0$ für alle $v \in V$
			\end{itemize}
		\end{itemize}
	
	\subsection{Winkel, Längen, Abstände}
	\section{Orthogonalsysteme}
	\subsection{Gram-Schmidt-Orthogonalisierung und -normalisierung}
	\subsection{Orthogonale Komplemente und Projektionen}
	\subsection{Iwasawa-Zerlegung}
	\section{Isometrien}
	\subsection{Isometrien}
	\subsection{Isometrie-Normalform}
	\section{Adjungierte und normale Abbildungen}
	\subsection{Adjungierte Abbildungen}
	\subsection{Selbstadjungierte Homomorphismen}
	\subsection{Normale Homomorphismen}
\end{document}