\section{Objekt-Tracking durch Partikelfilter}%
\label{track:sec:objekt_tracking_durch_partikelfilter}

\begin{itemize}
	\item \textbf{Ziel}: Verfolgung von Objekten
	\item \textbf{Idee}: Schätze nächsten ($i$-ten) Objektzustand $s_t^{(i)}$ zum Zeitpunkt $t$ aus aktuellem Zustand, überprüfe mit Sensordaten $z_t$ auf Richtigkeit
	\item \textbf{Partikel} entsprechen Zustandsvermutungen, welche anhand ihrer \textbf{Eintrittswahrscheinlichkeit} $\pi_t^{(i)}$ \textbf{gefiltert} werden
	\item \textbf{Bewertungsfunktion} $\beta(s_t^{(i)}, z_t)$ liefert Eintrittswahrscheinlichkeit $\pi_t^{(i)}$ der Zustandsvermutung
	\item Nur die $N$ wahrscheinlichsten Partikel \textbf{werden verfolgt (Performance!)}
\end{itemize}

\subsection{Algorithmus einer Filterung}%
\label{track:sub:algorithmus_einer_filterung}

\begin{enumerate}
	\item \textbf{Initialisierung}:
	\begin{itemize}
		\item $t = 0$, $N$ Partikel werden gemäß einer gewählten Konfiguration initialisiert, meist mit Wahrscheinlichkeit $\frac{1}{N}$
		\item Initialisiere leere Menge aller gefilterten Partikel
	\end{itemize}
	\item \textbf{Gewichtete Partikelwahl}:
	\begin{itemize}
		\item Partikel wird mit einer Wahrscheinlichkeit \textbf{proportional zu seiner Bewertung} aus der Menge der letzten Filterung gezogen
		\item Ein gut bewerteter Partikel wird wahrscheinlich \textbf{mehrfach gewählt}, ein schlechter womöglich \textbf{niemals}
	\end{itemize}
	\item \textbf{Neuberechnung des Partikels}:
	\begin{itemize}
		\item Berechne mögliche Zustandsänderung des Partikels anhand von z.B seiner aktuellen Geschwindigkeit
		\item \textbf{Auch}: Hinzufügen von \textbf{Rauschen} zur Abschätzung \textbf{zufälliger Bewegungen} und zur \textbf{Streuung der Partikel}
	\end{itemize}
	\item \textbf{Berechnung der Wahrscheinlichkeit}:
	\begin{itemize}
		\item Bewerte neuen Partikel anhand von Sensordaten
	\end{itemize}
	\item \textbf{Wiederholung}:
	\begin{itemize}
		\item Füge neuen Partikel der gefilterten Menge hinzu
		\item Springe zurück zu $2.$, sollten noch nicht alle $N$ Partikel generiert worden sein
	\end{itemize}
\end{enumerate}