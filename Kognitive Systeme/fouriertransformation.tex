\section{Fouriertransformationen}%
\label{four:sec:fouriertransformationen}

\begin{itemize}
	\item Jede \textbf{stückweise stetige und periodische Funktion} lässt sich als \textbf{Summe von Sinus- und Cosinusfunktionen} darstellen
	\item \textbf{Fouriertransformation} findet die passende Darstellung
	\item Nützlich für \textbf{Signalverarbeitung} (Digitalisierung, Filtern, Kompression etc.), Signal entspricht hinlänglich glatter, wertbeschränkter \textbf{Funktion}
	\item \textbf{Digitalisierung}:
	\begin{itemize}
		\item \textbf{Abtastung/Sampling} erzeugt diskrete Funktion durch (meist gleichmäßiges) Auswählen von Funktionswerten, Häufigkeit entspricht der \textbf{Abtastrate} (in Hz)
		\item \textbf{Quantisierung} rundet Messwerte für ein digitales Format, Auflösung bestimmt Qualität
		\item \textbf{Aliasing} ist ein Fehler durch Unterabtastung, wichtige Funktionswerte werden nicht abgetastet; Lösung ist höhere Abtastrate oder Anti-Aliasing-Filter
		\item \textbf{Nyquist-Shannon-Theorem}: \itquote{Ein kontinuierliches, bandbegrenztes Signal mit Grenzfrequenz F muss mit einer Frequenz \textbf{größer} als 2F abgetastet werden, damit das ursprüngliche Signal wieder rekonstruiert werden kann.}
	\end{itemize}
\end{itemize}

\subsection{Fourierreihe}%
\label{four:sub:fourierreihe}

\begin{itemize}
	\item Sei $f(t)$ eine \textbf{periodische} und \textbf{stückweise stetige} Funktion
	\item \textbf{Reelle Darstellung}: für die Funktion $f(t)$ entspricht die reelle Darstellung
	$$f(t) = \sum^\infty_{k=0}(A_k\ cos\ k\omega t + B_k\ sin\ k\omega t)\text{, wobei}$$
	$$A_k = \frac{2}{T}\int^{\frac{T}{2}}_{-\frac{T}{2}}f(t)\ cos\ k\omega t\ dt\ \mathbf{und}\ B_k = \frac{2}{T}\int^{\frac{T}{2}}_{-\frac{T}{2}}f(t)\ sin\ k\omega t\ dt$$
	mit der Periodendauer $T$ und der Kreisfrequenz $\omega = \frac{2\pi}{T}$
	\item \textbf{Komplexe Darstellung}: für die Funktion $f(t)$ entspricht die komplexe Darstellung
	$$f(t) = \sum^\infty_{k=-\infty}C_ke^{ik\omega t}\text{, wobei}\ C_k = \frac{1}{T}\int^{\frac{T}{2}}_{-\frac{T}{2}}f(t)e^{-ik\omega t}dt$$
	\item \textbf{Eigenschaften der Fourierreihe}:
	\begin{itemize}
		\item \textbf{Linearität}: Entwicklung ist lineare Operation, Summanden einer Funktion können \textbf{einzeln entwickelt und dann addiert werden}
		\item \textbf{Verschiebung der Zeit}: Verschiebung der Zeit verändert \textbf{die Amplituden der Frequenzen}
		\item \textbf{Verschiebung der Fourierkoeffizienten}: Verschiebung der Koeffizienten um $a$ multipliziert die Funktion mit $e^{-i\omega at}$
		\item \textbf{Skalierung}: Streckung auf der Zeitskala entspricht einer Stauchung auf der Frequenzskala und umgekehrt
	\end{itemize}
\end{itemize}

\subsection{Kontinuierliche Fouriertransformation}%
\label{four:sub:kontinuierliche_fouriertransformation}

\begin{itemize}
	\item Fourierreihe erzeugt Summe von Schwingungen \textbf{diskreter Frequenzen}, stattdessen nun kontinuierliche Transformation für \textbf{kontinuierliches Frequenzspektrum}
	\item \textbf{Fouriertransformation}:
	$$F(\omega) = \int^\infty_{-\infty}f(t)\ e^{-i\omega t}dt$$
	\item \textbf{Fourier-Rücktransformation}:
	$$f(t) = \frac{1}{2\pi}\int^\infty_{-\infty}F(\omega)\ e^{+i\omega t}d\omega$$
	\item \textbf{Eigenschaften der Fouriertransformation}:
	\begin{itemize}
		\item \textbf{Linearität, Zeitverschiebung, Koeffizientenverschiebung} sind \textbf{wie bei der Fourierreihe}
		\item \textbf{Skalierung}: Streckung auf der Zeitskala um $a$ entspricht Stauchung auf der Frequenzskala und der Amplitude um $a$ und umgekehrt
	\end{itemize}
	\item \textbf{Zusammenhänge von Funktion und Transformierter}:
	\begin{center}
		\begin{tabular}{r c l}
			\textbf{Funktion} 	& 					& \textbf{Transformierte}\\
			periodisch 			& $\leftrightarrow$ & diskret\\
			diskret 			& $\leftrightarrow$ & periodisch\\
			reell 				& $\leftrightarrow$ & gerade\\
			imaginär 			& $\leftrightarrow$ & ungerade
		\end{tabular}
	\end{center}
\end{itemize}

\subsection{Faltungen}%
\label{four:sub:faltungen}

\begin{itemize}
	\item Faltung zweier Funktionen kann als \textbf{Signalfilterung} verstanden werden
	\item \textbf{Faltung zweier Funktionen} im \textbf{Zeitbereich} entspricht einer \textbf{Multiplikation} ihrer Fouriertransformierten im \textbf{Frequenzbereich} und umgekehrt
	\item \textbf{Mathematisch}: $$f(t) * g(t) = \int^\infty_{-\infty}f(\xi)g(t - \xi)d\xi$$
\end{itemize}

\newpage
\subsection{Diskrete Fouriertransformation}%
\label{four:sub:diskrete_fouriertransformation}

\begin{itemize}
	\item \textbf{Praxis}: Statt bekannten kontinuierlichen Funktionen analysiert man meist $N$ \textbf{Abtastwerte} und möchte ein \textbf{diskretes, endliches Frequenzspektrum} erhalten; dies erreicht die \textbf{diskrete Fouriertransformation}
	\item \textbf{Fouriertransformation}: Für Messwerte $f[k], k = 0..N-1$ ergeben sich die Amplituden $F[\omega], \omega = 0..N-1$ mit:
	$$F[\omega] = \frac{1}{N}\sum^{N-1}_{k=0}f[k]\ e^{-i2\pi\frac{k}{N}\omega}$$
	\item \textbf{Fourier-Rücktransformation}:
	$$f[k] = \sum^{N-1}_{\omega=0}F[\omega]\ e^{+i2\pi\frac{\omega}{N}k}$$
	\item \textbf{Fast-Fourier-Transform}: Nutze Fakt, dass $F[0] = f[0]$; halbiere Messreihe (Zahl an Abtastwerten muss \textbf{Zweierpotenz} sein!) bis Länge 1 und rekombiniere die Teilergebnisse in $O(N log N)$
\end{itemize}

\subsection{Bildverarbeitung}%
\label{four:sub:bildverarbeitung}

\begin{itemize}
	\item \textbf{Idee}: Fourier-Transformierte ermöglichen einfaches Anwenden von \textbf{Frequenzfiltern} per Multiplikation (ohne Fourier wäre Faltung notwendig)
	\begin{itemize}
		\item \textbf{Idealer Filter}: Filter, welche Frequenzen außerhalb eines Bereiches komplett auslöscht
		\item \textbf{Hochpassfilter}: Erhält hohe Frequenzen, eliminiert niedrige (\textbf{Tiefpassfilter} analog)
		\item \textbf{Bandpassfilter}: Erhält nur Frequenzen innerhalb eines Intervalles
		\item \textbf{Gaußfilter}: Nicht-idealer Hoch- bzw. Tiefpassfilter, welcher Bereiche außerhalb langsam abschwächt statt sie auszulöschen
	\end{itemize}
\end{itemize}

\subsection{Funktionen und ihre Fouriertransformierten}%
\label{four:sub:funktionen_und_ihre_fouriertransformierten}

\begin{center}
	\begin{tabular}{r c l}
		$\mathbf{f(t)}$ 							& 					& $\mathbf{F(\omega)}$\\
		$e^{i\alpha t}$ 							& $\leftrightarrow$ & $\delta(\omega - \alpha)$\\
		$A\ sin(2\pi\alpha t)$ 						& $\leftrightarrow$ & $\frac{A}{2}i\delta(\omega + \alpha) - \frac{A}{2}i\delta(\omega - \alpha)$\\
		$A\ cos(2\pi\alpha t)$ 						& $\leftrightarrow$ & $\frac{A}{2}\delta(\omega + \alpha) + \frac{A}{2}\delta(\omega - \alpha)$\\
		$\sum^\infty_{n=-\infty}\delta(t - nT)$ 	& $\leftrightarrow$ & $\frac{1}{T}\sum^\infty_{n=-\infty}\delta(\omega - \frac{n}{T})$
	\end{tabular}
\end{center}