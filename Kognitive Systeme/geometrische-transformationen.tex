\section{Geometrische Transformationen}%
\label{gtrans:sec:geometrische_transformationen}

\begin{itemize}
	\item \textbf{Ziel}: Objekte/Punkte (Ortsvektoren) im Raum \textbf{transformieren}
	\item \textbf{Beispiel}: \textbf{Verschiebung (Translation) und Rotation} einer Kugel über einen Billardtisch
\end{itemize}

\subsection{Homogene Koordinaten}%
\label{gtrans:sub:homogene_koordinaten}

\begin{itemize}
	\item \textbf{Ziel}: Darstellung der gesamten Vektortransformation mit Matrizen, (unhomogenisierte) Translation z.B so jedoch nicht darstellbar
	\item \textbf{Homogenisierung eines Vektors} durch Hinzufügen eines \textbf{Skalierungsfaktors} (i.d.R. 1):
	$$p = \begin{pmatrix} p_0 \\ p_1 \\ p_2 \end{pmatrix} \xrightarrow[]{Homogenisierung} \hat{p} = \begin{pmatrix} p_0 \\ p_1 \\ p_2 \\ 1 \end{pmatrix}$$
	Der Skalierungsfaktor kann beliebig sein, solang für den nicht-homogenisierten N-Vektor $p$ und seinen homogenisierten N+1-Vektor $\hat{p}$ gilt:
	$$p_k = \frac{\hat{p}_k}{\hat{p}_{N+1}}, k = 0..N$$
\end{itemize}

\subsection{Translationen}%
\label{gtrans:sub:translationen}

\begin{itemize}
	\item \textbf{3D Translation/Verschiebung} eines homogenen Vektors $p$ um einen Vektor $u$ mit Hilfe des \textbf{Skalierungsfaktors} per Translationsmatrix $T$ möglich, \textbf{zweidimensional analog}:
	$$u = \begin{pmatrix}
		u_0 \\ u_1 \\ u_2
	\end{pmatrix} \Rightarrow T = \begin{pmatrix}
		1 & 0 & 0 & u_0\\
		0 & 1 & 0 & u_1\\
		0 & 0 & 1 & u_2\\
		0 & 0 & 0 & 1
	\end{pmatrix}
	\hspace*{1cm}T \cdot p = \begin{pmatrix}
		1 & 0 & 0 & u_0\\
		0 & 1 & 0 & u_1\\
		0 & 0 & 1 & u_2\\
		0 & 0 & 0 & 1
	\end{pmatrix} \begin{pmatrix}p_0 \\ p_1 \\ p_2 \\ 1\end{pmatrix} = \begin{pmatrix}
		p_0 + u_0 \\ p_1 + u_1 \\ p_2 + u_2 \\ 1
	\end{pmatrix}$$
\end{itemize}

\subsection{Rotationen}%
\label{gtrans:sub:rotationen}

\begin{itemize}
	\item \textbf{Rotationsmatrizen} in einer Ebene (zweidimensional) um den Winkel $\phi$ gegen den Uhrzeigersinn entsprechen:
	$$\begin{pmatrix}
		cos\phi & -sin\phi\\
		sin\phi & cos\phi
	\end{pmatrix} \xrightarrow[]{Homogenisierung} \begin{pmatrix}
		cos\phi & -sin\phi & 0\\
		sin\phi & cos\phi & 0\\
		0 & 0 & 1
	\end{pmatrix}$$
	\newpage
	\item \textbf{Dreidimensionale Rotationsmatrizen} unterscheiden sich anhand der Rotationsebene (XY, XZ, YZ), jede beliebige Rotation kann als \textbf{Kombination dieser Rotationen} ausgedrückt werden:
	$$R_x(\phi) = \begin{pmatrix}
		1 & 0 & 0\\
		0 & cos\phi & -sin\phi\\
		0 & sin\phi & cos\phi
	\end{pmatrix} \xrightarrow[]{Homogenisierung} \begin{pmatrix}
		1 & 0 & 0 & 0\\
		0 & cos\phi & -sin\phi & 0\\
		0 & sin\phi & cos\phi & 0\\
		0 & 0 & 0 & 1
	\end{pmatrix}$$
	$$R_y(\phi) = \begin{pmatrix}
		cos\phi & 0 & sin\phi\\
		0 & 1 & 0\\
		-sin\phi & 0 & cos\phi
	\end{pmatrix} \xrightarrow[]{Homogenisierung} \begin{pmatrix}
		cos\phi & 0 & sin\phi & 0\\
		0 & 1 & 0 & 0\\
		-sin\phi & 0 & cos\phi & 0\\
		0 & 0 & 0 & 1
	\end{pmatrix}$$
	$$R_z(\phi) = \begin{pmatrix}
		cos\phi & -sin\phi & 0\\
		sin\phi & cos\phi & 0\\
		0 & 0 & 1
	\end{pmatrix} \xrightarrow[]{Homogenisierung} \begin{pmatrix}
		cos\phi & -sin\phi & 0 & 0\\
		sin\phi & cos\phi & 0 & 0\\
		0 & 0 & 1 & 0\\
		0 & 0 & 0 & 1
	\end{pmatrix}$$
	\item \textbf{Eigenschaften}: Jede Rotationsmatrix $R$ ist orthogonal, d.h. $R^T = R^{-1}$
	\item \textbf{Hinweis}: Transformationsmatrizen (Translationen, Rotationen etc.) lassen sich per Multiplikation kombinieren, die resultierende Matrix entspricht einer \textbf{Durchführung aller Transformationen von rechts nach links}; Transformationen sind \textbf{nicht immer eindeutig!}
\end{itemize}

\subsection{Quaternionen}%
\label{gtrans:sub:quaternionen}

\begin{itemize}
	\item \textbf{Problem}: Rotationsmatrizen sind \textbf{ineffizient} und können Probleme wie z.B das \textbf{Gimbal Lock} hervorrufen
	\item \textbf{Lösung}: \textbf{Quaternionen} definieren Rotationen anhand einer normierten \textbf{Rotationsachse} und dem \textbf{Rotationswinkel}
	\item Jede Rotation hat genau \textbf{zwei Quaternionen (im und entgegen dem Uhrzeigersinn)}
	\item \textbf{Eigenschaften}:
	\begin{itemize}
		\item Darstellung als Vektor oder analog zu den \textbf{komplexen Zahlen}: $h = h_0 + ih_1 + jh_2 + kh_3$
		\item $h_0$ ist der Realteil, $h_1, h_2, h_3$ der Imaginärteil, wobei:
		\begin{enumerate}
			\item $i^2 = j^2 = k^2 = -1$
			\item $ijk = -1$
		\end{enumerate}
	\end{itemize}
	\newpage
	\item \textbf{Operationen}:
	\begin{itemize}
		\item Alle Quaternion-Operationen erfüllen das \textbf{Distributivgesetz}
		\item \textbf{Addition}: Erfolgt komponentenweise, ist assoziativ und kommutativ:
		$$h + \hat{h} = (h_0 + \hat{h}_0) + i(h_1 + \hat{h}_1) + j(h_2 + \hat{h}_2) + k(h_3 + \hat{h}_3)$$
		\item \textbf{Multiplikation}: Erfolgt durch Ausmultiplizieren und Vereinfachen der Ausdrücke, ist assoziativ aber nicht kommutativ
		\item \textbf{Multiplikatives Inverses}:
		$$h^{-1} = (\frac{h_0}{l} - i\frac{h_1}{l} - j\frac{h_2}{l} - k\frac{h_3}{l}),\ l = h_0^2 + h_1^2 + h_2^2 + h_3^2$$
		\item \textbf{Konjugiertes Quaternion} $\bar{h}$ entspricht dem Quaternion mit \textbf{invertierten Vorzeichen} der \textbf{imaginären} Komponenten
		\item \textbf{Betrag} eines Quaternions analog zu Vektoren: $$|h| = \sqrt{h\bar{h}} = \sqrt{h_0^2 + h_1^2 + h_2^2 + h_3^2}\ (d.h.\ h^{-1} = \frac{\bar{h}}{|h|^2})$$
		\item \textbf{Übliche Schreibweise}: $(s, v)$ mit $s$ als reelle Komponente und $v$ als Vektor der imaginären Komponente (z.B $(1, 0)$)
	\end{itemize}
	\item \textbf{Quaternion zu Rotationsmatrix}: Sei $q = (s, (x, y, z))$, dann gilt:
	$$R_q = \begin{pmatrix}
		1 - 2(y^2 + z^2) & 2xy - 2sz & 2sy + 2xz\\
		2xy + 2sz & 1 - 2(x^2 + z^2) & -2sx + 2yz\\
		-2sy + 2xz & 2sx - 2yz & 1 - 2(x^2 + y^2)
	\end{pmatrix}$$
	\item \textbf{Rotationsmatrix zu Quaternion}: Sei $R$ eine Rotationsmatrix mit Einträgen $r_{i,j}, j \in \{1, 2, 3\}$, dann gilt:
	$$s = \frac{1}{2}\sqrt{1 + r_{11} + r_{22} + r_{33}}$$
	$$x = \frac{r_{32} - r_{23}}{4s}$$
	$$y = \frac{r_{13} - r_{31}}{4s}$$
	$$z = \frac{r_{21} - r_{12}}{4s}$$
\end{itemize}
\newpage
\subsection{Kameramodell}%
\label{gtrans:sub:kameramodell}

\begin{itemize}
	\item \textbf{Optische Achse}: Gerade durch Projektionszentrum, senkrecht zur Bildebene
	\item \textbf{Bildhauptpunkt}: $C(c_x, c_y)$, Schnittpunkt der optischen Achse mit der Bildebene
	\item \textbf{Bildkoordinatensystem}: Ursprung in der linken oberen Ecke des Bildes, $u, v$ jeweils positiv nach rechts unten
	\item \textbf{Kamerakoordinatensystem}: Ursprung im Projektionszentrum, Achsen parallel zu den Achsen des Bildkoordinatensystems, $z$-Achse nach vorne
	\item \textbf{Weltkoordinatensystem}: Basiskoordinatensystem beliebig im Raum
	\item \textbf{Brennweite}: Pixel sind nicht zwingend quadratisch, Brennweite dargestellt als $f_x, f_y$
	\item \textbf{Kalibriermatrix}:
	$$
		K = \begin{pmatrix}
			f_x & 0 & c_x \\ 0 & f_y & c_y \\ 0 & 0 & 1
		\end{pmatrix}
	$$
	\item \textbf{Intrinsische Kamerakalibrierung} (Projektion von Kamerakoordinaten zu Bildkoordinaten):
	$$
		\begin{pmatrix}u\\v\end{pmatrix} = \begin{pmatrix}c_x\\c_y\end{pmatrix} + \frac{1}{Z}\begin{pmatrix}f_x \cdot X\\f_y \cdot Y\end{pmatrix}
	$$
	\item \textbf{Extrinsische Kamerakalibrierung} (Projektion einer Kamera, welche in der Welt bewegt wurde):
	\begin{itemize}
		\item Koordinatentransformation bestehend aus Rotation $R$ und Translation $t$:
		$$
			x_c = Rx_w + t
		$$
	\end{itemize}
	\item \textbf{Epipolargeometrie}:
	\begin{itemize}
		\item Beschreibt \textbf{Zusammenhang zwischen Kameras}
		\item \textbf{Epipole}: Schnittpunkte der Gerade durch die Projektionszentren mit den Bildebenen
		\item \textbf{Epipolarebene}: Ebene aufgespannt durch Szenenpunkt $X$ sowie die Projektionszentren $C, C'$
		\item \textbf{Fundamentalmatrix}: Mathematische Beschreibung der Epipolargeometrie
		\item \textbf{Berechnung von F über Essentialmatrix}: Kamera $1$ habe die Identität als Transformation, Kamera $2$ habe $(R | t)$:
		$$
			E = \begin{pmatrix}
				0 & -t_3 & t_2 \\ t_3 & 0 & -t_1 \\ -t_2 & t_1 & 0
			\end{pmatrix}R,\ F = K'^{-T}EK^{-1}
		$$
	\end{itemize}
\end{itemize}