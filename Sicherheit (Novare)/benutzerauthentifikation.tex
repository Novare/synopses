\section{Benutzerauthentifikation}%
\label{benauth:sec:benutzerauthentifikation}

\begin{itemize}
	\item \textbf{Ziel}: Sichere Authentifizierung von Benutzern
	\item \textbf{Lösung}: Passwörter; sehr verbreitet, benötigen aber wichtige Sicherheitsmaßnahmen
	\item \textbf{Klartext ist unsicher}, deshalb: Keine Übertragung oder serverseitige Speicherung des Klartextpassworts, stattdessen \textbf{Hashing des Passworts bei Übertragung und Speicherung}, dann aber immer noch verwundbar bzgl. Man-in-the-Middle!
	\item \textbf{Wörterbuchangriffe}: Passwörter aus natürlichsprachlichen Worten können mit \quotestyle{Wörterbücher} durchgetestet und damit gebrochen werden
	\item \textbf{Brute-Force-Angriffe}: Schwer gewählte Passwörter können aufwendig durch Durchtesten erraten werden; Hashtabellen aller möglichen Passwörter beschleunigen die Suche, sind aber speicherintensiv und man \textbf{benötigt die genutzte Hashfunktion}
	\item \textbf{Kompression von Hashtabellen}:
	\begin{itemize}
		\item \quotestyle{Time-Memory-Tradeoff} durch Kompression der Tabelle
		\item \textbf{Problem}: Hashwerte sind schwer zu komprimieren da nahezu zufällig
		\item \textbf{Lösung}: Hashketten
	\end{itemize}
	\item \textbf{Hashketten}:
	\begin{itemize}
		\item \textbf{Idee}: Wähle erstes Passwort, hashe dieses, nutze \textbf{Reduktionsfunktion f} um aus dem Hash das nächste mögliche Passwort im Passwortraum zu generieren usw.
		$$
			pw_1 \xrightarrow{H} H(pw_1) \xrightarrow{f} pw_2 \xrightarrow{H} H(pw_2) \xrightarrow{f} pw_3 \dots
		$$
		\item Speichere $n$ Hashketten der Länge $m$; eine Hashkette ist durch das erste Passwort und den letzten Hashwert ausreichend definiert
		\item Optimiere Time-Memory-Tradeoff durch Verlängerung der Kettenlänge bzw. Erhöhung der Kettenzahl
		\item \textbf{Problem}: Ketten decken womöglich nicht den gesamten Passwortraum ab; weiter laufen Ketten womöglich zusammen
	\end{itemize}
	\item \textbf{Rainbow Tables}:
	\begin{itemize}
		\item \textbf{Idee}: Nutze $m - 1$ Reduktionsfunktionen statt einer:
		$$
			pw_1 \xrightarrow{H} H(pw_1) \xrightarrow{f_1} pw_2 \xrightarrow{H} H(pw_2) \xrightarrow{f_2} pw_3 \dots			
		$$
		\item \textbf{Vorteil}: Kein Zusammenlaufen von Ketten, solange die Kollision an verschiedenen Stellen in den Hashketten auftreten; Passwortabdeckung ist i.d.R. auch besser als bei reinen Hashketten
	\end{itemize}
	\item \textbf{Verteidigungsmaßnahmen}:
	\begin{itemize}
		\item \textbf{Salted Hashes}: Jeder Nutzer bekommt ein individuelles \quotestyle{Salz}, z.B ein Zufallsstring, der Passworthash ist dann $H(s||pw)$
		\item \textbf{Key Strengthening}: Mehrfaches Hashing des Passworts
	\end{itemize}
\end{itemize}