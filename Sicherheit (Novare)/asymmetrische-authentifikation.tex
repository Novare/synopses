\section{Asymmetrische Authentifikation von Nachrichten}%
\label{asauth:sec:asymmetrische_authentifikation_von_nachrichten}

\begin{itemize}
	\item \textbf{Problem}: Symmetrische Authentifikation hat ähnliche Probleme wie symmetrische Verschlüsselung
	\item \textbf{Lösung}: Asymmetrische Authentifikation; Signieren mit privatem Schlüssel $pk$ und Verifizieren mit öffentlichem Schlüssel $pk$ $(pk, sk) \leftarrow KEYGEN(1^k)$
	\item \textbf{Asymmetrische EUF-CMA-Sicherheit}: PPT-Angreifer $A$ muss das folgende Spiel mit in $k$ vernachlässigbarer Wahrscheinlichkeit gewinnen:
	\begin{enumerate}
		\item $A$ generiert Schlüsselpaar $(pk, sk)$, erhält Signaturorakel und den zugehörigen öffentlichen Schlüssel
		\item $A$ darf per Orakel polynomiell viele Nachrichten signieren, gibt dann eine potentielles gefälschtes Nachrichten-Signatur-Paar aus
		\item $A$ gewinnt, ist das Nachrichten-Signatur-Paar gültig und die enthaltene Nachricht nicht über das Orakel signiert worden
	\end{enumerate}
	\item \textbf{RSA-Signaturen}:
	\begin{itemize}
		\item Analog zur Verschlüsselung können Nachrichten signiert und verifiziert werden
		\item \textbf{Probleme}:
		\begin{itemize}
			\item Angreifer kann beliebige Signatur wählen und dazu passende Nachricht generieren; Versendung unsinniger Nachrichten
			\item Homomorphie von RSA bedeutet, dass Angreifer aus zwei Signaturen zu zwei Nachrichten $M_1, M_2$ eine Signatur für die Nachricht $M_3 = M_1 \cdot M_2\ mod\ N$ bauen kann; hiermit kann man das EUF-CMA-Experiment gewinnen; lösbar mit \textbf{RSA-OAEP}, damit entsteht das Signaturverfahren \textbf{RSA-PSS} (Probabilistic Signature Scheme)
		\end{itemize}
	\end{itemize}
	\item \textbf{ElGamal-Signaturen}:
	\begin{itemize}
		\item \textbf{Schlüsselwahl}: $sk = (G, g, x), pk = (G, g, g^x)$
		\item \textbf{Signaturerzeugung}:
		\begin{itemize}
			\item Wähle zufällige, in den ganzen Zahlen invertierbare Zahl $e \in \{1, \dots, p - 1\}, p - 1 = |G|$
			\item \textbf{Signatur}: $SIG(sk, M) = (a, b)$ mit
			\begin{align*}
				a &:= g^e \in G\\
				b &:= (M - a \cdot x) \cdot e^{-1}\ mod\ |G|
			\end{align*}
			\item \textbf{Verifikation}: $VER(pk, \sigma, M) = 1 \Leftrightarrow v_1 = v_2$ mit
			\begin{align*}
				v_1 &= (g^x)^a \cdot a^b\\
				v_2 &= g^M
			\end{align*}
		\end{itemize}
		\item \textbf{Probleme}:
		\begin{itemize}
			\item Doppelte Verwendung von $e$ ermöglicht Berechnung des geheimen Schlüssels; sollte vermieden werden
			\item Günstige Parameterwahl ermöglicht Erzeugung gültiger \quotestyle{unsinniger} Signaturen ohne Schlüsselkenntnis
		\end{itemize}
	\end{itemize}
	\item \textbf{Hash-Then-Sign}: Analog zu symmetrisch, ermöglicht Signieren von Nachrichten beliebiger Länge
	\item \textbf{DSA (Digital Signature Algorithm)}:
	\begin{itemize}
		\item Hash-Then-Sign-Paradigma + ElGamal-Signaturverfahren + kollisionsresistente Hashfunktion $H$
		\item Zweitwichtigster Signaturalgorithmus (nach RSA)
	\end{itemize}
	\item \textbf{Digitale Zertifikate}:
	\begin{itemize}
		\item \textbf{Public Key Infrastructures (PKIs)} ermöglichen Bestätigung der Integrität des öffentlichen Schlüssels
		\item \textbf{Certificate Authority} stellen Zertifikate aus, welche Aussteller, Schlüssel und Inhaber assoziieren und mit dem privaten Schlüssel des Inhabers signiert sind
		\item \textbf{Zertifikate} stellen sicher, dass ein Schlüssel tatsächlich zu einer Person/Institution gehört
		\item \textbf{Alle} Certificate Authorities müssen \textbf{vertrauenswürdig} sein, damit das System funktioniert
	\end{itemize}
	\item \textbf{X.509}:
	\begin{itemize}
		\item \textbf{Zertifikatsstandard}
		\item Unterteilt Zertifikat in \textbf{Data-Block} (eigentliche Zertifikatsdaten) und den \textbf{Signature-Algorithm}, der das Signaturverfahren enthält
	\end{itemize}
\end{itemize}