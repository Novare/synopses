\section{Grundlagen}%
\label{grund:sec:grundlagen}

\begin{itemize}
	\item \textbf{Betriebssicherheit (Safety)}: Sicherheit, die das System etabliert (z.B Backup-Systeme, elektrische Sicherungen)
	\item \textbf{Angriffssicherheit (Security)}: Sicherheit bzgl. externer Einwirkungen (z.B Türschlösser, Wasserzeichen)
	\item \textbf{Verschlüsselung (Encryption)}: Verschlüsselungsverfahren $ENC$ (Chiffren) übersetzen mit Schlüsseln $K$ Klartext $M$ (Nachricht) in Geheimtext $C$ (Chiffrat), um Information geheim zu halten
	\item \textbf{Entschlüsselung (Decryption)}: Entschlüsselungsverfahren $DEC$ dechiffrieren $C$ zurück zu $M$
	\item \textbf{Geheime Verfahren}: Geheimhaltung ohne Schlüssel, stattdessen geheimer Algorithmus; nur historisch relevant, da unsicher
	\item \textbf{Kerckhoffs Prinzip}: Sicherheit einer Chiffre darf nur vom Schlüssel, nicht vom Algorithmus abhängig sein
\end{itemize}