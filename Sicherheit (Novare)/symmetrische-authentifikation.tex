\section{Symmetrische Authentifikation von Nachrichten}%
\label{symauth:sec:symmetrische_authentifikation_von_nachrichten}

\begin{itemize}
	\item \textbf{Ziel}: Bestätigung, dass eine erhaltene Nachricht nicht verändert wurde (\textbf{Integrität}) und vom richtigen Absender stammt (\textbf{Authentizität})
	\item \textbf{Lösung}: \quotestyle{Unterschrift} $\sigma$ zusätzlich zur eigentlichen Nachricht
	\item \textbf{Message Authentication Code (MAC)}:
	\begin{itemize}
		\item Symmetrisches Verfahren für Signatur- und Verifikation einer Nachricht $M$ mit einem gemeinsamen Geheimnis $K$ (meist zufällig gewählter Bitstring, analog zu Verschlüsselung)
		\item \textbf{Signatur}: $\sigma \leftarrow SIG(K, M)$
		\item \textbf{Verifikation}: $VER(K, M, \sigma) \in \{0, 1\}$
		\item \textbf{Korrektheit}: $\forall M \forall K: VER(K, M, SIG(K, M)) = 1$
	\end{itemize}
	\item \textbf{Symmetrische EUF-CMA-Sicherheit}:
	\begin{itemize}
		\item Kein PPT-Angreifer sollte selbstständig ein gültiges \textbf{Nachrichten-Signatur-Paar} finden können
		\item \textbf{EUF-CMA-Sicherheit} ist gegeben, falls jeder PPT-Algorithmus $A$ folgendes Spiel mit im Sicherheitsparameter $k$ vernachlässigbarer Wahrscheinlichkeit gewinnt:
		\begin{enumerate}
			\item A erhält \textbf{Signaturorakel}, mit dem er polynomiell beschränkt viele Nachrichten signieren lassen kann
			\item A gibt als potentielle Fälschung ein Nachrichten-Signatur-Paar aus
			\item A gewinnt, wenn er eine gültige Signatur für eine Nachricht ausgegeben hat, die er nicht in Schritt 1. hat signieren lassen
		\end{enumerate}
	\end{itemize}
	\item \textbf{Hash-then-Sign Paradigma}: Signiere Hashwert einer Nachricht statt die Nachricht selbst; Signaturverfahren signieren nur Nachrichten fester Länge, Hashing-Verfahren schaffen Abhilfe
	\item \textbf{Pseudorandomisierte Funktionen}: MAC-Berechnung sollte keine Regelmäßigkeiten aufweisen; pseudorandomisierte Funktionen $PRF$ verteilen Funktionswerte zufällig auf Urbilder (z.B durch Hash-Funktionen); \textbf{aber}: basieren diese auf Merkle-Damgard-Konstruktionen, gibt es oft Probleme mit EUF-CMA-Sicherheit!
	\item \textbf{Keyed-Hash MAC (HMAC)}:
	\begin{itemize}
		\item Basiert auf Merkle-Damgard-Konstruktion (Hashfunktion $H$), aber erzeugt EUF-CMA-sichere pseudorandomisierte Funktionen für MACs
		\item $ipad$ und $opad$ sind das $inner$ bzw. $outer\;padding$, zwei Konstanten die bei jedem Signaturvorgang identisch bleiben
		\item \textbf{Signatur}: $SIG(K, M) = H((K \oplus \textit{opad})||H((K \oplus \textit{ipad})|| M))$
		\item \textbf{Verifikation}: $VER(K, M, \sigma) = 1 \Leftrightarrow \sigma = H((K \oplus \textit{opad})|| H((K \oplus \textit{ipad})|| M))$
	\end{itemize}
\end{itemize}