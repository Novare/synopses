\section{Schwingungen \& Wellen}%
\label{schwellen:sec:schwellen}

\subsection{Schwingungen}%
\label{schwellen:sub:schwingungen}

\begin{itemize}
	\item \textbf{Schwingung}: Sich wiederholende zeitliche Schwankung einer Zustandsgröße eines Systems
	\item \textbf{Harmonische Schwingung}: Schwingung kann durch Sinusfunktion ausgedrückt werden
	\item \textbf{Allgemeine Schwingungsgleichung} ($x_0, \phi, A, B$ vorgegeben durch Randbedingungen):
	\begin{align*}
		x(t) &= x_0 + sin(\omega t + \phi)\\
			 &= Ae^{i\omega t} + Be^{-i\omega t}\\
			 &= A(cos(\omega t) + isin(\omega t)) + B(cos(\omega t) - isin(\omega t))
	\end{align*}
	\item \textbf{Bewegungsgleichung aus Energiebilanz}:
	\begin{equation}
		E_{pot} + E_{kin} = \frac{1}{2}kx^2 + \frac{1}{2}m\dot{x} = const. \Rightarrow \frac{d}{dt}(\frac{1}{2}kx^2 + \frac{1}{2}m\dot{x}) = 0 \Rightarrow \dot{x}(kx + m\ddot{x}) = 0
	\end{equation}
	\item \textbf{Mathematisches Pendel} (Masse $m$ an Seil der Länge $l$ mit Auslenkung $\phi$):
	\begin{equation}
		l\ddot{\phi} + g\phi = 0
	\end{equation}
	\item \textbf{Gedämpfte Schwingung} (Schwingung mit Reibung, $b, k$ Parameter der dämpfenden Kraft):
	\begin{equation}
		\ddot{x} + \frac{b}{m}\dot{x} + \frac{k}{m}x = 0
	\end{equation}
	\item \textbf{Erzwungene Schwingung} (einer Masse an einer Feder, die mit einer weiteren Feder an einem Rad mit Kreisfrequenz $\omega$ befestigt ist):
	\begin{equation}
		\ddot{x} + \frac{b}{m}\dot{x} + \frac{k}{m}x = F_0cos(\omega t)
	\end{equation}
	\item \textbf{Gekoppelte Schwingung} (Pendel an Pendel): Gesamtschwingung ist Superposition der Eigenschwingungen
\end{itemize}

\subsection{Wellenausbreitung und Wellengleichung}%
\label{schwellen:sub:wellenausbreitung_und_wellengleichung}

\begin{itemize}
	\item \textbf{Welle}: Sich räumlich ausbreitende Schwingung
	\item \textbf{Frequenz} (Einheit: \textbf{Herz [Hz]}, mit Periodendauer $T$):
	\begin{equation}
		f = \frac{Schwingungen}{\Delta Zeit} = \frac{1}{T}
	\end{equation}
	\item \textbf{Kreisfrequenz}:
	\begin{equation}
		\omega = 2\pi f
	\end{equation}
	\item \textbf{Wellenzahl} (mit Wellenlänge $\lambda$)
	\begin{equation}
		k = \frac{2\pi}{\lambda}
	\end{equation}
	\item \textbf{Phasengeschwindigkeit}:
	\begin{equation}
		v_{ph} = \frac{\omega}{k}
	\end{equation}
	\item \textbf{Gruppengeschwindigkeit}:
	\begin{equation}
		v_g = \frac{d\omega}{dk}
	\end{equation}
	\item \textbf{Longitudinalwelle}: Bewegung senkrecht zur Ausbreitungsrichtung
	\item \textbf{Transversalwelle}: Bewegung entlang der Ausbreitungsrichtung
	\item \textbf{Wellenausbreitung}:
	\begin{equation}
		y(x, t) = Asin(\omega t - kx)
	\end{equation}
	\item \textbf{Wellengleichung für elektromagnetische Wellen}:
	\item \textbf{Energie und Impuls einer elektromagnetischen Welle}:
	\item \textbf{Fouriertheorem}:
	\item \textbf{Fourieranalyse einer periodischen Bewegung}:
	\item \textbf{Fourierintegral für nichtperiodische Bewegungen}:
	\item \textbf{Unschärfebeziehungen}:
\end{itemize}

\subsection{Interferenz und Beugung}%
\label{schwellen:sub:interferenz_und_beugung}

\begin{itemize}
	\item \textbf{Reflexion und Brechung}:
	\item \textbf{Superposition (Interferenz)}:
	\item \textbf{Interferenz von zwei synchron emittierten Wellen}:
	\item \textbf{Interferenz von N synchron emittierten Wellen}:
	\item \textbf{Stehende Wellen}:
	\item \textbf{Beugung}:
	\item \textbf{Beugung am Einfachspalt}:
	\item \textbf{Beugung am Doppelspalt}:
	\item \textbf{Beugungsgitter (N Spalte)}:
	\item \textbf{Röntgenbeugung an Kristallen}:
\end{itemize}